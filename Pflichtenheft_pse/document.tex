%%This is a very basic article template.
%%There is just one section and two subsections.
\documentclass[parskip=full]{scrartcl}
\usepackage[utf8]{inputenc}
\usepackage[T1]{fontenc}
\usepackage[ngerman]{babel}



\begin{document}

\title{Pflichtenheft \\
        \large Automatische Einteilung für das PSE}

\author{D. Biesinger, E.Dohse, P. Loth L. Seufert, Sam}
        
\maketitle
\vfill

\tableofcontents


\section{Zielbestimmung}


\subsection{Musskriterien}

\subsection{Wunschkriterien}
Plain text.

\subsection{Abgrenzungskriterien}

\section{Produkteinsatz}

\subsection{Anwendungsbereiche}

\subsection{Zielgruppe}

\subsection{Betriebsbedingungen}

\section{Produktumgebung}

\subsection{Software}

\subsection{Hardware}
// hier überlegen ob man noch Orgware und Produkt-Schnittstellen dazu noimmt
\section{Funktionale Anforderungen}

\subsection{Funktionsübersicht}

\section{Produktdaten}

\subsection{Musskriterien} // hier noch die unterpunkte

\section{Nichtfunktionale Anforderungen}

\section{Globale Testfälle}

\section{Systemmodelle}

\subsection{Szenarien}

\subsection{Anwendungsfälle}

\subsection{Objektmodelle}

\subsection{Dynamische Modelle}

\subsection{Benutzerschnittstelle}


\end{document}
