\documentclass[parskip=full]{scrartcl}
\usepackage[utf8]{inputenc}
\usepackage[T1]{fontenc}
\usepackage[ngerman]{babel}
\usepackage{enumitem}

\newcommand{\swtLabel}[1]{\textbf{\textbackslash #1\arabic*0\textbackslash}}

\begin{document}

\title{Pflichtenheft \\
        \large Automatische Einteilung für das PSE}

\author{D. Biester, E.Dohse, P. Faller, P. Loth, L. Seufert, Sam}
        
\maketitle
 

\tableofcontents
 

\section{Zielbestimmung}
Wissenschaftliche Mitarbeiter des IPD sollen durch das Produkt die Verwaltung
der Projekte und die Einteilung zur PSE rechnerunterstützt durchzuführen können.


\subsection{Musskriterien}
 \begin{enumerate}[label=\swtLabel{M}]
   \item Datenerfassung: Studierende sollen auf einer Internetseite:   
   \begin{itemize}
     \item ihre Kontaktdaten bestehend aus Name und E-mail,
     \item ihre Lerngruppen von 2 bis 6 Personen,
     \item ihre Themenvorlieben 
     \item sowie ihre erbrachten Studienleistungen 
   \end{itemize}
   in das System eingeben können.
   \item Einteilung: Das Problem der Zuteilung von Studierenden zu PSE-Teams
   soll geeignet modelliert und gelöst werden. 
   \begin{itemize}
     \item Wer die Voraussetzungen nicht erfüllt wird nicht eingeteilt
     \item Lerngruppen sollten zusammenbleiben.
     \item Präferenzen der Studierenden berücksichtigen.
   \end{itemize}
   \item Abbrechen der Berechnung der Zuteilung möglich (manuell oder Timeout)
   \item Nachjustieren der Einteilung “von Hand”
   \item Berechnung von Gütekriterien (Studierenden-Happiness,
   Anteil-Nicht-Eingeteilter, etc.)
   \item Export/Import der Aus- und Eingabe der Studierendendaten und
   Einteilung(en)
   \item Möglichkeit der sukzessiven Eintragung von angebotenen Themen und
   Anzahl der Plätze und Teams, durch die jeweiligen Projektleiter
   \item GUI zur einfacheren Bedienung    
 \end{enumerate}


\subsection{Wunschkriterien}
\begin{enumerate}[label=\swtLabel{W}]
    \item Möglichkeit des mehrmaligen Wechselns zwischen Lerngruppen und \\
    Einzelbewertungen
    \item Verwaltung von Lerngruppen: Mitgliederverwaltung und Anpassung der
    Gruppengröße
    \item Wunschkriterien zur Lösung des Problems der Einteilung:
    \begin{itemize}
        \item Eher 5er Teams als 6er Teams
        \item Möglichst gleiche Semester im Team.
        \item Studierende bevorzugen, die bereits mehr Sachen aus dem ersten Jahr
        bestanden haben. (über die Voraussetzungen hinaus, z.B. Algo)
    \end{itemize}    
    \item Authentifizierung via Shibboleth
    \item Benachrichtigung der Studierenden und Mitarbeiter über Einteilung
    \item Verwaltung mehrerer Einteilungsergebnisse
    \item Erweiterbarkeit bzgl. “Betreuer-Sicht”: Themenerfassung, -pflege und Notenerfassung
    \item Nachjustierbarkeit bei Änderungen, z.B. der SPO.
    \item Verwaltung über mehrere Semester hinweg
    \item Anzeige der Projekt-Details für Studenten
    \item Stapelsystem zur Berechnungen verschiedener Einteilungen mit unterschiedlichen Konfigurationen.
    \item Verifikation der E-Mail-Adresse
    \item Handhabung von Formularen des Campus Management Portals
    
    
    
\end{enumerate}

\subsection{Abgrenzungskriterien}
\begin{enumerate}[label=\swtLabel{A}]
 
  \item Die Produkt ist nicht für die Nutzung außerhalb des KIT's gedacht.

\item Die Einteilungsfunktion ist nur für die Einteilung zum PSE gedacht, nicht
für die Vergabe sonstiger Praktikums- oder Tutorienplätze.
  
\end{enumerate}
\section{Produkteinsatz}
\begin{itemize}
  \item Das Produkt dient zur Einpflegung von Projekten und zur Zuteilung von
Studierenden zu den Projekten.
\end{itemize}

\subsection{Anwendungsbereiche}

\begin{itemize} 
  \item Anwendung im Universitären Bereich. 
\end{itemize}

\subsection{Zielgruppe}
\begin{itemize} 
  \item Studierende 
  \item Mitarbeiter des IPD 
\end{itemize}

\subsection{Betriebsbedingungen}
\begin{itemize} 
  \item Der Betrieb erfolgt über eine Web-Schnittstelle.
\end{itemize}
\section{Produktumgebung}
\begin{itemize} 
  \item Das Produkt läuft auf einem Server des IPD.
\end{itemize}
\subsection{Software}
\begin{itemize} 
  \item Betriebssystem: Linux 64Bit
  \item ILP- Solver Gurobi
  \item Play - Framework
\end{itemize}
\subsection{Hardware}
\begin{itemize} 
  \item 16 GB Ram
  \item Intel I7 Quadcore mit 4 weiterne virtuellen Kernen 
\end{itemize}
// hier überlegen ob man noch Orgware und Produkt-Schnittstellen dazu noimmt
\section{Funktionale Anforderungen}

\subsection{Funktionsübersicht}

\subsection{Studentenfunktionen}

Pflichtfunktionen

\begin{enumerate}[label=\swtLabel{FA}]
  \item Registrierung mit Datenerfassung:
  \begin{itemize}
    \item Name, Matrikelnummer, Email-Adresse, Semester
    \item Auswahl bestandener Teilleistungen und der SPO
  \end{itemize}
  \item Anmeldung
  \item Bewertung der Themen
  \item Erstellung einer Lerngruppe mit Name und Passwort
  \item Bewertung der Themen für die Lerngruppe
  \item Beitritt zu einer Lerngruppe
  \item Übersicht eigene Lerngruppe (Bewertung und Mitglieder)
\end{enumerate}

\subsection{Betreuerfunktionen}

Wunschfunktionen

\begin{enumerate}[label=\swtLabel{FA}, resume]
  \item Anmeldung
  \item Erstellung eines Projektes
  \item Ändern der Projektdetails: Name, Beschreibung, Projektbetreuer, minimale und maximale Teilnehmerzahl, Anzahl der Teams
\end{enumerate}

\subsection{Adminfunktionen}

Pflichtfunktionen

\begin{enumerate}[label=\swtLabel{FA}, resume]
  \item Anmeldung
  \item Setzen der Anmelde- und Bewertungsdeadline
  \item Konfiguration der Einteilungparameter
  \item Starten der Einteilungsberechnung
  \item Abbrechen der Einteilungsberechnung
  \item Übersicht über die aktuelle Einteilung (Gütekriterien, ...)
  \item Manuelle Änderung der Einteilung
  \item Im- und Export der Einteilung
\end{enumerate}

Wunschfunktionen

\begin{enumerate}[label=\swtLabel{FA}, resume]
  \item Hochladen einer Liste von Studenten, die zum PSE angemeldet sind und setzen der bestandenen Teilleistungen
  \item Hinzufügen/Löschen von Studenten und Bearbeitung der Daten
  \item Hinzufügen von Berechnungen zur Stapelverarbeitung
\end{enumerate}

\section{Produktdaten}

\subsection{Projektdaten} 
Über ein Projekt sind folgende Daten zu speichern:
\begin{enumerate}[label=\swtLabel{D}] 
  \item Name,
  \item Anzahl der Teams,
  \item Beschreibung,
  \item Namen und E-Mail Adressen der Betreuer
\end{enumerate}
\subsection{Studierenden-Daten} 
Zu jedem Studierenden sollen folgende Daten gespeichert werden:
\begin{enumerate}[label=\swtLabel{D}, resume] 
  \item E-Mail Adresse
  \item Name
  \item Matrikelnummer
  \item Semester
  \item SPO
  \item bestandene Teilleistungen
  \item Falls keiner Lerngruppe beigetreten: Bewertung, welchem Projekt er oder
  sie am liebsten zugeteiilt würde
  \item Falls einer Lerngruppe beigetreten: Die zugehörige Lerngruppe.
\end{enumerate}
\subsection{Lerngruppen-Daten} 
\begin{enumerate}[label=\swtLabel{D}, resume] 
  \item Bewertung der Projekte, zu welchem die Lerngruppenmitglieder
  (Studierenden) am liebsten zugeteilt würden.
  \item Mitglieder der Lerngruppe
\end{enumerate}

\section{Nichtfunktionale Anforderungen}
\begin{enumerate}
  \item 
\end{enumerate}
\section{Globale Testfälle}
\begin{enumerate}
  \item 
\end{enumerate}
\section{Systemmodelle}

\subsection{Szenarien}

\subsection{Anwendungsfälle}

\subsubsection{Student}

\begin{enumerate}[label=\swtLabel{S}]
	\item
    \begin{description}
  	\item[Anwendung:] Registrierung eines Studenten
  	\item[Ziel:] Speichern der Studentendaten in der Datenbank
  	\item[Vorbedingung:] Studentenregistrierung wurde freigeschaltet
  	\item[Nachbedingung(Erfolg):] Erfolgreiche Registrierung des Studenten
  	\item[Nachbedingung(Fehlschlag):] Student ist weiterhin nicht registriert
  	\item[Akteure:] Student
  	\item[Auslösen des Ereignisses:] Student will sich registrieren
  	\item[Beschreibung:]~
  	\begin{enumerate}
  	  \item Öffnen der Internetseite
      \item Klick auf Registrieren
      \item Ausfüllen der Registrierungsmaske
      \item Abschicken der Daten
      \item Existiert der registrierte Account noch nicht, so wirde der Student
      der Datenbank hinzugefügt
  	\end{enumerate}
  	\item[Erweiterungen:]~
  	\begin{enumerate}
  	  \item[zu 3)] Statt der Registrierungsmaske wird hier der u-Account
  	  verwendet
  	  \item[nach 4)] Dem Studenten wird nach dem Abschicken seiner Daten eine \\
  	  Verifikations-Email gesendet.
  	 \end{enumerate} 
  	\item[Alternativen:]~
  	\begin{enumerate}
  	  \item[5a)] Wenn doch, so wird kein weiterer Eintrag hinzugefügt, sondern
  	  die zweite Registrierung abgebrochen
  	\end{enumerate} 
  \end{description}
%   
  
  \item
    \begin{description}
    \item[Anwendung:] Verifikation der E-Mail
    \item[Ziel:] Verifikation der E-mail
    \item[Vorbedingung:] Student muss sich registriert haben und seine E-mail
    korrekt angegeben haben.
    \item[Nachbedingung(Erfolg):] Studierender kann sich nun Anmelden und hat
    Zugriff auf die Studierenden-Sicht
    \item[Nachbedingung(Fehlschlag):] Student kann sich nicht mit seinem Account
    anmelden
    \item[Akteure:] Student
    \item[Auslösen des Ereignisses:] Studenten registrieren sich und werden
    darauf hingewiesn inhre E-mail zu verifizieren.
    \item[Beschreibung:]~
    \begin{enumerate}
      \item E-mails über Drittsoftware abrufen.
      \item Klick auf den Verifikations-Link in der E-mail.      
    \end{enumerate}
    \item[Erweiterungen:] -keine-
    \item[Alternativen:] -keine-
      \end{description}
  
  \item
    \begin{description}
    \item[Anwendung:] Anmeldung eines Studenten
    \item[Ziel:] Student wird angemeldet und kann seine Bewertungen ändern
  	\item[Vorbedingung:] Der Student hat sich vorher registriert
  	\item[Nachbedingung(Erfolg):] Der Student ist angemeldet
  	\item[Nachbedingung(Fehlschlag):] Der Student ist nicht angemeldet
  	\item[Akteure:] Student
  	\item[Auslösen des Ereignisses:] Student will sich anmelden
  	\item[Beschreibung:]~
  	\begin{enumerate}
  	  \item Student öffnet die Website
  	  \item Student füllt das Formular zum Anmelden aus und schickt dieses ab
  	  \item Sind die Anmeldedaten korrekt, so wird der Student zur Studentensicht
  	  weitergeleitet
  	\end{enumerate}
  	\item[Erweiterungen:]
  	\begin{enumerate}
  	  \item[2)] Kein Formular, sondern Anmeldung über u-Account
  	\end{enumerate}    	
  	\item[Alternativen:]
	\begin{enumerate}
  	  \item[3a)] Wenn nicht, so erhält er eine Fehlermeldung
  	\end{enumerate}
  \end{description}
   
  
  \item
  \begin{description}
  \item[Anwendung:] Student erstellt Lerngruppe
  \item[Ziel:] Eine neue Lerngruppe ist in der Datenbank vermerkt und der Ersteller ist erstes Mitglied dieser
  	\item[Vorbedingung:] Student ist angemeldet und befindet sich auf der Website im Studentenportal
  	\item[Nachbedingung(Erfolg):] Die Lerngruppe wurde erfolgreich erstellt
  	\item[Nachbedingung(Fehlschlag):] Die Lerngruppe wurde nicht erstellt
  	\item[Akteure:] Student
  	\item[Auslösen des Ereignisses:] Wille des Studenten
  	\item[Beschreibung:]~
  	\begin{enumerate}
  	  \item Student gibt Name und Passwort in das Formular zum Erstellen einer
  	  Lerngruppe ein und schickt dieses ab
  	  \item Wenn noch keine Lerngruppe mit diesem Namen existiert, so wird die
  	  Lerngruppe in der Datenbank angelegt und der Ersteller wird erstes Mitglied
  	\end{enumerate}
  	\item[Erweiterungen:] -keine-
  	\item[Alternativen:] ~
  	\begin{enumerate}
  	  \item[2a)] Eine Lerngruppe mit diesem Namen existiert schon. Dann wird
  	  keine weitere Lerngruppe angelegt und dem Studenten wird mitgeteilt, dass er den Namen der Lerngruppe verändern soll
  	 \end{enumerate}  
  \end{description}
   
  
  \item
  \begin{description}
  \item[Anwendung:] Student tritt Lerngruppe bei
  \item[Ziel:] Student ist Mitglied einer Lerngruppe
  	\item[Vorbedingung:] Student ist angemeldet und befindet sich auf der Website im Studentenportal
  	\item[Nachbedingung(Erfolg):] Der Student ist Mitglied der Lerngruppe
  	\item[Nachbedingung(Fehlschlag):] Der Student ist nicht Mitglied der
  	Lerngruppe
  	\item[Akteure:] Student
  	\item[Auslösen des Ereignisses:] Der Student will einer Lerngruppe beitreten
  	\item[Beschreibung:]~
  	\begin{enumerate}
  	  \item Der Student gibt die Anmeldedaten der Lerngruppe in das passende
  	  Formular ein und schickt dieses ab
  	  \item Existiert die angegebene Lerngruppe, so wird die Mitgliederliste
  	  derer um den Studenten erweitert
  	\end{enumerate}
  	\item[Erweiterungen:]~
  	\begin{enumerate}
  	  \item[2a)] Ist die Lerngruppe schon vollständig, so wird der Student nicht
  	  der Gruppe zugewiesen
  	  \item[3)] Ist der Student nicht der Lerngruppe zugewiesen worden, so werden
  	  die zuletzt gespeicherten Bewertungen verwendet
  	  \item[4)] Der Lerngruppenersteller wird per E-Mail informiert
  	 \end{enumerate}
  	\item[Alternativen:] ~
  	\begin{enumerate}
  	  \item[2a)] Existiert keine solche Gruppe, so schlägt der Beitritt fehl und
  	  der Student ist kein Mitglied der Lerngruppe
  	 \end{enumerate}  
  \end{description}
%   
  
  \item
  \begin{description}
  \item[Anwendung:] Student bewertet Projekte
  \item[Ziel:] Der Student hat eine Bewertung abgegeben
  	\item[Vorbedingung:] Der Student ist angemeldet und im Studentenportal der Website
  	\item[Nachbedingung(Erfolg):] Für den Student ist eine Bewertung eingetragen
  	\item[Nachbedingung(Fehlschlag):] Der Student hat keine Bewertung eingetragen
  	\item[Akteure:] Student
  	\item[Auslösen des Ereignisses:] Des Studentens Wille
  	\item[Beschreibung:]~
  	 \begin{enumerate}
  	   \item Der Student füllt die Bewertungsmaske aus
  	   \item Der Student speichert die Bewertung ab
  	 \end{enumerate}
  	\item[Erweiterungen:] -keine-
  	\item[Alternativen:] -keine-
  \end{description}
   
  
  \item
  \begin{description}
  \item[Anwendung:] Student bewertet Projekte für eine Lerngruppe
  \item[Ziel:] Die Lerngruppe assoziiert mit dem Studenten hat eine Bewertung abgegeben
  	\item[Vorbedingung:] Der Student ist als Leiter einer Lerngruppe angemeldet
  	und im Studentenportal der Website
  	\item[Nachbedingung(Erfolg):] Für die Lerngruppe ist eine Bewertung eingetragen
  	\item[Nachbedingung(Fehlschlag):] Für die Lerngruppe ist keine (neue)
  	Bewertung eingetragen
  	\item[Akteure:] Student
  	\item[Auslösen des Ereignisses:] Des Studentens Wille
  	\item[Beschreibung:]~
  	 \begin{enumerate}
  	   \item Der Student füllt die Bewertungsmaske aus
  	   \item Der Student schließt die Bewertung ab
  	 \end{enumerate}
  	\item[Erweiterungen:] -keine-
  	\item[Alternativen:] -keine-
  \end{description}
   
  
  \item
    \begin{description}
  	\item[Anwendung:] Passwort vergessen
  	\item[Ziel:] Erhalten des Passworts
  	\item[Vorbedingung:] Der Student hat schon einen Account
  	\item[Nachbedingung(Erfolg):] Der Student erhält eine E-Mail mit seinem
  	Passwort
  	\item[Nachbedingung(Fehlschlag):] Der Student erhält keine E-Mail
  	\item[Akteure:] Student
  	\item[Auslösen des Ereignisses:] Student hat sein Passwort vergessen
  	\item[Beschreibung:]~
  	\begin{enumerate}
  	  \item Öffnen der Internetseite
      \item Klick auf "Passwort vergessen"
      \item Eingabe der E-Mail Adresse
      \item Der Student erhält eine E-Mail mit seinem Passwort
  	\end{enumerate}
  	\item[Erweiterungen:] -keine-
  	\item[Alternativen:] -keine-
  \end{description}
   
\end{enumerate}

\subsubsection{Betreuer}
\begin{enumerate} [label=\swtLabel{B}]
  \item
	 \begin{description}
		\item[Anwendung:] Betreuer meldet sich an
  		\item[Ziel:] Anmeldung des Betreuers
  		\item[Vorbedingung:] Konto des Betreuers wurde angelegt
  		\item[Nachbedingung(Erfolg):] Betreuer ist angemeldet
  		\item[Nachbedingung(Fehlschlag):] Betreuer ist nicht angemeldet
  		\item[Akteure:] Betreuer
  		\item[Auslösen des Ereignisses:] Betreuer möchte sich anmelden
  		\item[Beschreibung]~
  		\begin{enumerate}
  			\item Der Betreuer gibt seine Zugangsdaten ein
  			\item Wenn die Anmeldedaten korrekt sind, wird er auf die Betreueransich
  			weitergeleitet
  		\end{enumerate}
  		\item[Erweiterungen:] -keine-
  		\item[Alternativen:] ~
  		\begin{enumerate}
  		  \item[2a)] Wenn nicht, erhält er eine Fehlermeldung
  		\end{enumerate}  
  	\end{description}
   
  
  \item
	\begin{description}
  		\item[Anwendung:] Thema erstellen/ändern
  		\item[Ziel:] Eröffnung/Änderung eines PSE-Themas
  		\item[Vorbedingung:] -keine-
  		\item[Nachbedingung(Erfolg):] Themendaten sind im System eingetragen
  		\item[Nachbedingung(Fehlschlag):] Themenfaten sind nicht im System
  		eingetragen
  		\item[Akteure:] Betreuer
  		\item[Auslösen des Ereignisses:] Betreuer möchte ein PSE-Thema ins System
  		einfügen/ändern.
  		\item[Beschreibung:]~
  	\begin{enumerate}
  	  \item Betreuer befindet sich auf der Website mit den PSE-Themen
  	  \item Existiert das Thema schon, so wählt der Betreuer dieses aus
  	  \item Betreuer füllt die Eingabemaske aus
  	  \item Betreuer speichert die neuen Daten ab
  	\end{enumerate}
  	\item[Erweiterungen:] -keine-
  	\item[Alternativen:]~
  	\begin{enumerate}
  	  \item[2a)] Wenn nicht, so fügt er ein neues hinzu
  	\end{enumerate}  
  \end{description}

 
  \item
	\begin{description}
  		\item[Anwendung:] Gruppenbetreuer werden
  		\item[Ziel:] Betreuer einer bestimmten Gruppe werden
  		\item[Vorbedingung:] Man ist noch nicht Betreuer der ausgewählten Gruppe
  		\item[Nachbedingung(Erfolg):] Erfolgreich Betreuer der ausgewählten
  		Gruppe geworden
  		\item[Nachbedingung(Fehlschlag):] Man ist nicht Betreuer der Gruppe
  		\item[Akteure:] Betreuer
  		\item[Auslösen des Ereignisses:] Betreuer möchte sich einer Gruppe
  		zuordnen
  		\item[Beschreibung:]~
  	\begin{enumerate} 
  	  \item Betreuer befindet sich auf der Website mit den PSE-Gruppen
  	  \item Betreuer wählt die Gruppe aus, welcher er betreuen möchte
  	  \item Betreuer tritt der Gruppe bei
  	\end{enumerate}
  	\item[Erweiterungen:] -keine-
  	\item[Alternativen:] -keine-
  \end{description}
   
  
  \item
	\begin{description}
  		\item[Anwendung:] Gruppe verlassen
  		\item[Ziel:] Eine Gruppe nicht mehr betreuen
  		\item[Vorbedingung:] Man ist Betreuer einer Gruppe, in der noch mindestens
  		ein anderer Betreuer ist.
  		\item[Nachbedingung(Erfolg):] Man betreut die Gruppe nicht länger.
  		\item[Nachbedingung(Fehlschlag):] Man ist immernoch Betreuer der Gruppe.
  		\item[Akteure:] Betreuer
  		\item[Auslösen des Ereignisses:] Betreuer möchte aus Gruppe austreten.
  		\item[Beschreibung:]~
  	\begin{enumerate} 
  	  \item Betreuer befindet sich auf der Website mit den PSE-Gruppen, die er
  	  betreut.
  	  \item Betreuer wählt die Gruppe aus, welche er verlassen möchte.
  	  \item Betreuer verlässt die Gruppe.
  	\end{enumerate}
  	\item[Erweiterungen:] -keine-
  	\item[Alternativen:] -keine-
  \end{description}
   
  
  \item
    \begin{description}
  	\item[Anwendung:] Noten für Gruppenteilnehmer eintragen
  	\item[Ziel:] Speichern der Noten in der Datenbank
  	\item[Vorbedingung:] Gruppe hat Studenten als Teilnehmer
  	\item[Nachbedingung(Erfolg):] Eintragung/Änderung der Noten
  	\item[Nachbedingung(Fehlschlag):] Notenänderung wird verworfen
  	\item[Akteure:] Betreuer
  	\item[Auslösen des Ereignisses:] Betreuer möchte Noten eintragen
  	\item[Beschreibung:]~
  	\begin{enumerate} 
  	  \item Betreuer befindet sich auf der Website mit der Notenübersicht
  	  \item Betreuer trägt neue Noten für eine beliebige Phase ein oder ändert bestehende Noten.
  	\end{enumerate}
  	\item[Erweiterungen:] -keine-
  	\item[Alternativen] -keine-
  \end{description}
   
\end{enumerate}


\subsubsection{Administrator}
\begin{enumerate} [label=\swtLabel{A}]
  \item
    \begin{description}
  	\item[Anwendung:] Hinzufügen/Ändern eines Projekts
  	\item[Ziel:] Einfügen/Änderung von Projektdaten in der Datenbank
  	\item[Vorbedingung:] -keine-
  	\item[Nachbedingung(Erfolg):] Neue Projektdaten sind eingetragen
  	\item[Nachbedingung(Fehlschlag):] Neue Projektdaten sind nicht eingetragen
  	\item[Akteure:] Administrator, Institut
  	\item[Auslösen des Ereignisses:] Administrator bekommt Projektdaten von einem
  	Institut\\ vorgelegt
  	\item[Beschreibung:]~
  	\begin{enumerate} 
  	  \item Projektübersicht öffnen
  	  \item Wenn das Projekt schon existiert, so werden die Projektdaten
  	  verglichen
  	  \item Stimmen die Projektdaten nicht überein, so werden die betroffenen
  	  Daten\\ angepasst
  	  \item Durch Abspeichern der Eingaben werden die Daten in der Datenbank
  	  angepasst
  	\end{enumerate}
  	\item[Erweiterungen:] -keine-
  	\item[Alternativen:]~
  	\begin{enumerate}
  	  \item[2a)] Wenn nicht, dann wird das Projekt neu angelegt
  	  \item[3a)] Falls doch, so ist dieser Anwendungsfall abgeschlossen
  	\end{enumerate} 
  \end{description}
   
  
  \item
  \begin{description}
  \item[Anwendung:] Löschen eines Projekts
  \item[Ziel:] Löschen eines Projekts in die Datenbank
  	\item[Vorbedingung:] Projekt existiert bereits in der Datenbank
  	\item[Nachbedingung(Erfolg):] Projekt ist aus der Datenbank entfernt
  	\item[Nachbedingung(Fehlschlag):] Projekt ist immernoch in der Datenbank
  	\item[Akteure:] Administrator, Institut
  	\item[Auslösen des Ereignisses:] Institut zieht ein Projekt zurück oder es
  	fehlen Daten bei Beginn der Einteilung
  	\item[Beschreibung:]~
  	\begin{enumerate} 
  	  \item Projektübersicht öffnen
  	  \item Projekt aus Liste entfernen
  	  \item Projektdaten werden aus der Datenbank entfernt
  	\end{enumerate}
  	\item[Erweiterungen:] -keine-
  	\item[Alternativen:] -keine-
  \end{description}
   
  
  \item
  \begin{description}
  \item[Anwendung:] Teameinteilung
  \item[Ziel:] Finden einer Teameinteilung nach ausgewählten Parametern
  	\item[Vorbedingung:] Alle Projekte in der Datenbank sind vollständig
  	\item[Nachbedingung(Erfolg):] Es wurde eine Teameinteilung gefunden
  	\item[Nachbedingung(Fehlschlag):] Es wurde keine Teameinteilung gefunden
  	\item[Akteure:] Administrator
  	\item[Auslösen des Ereignisses:] Entscheidung des Administrators bzw.
  	zeitliche Deadline
  	\item[Beschreibung:]~
  	\begin{enumerate} 
  	  \item Einstellen der Parameter nach den eingeteilt werden soll
  	  \item Starten der Berechnung über Schaltfläche
  	  \item Sind alle Projekte vollständig angegeben, so startet die Berechnung
  	  \item Nach Abschluss der Berechnung wird das Ergebnis inklusive
  	  eingestellter Parameter abgespeichert.
  	\end{enumerate}
  	\item[Erweiterungen:] -keine-
  	\item[Alternativen:] ~
  	\begin{enumerate}
  	  \item[3a)] Wenn nicht, so startet die Berechnung nicht und der Fall ist
  	  beendet
  	  \item[4a)] Hier ist auch ein frühzeitiger Abbruch durch den Administrator
  	  möglich. Dies wird im Ergebnis vermerkt.
  	 \end{enumerate}  
  \end{description}
   
  
  \item
  \begin{description}
  \item[Anwendung:] Finale Auswahl der Einteilung
  \item[Ziel:] Finden der bestmöglichen Teameinteilung
  	\item[Vorbedingung:] Es wurde mindestens eine Berechnung durchgeführt
  	\item[Nachbedingung(Erfolg):] Es wurde eine Teameinteilung ausgewählt
  	\item[Nachbedingung(Fehlschlag):] Es wurde keine Teameinteilung ausgewählt
  	\item[Akteure:] Administrator
  	\item[Auslösen des Ereignisses:] Entscheidung des Administrators bzw.
  	zeitliche Deadline
  	\item[Beschreibung:]~
  	\begin{enumerate} 
  	  \item Der Administrator wählt eine der berechneten Einteilung final aus
  	  \item Wurde ein Student zugewiesen, so wird seinem Datenbankeintrag sein
  	  Team hinzugefügt
  	\end{enumerate}
  	\item[Erweiterungen:]~
  	\begin{enumerate}
  	  \item[nach 2)] Betreuer werden über ihre Teams informiert
  	  \item[nach 2)] Die Studenten werden über ihre Einteilung informiert
  	 \end{enumerate}
  	\item[Alternativen:] ~
  	\begin{enumerate}
  	  \item[2a)] Wird ein Student nicht zugewiesen, so wird dies ebenfalls in der
  	  Datenbank vermerkt
  	 \end{enumerate}  
  \end{description}
   
  
  \item
  \begin{description}
  \item[Anwendung:] Einteilung eines Studenten
  \item[Ziel:] Manuelle Einteilung eines Studenten
  	\item[Vorbedingung:] Automatische Teameinteilung ist abgeschlossen
  	\item[Nachbedingung(Erfolg):] Der Student wurde hinzugefügt
  	\item[Nachbedingung(Fehlschlag):] Der Student wurde nicht hinzugefügt
  	\item[Akteure:] Administrator, Student
  	\item[Auslösen des Ereignisses:] Student meldet sich beim Administrator
  	\item[Beschreibung:]~
  	 \begin{enumerate} 
  	   \item Wenn sich der Student noch nicht im System befindet überprüft der
  	   Administrator, ob die Anforderungen für das PSE erfüllt sind
  	   \item Hat der Student die Anforderungen erfüllt, so versucht der
  	   Administrator ihn einem Team zuzuteilen
  	   \item Hat der Administrator ein Team gefunden so fügt er den Student über
  	   eine Eingabemaske dem System hinzu
  	 \end{enumerate}
  	\item[Erweiterungen:]~
  	 \begin{enumerate}
  	   \item [nach 3)] Betreuer, sowie andere Mitglieder des Projekts, werden
  	   über das neue Mitglied per E-Mail informiert 
  	 \end{enumerate}  
  	\item[Alternativen:] ~
  	 \begin{enumerate}
  	  \item[1a)] Falls doch, so wurde er schon vorher abgelehnt/nicht zugewiesen
  	  und wird nicht eingeteilt. Dieser Fall ist damit abgeschlossen
  	  \item [2a)] Falls er die Anforderungen nicht erfüllt, so wird er auch nicht
  	  zugeteilt und dieser Fall ist abgeschlossen
  	 \end{enumerate}  
  \end{description}
   
  
  \item
  \begin{description}
  \item[Anwendung:] Erstellen eines Betreueraccounts
  \item[Ziel:] Hinzufügen eines Betreuers in das System
  	\item[Vorbedingung:] -keine-
  	\item[Nachbedingung(Erfolg):] Der Betreueraccount wurde erstellt
  	\item[Nachbedingung(Fehlschlag):] Der Betreueraccount wurde nicht erstellt
  	\item[Akteure:] Administrator
  	\item[Auslösen des Ereignisses:] Betreuer benötigt Account
  	\item[Beschreibung:]~
  	 \begin{enumerate} 
  	   \item Der Administrator öffnet die Maske zum Erstellen eines
  	   Betreueraccounts
  	   \item Der Administrator füllt die Maske aus und schickt sie ausgefüllt ab
  	   \item Wenn noch kein Account mit dem angegebenen Namen existiert, so wird
  	   der neue Account der Datenbank hinzugefügt
  	 \end{enumerate}
  	\item[Erweiterungen:]~
  	 \begin{enumerate}
  	   \item[4)] Der Betreuer erhält seine Zugangsdaten per E-Mail
  	 \end{enumerate}  
  	\item[Alternativen:] ~
  	 \begin{enumerate}
  	  \item[3a)] Falls doch, so wird eine Fehlermeldung angezeigt
  	 \end{enumerate}  
  \end{description}
  
\end{enumerate}  
\subsection{Objektmodelle}

\subsection{Dynamische Modelle}

\subsection{Benutzerschnittstelle}
\begin{enumerate}
  \item Es ist eine Benutzung rein über ein Web-Interface vorgesehen.
  \item Es sind drei Sichten zu unterscheiden:
        \begin{itemize}
          \item die Sicht des Studierenden
          \item die Sicht des Betreuers
          \item die Sicht des Administrators
        \end{itemize}
  \item Die jeweiligen sichten sind erst nach der Anmeldung einzusehen. 
  \item Studierende dürfen nur auf ihre Eigenen Daten zugreiifen.
  
\end{enumerate}
\end{document}
