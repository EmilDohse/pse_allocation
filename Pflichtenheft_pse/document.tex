%%This is a very basic article template.
%%There is just one section and two subsections.
\documentclass[parskip=full]{scrartcl}
\usepackage[utf8]{inputenc}
\usepackage[T1]{fontenc}
\usepackage[ngerman]{babel}
\usepackage{paralist}

\begin{document}

\title{Pflichtenheft \\
        \large Automatische Einteilung für das PSE}

\author{D. Biester, E.Dohse, P. Faller, P. Loth L. Seufert, Sam}
        
\maketitle
\pagebreak

\tableofcontents
\pagebreak

\section{Zielbestimmung}
Wissenschaftliche Mitarbeiter des IPD sollen durch das Produkt die Verwaltung
der Projekte und die Einteilung zur PSE rechnerunterstützt durchzuführen können.

 
\subsection{Musskriterien}


\begin{enumerate}[{Z}1]
    \item Datenerfassung: Studierende sollen auf einer Internetseite:   
    \begin{itemize}
        \item ihre Kontaktdaten bestehend aus Name und E-mail,
        \item ihre Lerngruppen von 2 bis 6 Personen,
        \item ihre Themenvorlieben 
        \item sowie ihre erbrachten Studienleistungen 
    \end{itemize}
    in das System eingeben können.
    
    \item  Verifikation der E-Mail-Adresse
    

    \item Einteilung: Das Problem der Zuteilung von Studierenden zu PSE-Teams
    soll geeignet modelliert und gelöst werden. 
        \begin{itemize}
        \item Wer die Voraussetzungen nicht erfüllt wird nicht eingeteilt.

        \item Möglichst gleiche Semester im Team.

        \item Lerngruppen sollten zusammenbleiben.

        \item Studierende bevorzugen, die bereits mehr Sachen aus dem ersten Jahr
        bestanden haben. (über die Voraussetzungen hinaus, z.B. Algo)

        \item Präferenzen der Studierenden berücksichtigen.
        \end{itemize}

    \item Abbrechen der Berechnung der Zuteilung möglich (manuell oder Timeout)

    \item Nachjustieren der Einteilung “von Hand”

    \item Berechnung von Gütekriterien (Studierenden-Happiness,
    Anteil-Nicht-Eingeteilter, etc.) 
    

    \item Export/Import der Aus- und Eingabe der Studierendendaten und
    Einteilung(en)

    \item Möglichkeit der sukzessiven Eintragung von angebotenen Themen und
    Anzahl der Plätze und Teams, durch die jeweiligen Projektleiter

    \item GUI zur einfacheren Bedienung

    

    
\end{enumerate}


\subsection{Wunschkriterien}
\begin{enumerate}[{W}1]
    \item Wunschkriterien zur Lösung des Lproblems der Einteilung:
    \begin{itemize}
        \item Eher 5er Teams als 6er Teams
    \end{itemize}    
    
    \item Authentifizierung via Shibboleth

    \item Benachrichtigung der Studierenden und Mitarbeiter über Einteilung

    \item Verwaltung mehrerer Einteilungsergebnisse

    \item  Erweiterbarkeit bzgl. “Betreuer-Sicht”: Themenerfassung, -pflege und Notenerfassung
    
    \item Nachjustierbarkeit bei Änderungen, z.B. der SPO.
    
    \item Verwaltung über mehrere Semester hinweg
    
    \item Anzeige der Projekt-Details für Studenten
    
    \item Stapelsystem zur Berechnungen verschiedener Einteilungen mit unterschiedlichen Konfigurationen.
    
    
\end{enumerate}

\subsection{Abgrenzungskriterien}
\begin{enumerate}[{A}1]
 
  \item Die Produkt ist nicht für die Nutzung außerhalb des KIT's gedacht.

\item Die Einteilungsfunktion ist nur für die Einteilung zum PSE gedacht, nicht
für die Vergabe sonstiger Praktikums- oder Tutorienplätze.
  
\end{enumerate}
\section{Produkteinsatz}
\begin{itemize}
  \item Das Produkt dient zur Einpflegung von Projekten und zur Zuteilung von
Studierenden zu den Projekten.
\end{itemize}

\subsection{Anwendungsbereiche}

\begin{itemize} 
  \item Anwendung im Universitären Bereich. 
\end{itemize}

\subsection{Zielgruppe}
\begin{itemize} 
  \item Studierende 
  \item Mitarbeiter des IPD 
\end{itemize}

\subsection{Betriebsbedingungen}
\begin{itemize} 
  \item Der Betrieb erfolgt über eine Web-Schnittstelle.
\end{itemize}
\section{Produktumgebung}
\begin{itemize} 
  \item Das Produkt läuft auf einem Server des IPD.
\end{itemize}
\subsection{Software}
\begin{itemize} 
  \item Betriebssystem: Linux 64Bit
  \item ILP- Solver Gurobi
  \item Play - Framework
\end{itemize}
\subsection{Hardware}
\begin{itemize} 
  \item 16 GB Ram
  \item Intel I7 Quadcore mit 4 weiterne virtuellen Kernen 
\end{itemize}
// hier überlegen ob man noch Orgware und Produkt-Schnittstellen dazu noimmt
\section{Funktionale Anforderungen}

\subsection{Funktionsübersicht}

\section{Produktdaten}

\subsection{Projektdaten} 
Über ein Projekt sind folgende Daten zu speichern:
\begin{itemize} 
  \item Name,
  \item Anzahl der Teams,
  \item Beschreibung,
  \item Namen und E-Mail Adressen der Betreuer
  \item 
\end{itemize}
\subsection{Studierenden-Daten} 
Zu jedem Studierenden sollen folgende Daten gespeichert werden:
\begin{itemize} 
  \item E-Mail Adresse
  \item Name
  \item Falls keiner Lerngruppe Beigetreten: Bewertung, welchem Projekt er oder
  sie am liebsten zugeteiilt würde
  \item Falls einer Lerngruppe beoigetreten: Die zugehörige Lerngruppe.
\end{itemize}
\subsection{Lerngruppen-Daten} 
\begin{itemize} 
  \item Bewertung der Projekte, zu welchem die Lerngruppenmitglieder
  (Studierenden) am liebsten zugeteilt würden.
\end{itemize}

\section{Nichtfunktionale Anforderungen}
\begin{enumerate}
  \item 
\end{enumerate}
\section{Globale Testfälle}
\begin{enumerate}
  \item 
\end{enumerate}
\section{Systemmodelle}

\subsection{Szenarien}

\subsection{Anwendungsfälle}

\subsection{Objektmodelle}

\subsection{Dynamische Modelle}

\subsection{Benutzerschnittstelle}
\begin{enumerate}
  \item Es istt eine Benutzung rein über ein Web-Interface vorgesehen.
  \item Es sind zwei Sichten zu unterscheiden:
        \begin{itemize}
          \item die Sicht des Studierenden
          \item die Sicht des Administrators
        \end{itemize}
  \item Die jeweiligen sichten sind erst nach der Anmeldung einzusehen. 
  \item Studierende dürfen nur auf ihre Eigenen daten zugreiifen.
  
\end{enumerate}




\end{document}
