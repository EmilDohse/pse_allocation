%%This is a very basic article template.
%%There is just one section and two subsections.
\documentclass[parskip=full]{scrartcl}
\usepackage[utf8]{inputenc}
\usepackage[T1]{fontenc}
\usepackage[ngerman]{babel}
\usepackage{paralist}


\begin{document}

\title{Pflichtenheft \\
        \large Automatische Einteilung für das PSE}

\author{D. Biester, E.Dohse, P. Faller, P. Loth L. Seufert, Sam}
        
\maketitle
\vfill

\tableofcontents


\section{Zielbestimmung}


\subsection{Musskriterien}


\begin{enumerate}[{Z}1]
    \item Datenerfassung: Studierende sollen auf einer Internetseite:   
    \begin{itemize}
        \item ihre Kontaktdaten bestehend aus Name und E-mail,
        \item ihre Lerngruppen von 2 bis 6 Personen,
        \item ihre Themenvorlieben 
        \item sowie ihre erbrachten Studienleistungen 
    \end{itemize}
    in das System eingeben können.
    
    \item  Verifikation der E-Mail-Adresse
    

    \item Einteilung: Das Problem der Zuteilung von Studierenden zu PSE-Teams
    soll geeignet modelliert und gelöst werden. 
        \begin{itemize}
        \item Wer die Voraussetzungen nicht erfüllt wird nicht eingeteilt.

        \item Möglichst gleiche Semester im Team.

        \item Lerngruppen sollten zusammenbleiben.

        \item Studierende bevorzugen, die bereits mehr Sachen aus dem ersten Jahr
        bestanden haben. (über die Voraussetzungen hinaus, z.B. Algo)

        \item Präferenzen der Studierenden berücksichtigen.
        \end{itemize}

    \item Abbrechen der Berechnung der Zuteilung möglich (manuell oder Timeout)

    \item Nachjustieren der Einteilung “von Hand”

    \item Berechnung von Gütekriterien (Studierenden-Happiness,
    Anteil-Nicht-Eingeteilter, etc.) <- hier noch genauer bei betreuern
    nachfragen

    

    \item Export/Import der Aus- und Eingabe der Studierendendaten und
    Einteilung(en)

    \item Möglichkeit der sukzessiven Eintragung von angebotenen Themen und
    Anzahl der Plätze und Teams, durch die jeweiligen Projektleiter

    \item GUI zur einfacheren Bedienung

    

    
\end{enumerate}


\subsection{Wunschkriterien}
\begin{enumerate}[{W}1]
    \item Wunschkriterien zur Lösung des Lproblems der Einteilung:
    \begin{itemize}
        \item Eher 5er Teams als 6er Teams
    \end{itemize}    
    
    \item Authentifizierung via Shibboleth

    \item Benachrichtigung der Studierenden und Mitarbeiter über Einteilung

    \item Verwaltung mehrerer Einteilungsergebnisse

    \item  Erweiterbarkeit bzgl. “Betreuer-Sicht”: Themenerfassung, -pflege und Notenerfassung
    
    \item Nachjustierbarkeit bei Änderungen, z.B. der SPO.
    
    \item Verwaltung über mehrere Semester hinweg
    
    \item Anzeige der Projekt-Details für Studenten
    
    \item Stapelsystem zur Berechnungen verschiedener Einteilungen mit unterschiedlichen Konfigurationen.
\end{enumerate}

\subsection{Abgrenzungskriterien}
\begin{enumerate}[{A}1]
 
  \item Die Produkt ist nicht für die Nutzung außerhalb des KIT's gedacht.

\item Die Einteilungsfunktion ist nur für die Einteilung zum PSE gedacht, nicht
für die Vergabe sonstiger Praktikums- oder Tutorienplätze.
  
\end{enumerate}
\section{Produkteinsatz}

\subsection{Anwendungsbereiche}

\subsection{Zielgruppe}

\subsection{Betriebsbedingungen}

\section{Produktumgebung}

\subsection{Software}

\subsection{Hardware}
// hier überlegen ob man noch Orgware und Produkt-Schnittstellen dazu noimmt
\section{Funktionale Anforderungen}

\subsection{Funktionsübersicht}

\section{Produktdaten}

\subsection{Musskriterien} // hier noch die unterpunkte

\section{Nichtfunktionale Anforderungen}

\section{Globale Testfälle}

\section{Systemmodelle}

\subsection{Szenarien}

\subsection{Anwendungsfälle}

\subsection{Objektmodelle}

\subsection{Dynamische Modelle}

\subsection{Benutzerschnittstelle}


\end{document}
