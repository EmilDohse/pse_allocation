%%This is a very basic article template.
%%There is just one section and two subsections.
\documentclass[parskip=full]{scrartcl}
\usepackage[utf8]{inputenc}
\usepackage[T1]{fontenc}
\usepackage[ngerman]{babel}
\usepackage{paralist}


\begin{document}

\title{Pflichtenheft \\
        \large Automatische Einteilung für das PSE}

\author{D. Biester, E.Dohse,P. Faller, P. Loth L. Seufert, Sam}
        
\maketitle
\vfill

\tableofcontents


\section{Zielbestimmung}


\subsection{Musskriterien}


\begin{enumerate}[{Z}1]
    \item Datenerfassung: Studierende sollen ihre Kontaktdaten, Gruppen und
    Themenvorlieben sowie ihre erbrachten Studienleistungen in das System eingeben können.

    \item Verifikation der E-Mail-Adresse

    \item Einteilung: Das Problem der Zuteilung von Studierenden zu PSE-Teams
    soll geeignet modelliert und gelöst werden. Eine Möglichkeit der Modellierung wäre ein ILP (Integer Linear Programming), das dann mit existierenden Werkzeugen gelöst werden kann. Richtlinien zur Einteilung:

        \item Wer die Voraussetzungen nicht erfüllt wird eher nicht eingeteilt.

        \item Möglichst gleiche Semester im Team.

        \item Lerngruppen sollten zusammenbleiben.

        \item Eher 5er Teams als 6er Teams

        \item Studenten bevorzugen, die bereits mehr Sachen aus dem ersten Jahr
        bestanden haben. (über die Voraussetzungen hinaus, z.B. Algo)

        \item Präferenzen der Studenten berücksichtigen.

    \item Abbrechen der Berechnung der Zuteilung möglich (manuell oder Timeout)

    \item Nachjustieren der Einteilung “von Hand”

    \item Berechnung von Gütekriterien (Studenten-Happiness,
    Anteil-Nicht-Eingeteilter, etc.)

    \item Benachrichtigung der Studenten und Mitarbeiter über Einteilung

    \item Export/Import der Ein- und Ausgabe der Studentendaten und
    Einteilung(en)

    \item Verwaltung mehrerer Einteilungsergebnisse

    \item Erweiterbarkeit bzgl. “Betreuer-Sicht”: Themenerfassung, -pflege und
    Notenerfassung

    \item Authentifizierung via Shibboleth

    \item Professionelle GUI
\end{enumerate}


\subsection{Wunschkriterien}
Plain text.

\subsection{Abgrenzungskriterien}

\section{Produkteinsatz}

\subsection{Anwendungsbereiche}

\subsection{Zielgruppe}

\subsection{Betriebsbedingungen}

\section{Produktumgebung}

\subsection{Software}

\subsection{Hardware}
// hier überlegen ob man noch Orgware und Produkt-Schnittstellen dazu noimmt
\section{Funktionale Anforderungen}

\subsection{Funktionsübersicht}

\section{Produktdaten}

\subsection{Musskriterien} // hier noch die unterpunkte

\section{Nichtfunktionale Anforderungen}

\section{Globale Testfälle}

\section{Systemmodelle}

\subsection{Szenarien}

\subsection{Anwendungsfälle}

\subsection{Objektmodelle}

\subsection{Dynamische Modelle}

\subsection{Benutzerschnittstelle}


\end{document}
