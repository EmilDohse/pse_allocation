%%This is a very basic article template.
%%There is just one section and two subsections.
\documentclass[a4paper]{article}
\usepackage{geometry}
\usepackage[ngerman]{babel}
\usepackage[T1]{fontenc}
\usepackage[utf8]{inputenc}
\usepackage{paralist}

\begin{document}
\author{Philipp Loth}
\title{\large Anwendungsfälle - Admin}
\maketitle

\begin{enumerate}%[\textbf{{\textbackslash Z}10{\textbackslash}}]
  
  \item[\textbf{\textbackslash Z10\textbackslash}] \begin{description}
  	\item[Anwendung:] Hinzufügen/Ändern eines Projekts
  	\item[Ziel:] Einfügen/Änderung von Projektdaten in der Datenbank
  	\item[Vorbedingung:] -keine-
  	\item[Nachbedingung(Erfolg):] Neue Projektdaten sind eingetragen
  	\item[Nachbedingung(Fehlschlag):] Neue Projektdaten sind nicht eingetragen
  	\item[Akteure:] Administrator, Institut
  	\item[Auslösen des Ereignisses:] Administrator bekommt Projektdaten von einem
  	Institut\\ vorgelegt
  	\item[Beschreibung:]~
  	\begin{enumerate}[1.]
  	  \item Projektübersicht öffnen
  	  \item Wenn das Projekt schon existiert, so werden die Projektdaten
  	  verglichen
  	  \item Stimmen die Projektdaten nicht überein, so werden die betroffenen
  	  Daten\\ angepasst
  	  \item Durch Abspeichern der Eingaben werden die Daten in der Datenbank
  	  angepasst
  	\end{enumerate}
  	\item[Erweiterungen:]
  	\item[Alternativen:]~
  	\begin{enumerate}
  	  \item[2a)] Wenn nicht, dann wird das Projekt neu angelegt
  	  \item[3a)] Falls doch, so ist dieser Anwendungsfall abgeschlossen
  	\end{enumerate} 
  \end{description}
  \pagebreak
  
  \item[\textbf{\textbackslash Z20\textbackslash}] \begin{description}
  \item[Anwendung:] Löschen eines Projekts
  \item[Ziel:] Löschen eines Projekts in die Datenbank
  	\item[Vorbedingung:] Projekt existiert bereits in der Datenbank
  	\item[Nachbedingung(Erfolg):] Projekt ist aus der Datenbank entfernt
  	\item[Nachbedingung(Fehlschlag):] Projekt ist immernoch in der Datenbank
  	\item[Akteure:] Administrator, Institut
  	\item[Auslösen des Ereignisses:] Institut zieht ein Projekt zurück oder es
  	fehlen Daten bei Beginn der Einteilung
  	\item[Beschreibung:]~
  	\begin{enumerate}[1.]
  	  \item Projektübersicht öffnen
  	  \item Projekt aus Liste entfernen
  	  \item Projektdaten werden aus der Datenbank entfernt
  	\end{enumerate}
  	\item[Erweiterungen:]
  	\item[Alternativen:]
  \end{description}
  \pagebreak
  
  \item[\textbf{\textbackslash Z30\textbackslash}] \begin{description}
  \item[Anwendung:] Teameinteilung
  \item[Ziel:] Finden einer Teameinteilung nach ausgewählten Parametern
  	\item[Vorbedingung:] Alle Projekte in der Datenbank sind vollständig
  	\item[Nachbedingung(Erfolg):] Es wurde eine Teameinteilung gefunden
  	\item[Nachbedingung(Fehlschlag):] Es wurde keine Teameinteilung gefunden
  	\item[Akteure:] Administrator
  	\item[Auslösen des Ereignisses:] Entscheidung des Administrators bzw.
  	zeitliche Deadline
  	\item[Beschreibung:]~
  	\begin{enumerate}[1.]
  	  \item Einstellen der Parameter nach den eingeteilt werden soll
  	  \item Starten der Berechnung über Schaltfläche
  	  \item Sind alle Projekte vollständig angegeben, so startet die Berechnung
  	  \item Nach Abschluss der Berechnung wird das Ergebnis inklusive
  	  eingestellter Parameter abgespeichert.
  	\end{enumerate}
  	\item[Erweiterungen:]
  	\item[Alternativen:] ~
  	\begin{enumerate}
  	  \item[3a)] Wenn nicht, so startet die Berechnung nicht und der Fall ist
  	  beendet
  	  \item[4a)] Hier ist auch ein frühzeitiger Abbruch durch den Administrator
  	  möglich. Dies wird im Ergebnis vermerkt.
  	 \end{enumerate}  
  \end{description}
  \pagebreak
  
  \item[\textbf{\textbackslash Z40\textbackslash}] \begin{description}
  \item[Anwendung:] Finale Auswahl der Einteilung
  \item[Ziel:] Finden der bestmöglichen Teameinteilung
  	\item[Vorbedingung:] Es wurde mindestens eine Berechnung durchgeführt
  	\item[Nachbedingung(Erfolg):] Es wurde eine Teameinteilung ausgewählt
  	\item[Nachbedingung(Fehlschlag):] Es wurde keine Teameinteilung ausgewählt
  	\item[Akteure:] Administrator
  	\item[Auslösen des Ereignisses:] Entscheidung des Administrators bzw.
  	zeitliche Deadline
  	\item[Beschreibung:]~
  	\begin{enumerate}[1.]
  	  \item Der Administrator wählt eine der berechneten Einteilung final aus
  	  \item Wurde ein Student zugewiesen, so wird seinem Datenbankeintrag sein
  	  Team hinzugefügt
  	\end{enumerate}
  	\item[Erweiterungen:]~
  	\begin{enumerate}
  	  \item[nach 2] Betreuer werden über ihre Teams informiert
  	  \item[nach 2] Die Studenten werden über ihre Einteilung informiert
  	 \end{enumerate}
  	\item[Alternativen:] ~
  	\begin{enumerate}
  	  \item[2a)] Wird ein Student nicht zugewiesen, so wird dies ebenfalls in der
  	  Datenbank vermerkt (!!!Wie? Nochmal Besprechen am Montag?!!!)
  	 \end{enumerate}  
  \end{description}
  \pagebreak
  
  \item[\textbf{\textbackslash Z50\textbackslash}] \begin{description}
  \item[Anwendung:] Einteilung eines Studenten
  \item[Ziel:] Manuelle Einteilung eines Studenten
  	\item[Vorbedingung:] Automatische Teameinteilung ist abgeschlossen
  	\item[Nachbedingung(Erfolg):] Der Student wurde hinzugefügt
  	\item[Nachbedingung(Fehlschlag):] Der Student wurde nicht hinzugefügt
  	\item[Akteure:] Administrator, Student
  	\item[Auslösen des Ereignisses:] Student meldet sich beim Administrator
  	\item[Beschreibung:]~
  	 \begin{enumerate}[1.]
  	   \item Wenn sich der Student noch nicht im System befindet überprüft der
  	   Administrator, ob die Anforderungen für das PSE erfüllt sind
  	   \item Hat der Student die Anforderungen erfüllt, so versucht der
  	   Administrator ihn einem Team zuzuteilen
  	   \item Hat der Administrator ein Team gefunden und sich dabei an die
  	   generellen Regeln gehalten, so wird der Student über eine Eingabemaske der
  	   Datenbank hinzugefügt
  	 \end{enumerate}
  	\item[Erweiterungen:]~
  	 \begin{enumerate}
  	   \item [nach 3] Betreuer, sowie andere Mitglieder des Projekts, werden
  	   über das neue Mitglied per E-Mail informiert 
  	 \end{enumerate}  
  	\item[Alternativen:] ~
  	 \begin{enumerate}
  	  \item[1a)] Falls doch, so wurde er schon vorher abgelehnt/nicht zugewiesen
  	  und wird nicht eingeteilt. Dieser Fall ist damit abgeschlossen
  	  \item [2a)] Falls er die Anforderungen nicht erfüllt, so wird er auch nicht
  	  zugeteilt und dieser Fall ist abgeschlossen
  	  \item[3a)] Wenn nicht\ldots (!!!Siehe Kundengespäch Montag!!!)
  	 \end{enumerate}  
  \end{description}
\end{enumerate}

\end{document}
