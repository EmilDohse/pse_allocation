%%This is a very basic article template.
%%There is just one section and two subsections.
\documentclass[a4paper]{article}
\usepackage{geometry}
\usepackage[ngerman]{babel}
\usepackage[T1]{fontenc}
\usepackage[utf8]{inputenc}
\usepackage{paralist}

\begin{document}
\title{\large Anwendungsfälle - Betreuer}
\author{ }
\maketitle
\pagebreak

\begin{enumerate}%[\textbf{{\textbackslash Z}10{\textbackslash}}]
  
  \item[\textbf{\textbackslash Z10\textbackslash}] 
	 \begin{description}
		\item[Anwendung:] Betreuer meldet sich an
  		\item[Ziel:] Anmeldung des Betreuers
  		\item[Mögliche Varianten]~
  		\begin{enumerate}[1.]
  			\item Man erhält ein vom Admin generiertes Zugangstupel 		(username,password).
  			\item Man erhält von Admin einen Link zum Anmelden.
  			\item Anmeldung via shibboleth. Es wird automatisch erkannt, ob es sich um ein Betreuer- oder Studentenkonto handelt.
  		\end{enumerate}
  	\end{description}
  \pagebreak
  \item[\textbf{\textbackslash Z20\textbackslash}]
	\begin{description}
  		\item[Anwendung:] Thema erstellen
  		\item[Ziel:] Eröffnung eines PSE-Themas mit zugehörigen Gruppen.
  		\item[Vorbedingung:] Thema existiert noch nicht.
  		\item[Nachbedingung(Erfolg):] Thema wird ins System aufgenommen, mit zugehörigem Namen, Beschreibung, Anzahl der angebotenen Gruppen und maximaler Gruppengröße.
  		\item[Nachbedingung(Fehlschlag):] Thema wird nicht ins System aufgenommen.
  		\item[Akteure:] Betreuer
  		\item[Auslösen des Ereignisses:] Betreuer möchte ein PSE-Thema ins System einfügen.
  		\item[Beschreibung:]~
  	\begin{enumerate}[1.]
  	  \item Betreuer befindet sich auf der Website mit den PSE-Themen
  	  \item Betreuer trägt Namen, Beschreibung, Anzahl der Gruppen und maximale Gruppengröße ein.
  	  \item Betreuer fügt Thema in den aktuellen Themenbestand ein.
  	\end{enumerate}
  	\item[Invariante:] Es existieren nie zwei Themen mit gleichem Namen.
  	\item[Erweiterungen:]
  \end{description}

\pagebreak
  \item[\textbf{\textbackslash Z30\textbackslash}]
	\begin{description}
  		\item[Anwendung:] Gruppenbetreuer werden
  		\item[Ziel:] Betreuer einer bestimmten Gruppe werden
  		\item[Vorbedingung:] Man ist noch nicht Betreuer der ausgewählten Gruppe.
  		\item[Nachbedingung(Erfolg):] Erfolgreich als Betreuer der ausgewählten Gruppe geworden.
  		\item[Nachbedingung(Fehlschlag):] Man ist nicht Betreuer der Gruppe.
  		\item[Akteure:] Betreuer
  		\item[Auslösen des Ereignisses:] Betreuer möchte sich einer Gruppe zuordnen.
  		\item[Beschreibung:]~
  	\begin{enumerate}[1.]
  	  \item Betreuer befindet sich auf der Website mit den PSE-Gruppen
  	  \item Betreuer wählt die Gruppe aus, welcher er betreuen möchte.
  	  \item Betreuer tritt der Gruppe bei.
  	\end{enumerate}
  	\item[Erweiterungen:]
  \end{description}
  \pagebreak
  \item[\textbf{\textbackslash Z40\textbackslash}]
	\begin{description}
  		\item[Anwendung:] Gruppe verlassen
  		\item[Ziel:] Eine Gruppe nicht mehr betreuen
  		\item[Vorbedingung:] Man ist Betreuer einer Gruppe, in der noch mindestens ein anderer Betreuer ist.
  		\item[Nachbedingung(Erfolg):] Man betreut die Gruppe nicht länger.
  		\item[Nachbedingung(Fehlschlag):] Man ist immernoch Betreuer der Gruppe.
  		\item[Akteure:] Betreuer
  		\item[Auslösen des Ereignisses:] Betreuer möchte aus Gruppe austreten.
  		\item[Beschreibung:]~
  	\begin{enumerate}[1.]
  	  \item Betreuer befindet sich auf der Website mit den PSE-Gruppen, die er betreut.
  	  \item Betreuer wählt die Gruppe aus, welche er verlassen möchte.
  	  \item Betreuer verlässt die Gruppe.
  	\end{enumerate}
  	\item[Erweiterungen:]
  \end{description}
  \pagebreak
  \item[\textbf{\textbackslash Z50\textbackslash}] \begin{description}
  	\item[Anwendung:] Noten für Gruppenteilnehmer eintragen
  	\item[Ziel:] Speichern der Noten in der Datenbank
  	\item[Vorbedingung:] Gruppe hat Studenten als Teilnehmer
  	\item[Nachbedingung(Erfolg):] Eintragung/Änderung der Noten
  	\item[Nachbedingung(Fehlschlag):] Notenänderung wird verworfen
  	\item[Akteure:] Betreuer
  	\item[Auslösen des Ereignisses:] Betreuer möchte Noten eintragen
  	\item[Beschreibung:]~
  	\begin{enumerate}[1.]
  	  \item Betreuer befindet sich auf der Website mit der Notenübersicht
  	  \item Betreuer trägt neue Noten für eine beliebige Phase ein oder ändert bestehende Noten.
  	\end{enumerate}
  	\item[Invariante:] Noten immer im Bereich von 1.0 bis 5.0
  	\item[Erweiterungen:]
  \end{description}
\end{enumerate}

\end{document}
