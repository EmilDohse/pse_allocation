%%This is a very basic article template.
%%There is just one section and two subsections.
\documentclass[a4paper]{article}
\usepackage{geometry}
\usepackage[ngerman]{babel}
\usepackage[T1]{fontenc}
\usepackage[utf8]{inputenc}
\usepackage{paralist}

\begin{document}
\author{Daniel Biester}
\title{\large Anwendungsfälle - Student}
\maketitle

\begin{enumerate}%[\textbf{{\textbackslash Z}10{\textbackslash}}]
  
  \item[\textbf{\textbackslash Z10\textbackslash}] \begin{description}
  	\item[Anwendung:] Registrierung eines Studenten
  	\item[Ziel:] Speichern der Studentendaten in der Datenbank
  	\item[Vorbedingung:] Studentenregistrierung wurde freigeschaltet
  	\item[Nachbedingung(Erfolg):] Erfolgreiche Registrierung des Studenten
  	\item[Nachbedingung(Fehlschlag):] Student ist weiterhin nicht registriert
  	\item[Akteure:] Student
  	\item[Auslösen des Ereignisses:] Student will sich registrieren
  	\item[Beschreibung:]~
  	\begin{enumerate}[1.]
  	  \item Student befindet sich auf der Website zum Registrieren
  	  \item Student füllt das Formular zum Registrieren aus und schickt es ab
  	\end{enumerate}
  	\item[Erweiterungen:]
  	\item[Alternativen:]~
  	\begin{enumerate}
  	  \item[2a)] Ist der Student noch nicht in der Datenbank, so wird ein neuer Eintrag erstellt und seine Daten werden gespeichert
  	  \item[3a)] Ist ein Eintrag in der Datenbank für diesen Studenten schon vorhanden, so wird kein weiterer Eintrag hinzugefügt, sondern die zweite Registrierung abgebrochen
  	\end{enumerate} 
  \end{description}
  \pagebreak
  
  \item[\textbf{\textbackslash Z20\textbackslash}] \begin{description}
  \item[Anwendung:] Anmeldung eines Studenten
  \item[Ziel:] Student wird angemeldet und kann seine Bewertung usw ändern
  	\item[Vorbedingung:] Der Student hat sich vorher registriert
  	\item[Nachbedingung(Erfolg):] Der Student ist angemeldet
  	\item[Nachbedingung(Fehlschlag):] Der Student ist nicht angemeldet
  	\item[Akteure:] Student
  	\item[Auslösen des Ereignisses:] Student will sich anmelden
  	\item[Beschreibung:]~
  	\begin{enumerate}[1.]
  	  \item Student öffnet die Website und füllt das Formular zum anmelden aus und schickt dieses ab
  	  \item Der Server prüft, ob der Student in der Datenbank eingetragen ist und ob das Passwort korrekt ist
  	\end{enumerate}
  	\item[Erweiterungen:]
  	\item[Alternativen:]
	\begin{enumerate}
  	  \item[a)] Student wurde erfolgreich angemeldet und gelangt ins Studentenportal
  	  \item[b)] Die Anmeldung ist fehlgeschlagen und der Student befindet sich auf der Haupseite
  	\end{enumerate}
  \end{description}
  \pagebreak
  
  \item[\textbf{\textbackslash Z30\textbackslash}] \begin{description}
  \item[Anwendung:] Student erstellt Lerngruppe
  \item[Ziel:] Eine neue Lerngruppe ist in der Datenbank vermerkt und der Ersteller ist erstes Mitglied dieser
  	\item[Vorbedingung:] Student ist angemeldet und befindet sich auf der Website im Studentenportal
  	\item[Nachbedingung(Erfolg):] Die Lerngruppe wurde erfolgreich erstellt
  	\item[Nachbedingung(Fehlschlag):] Die Lerngruppe wurde nicht erstellt
  	\item[Akteure:] Student
  	\item[Auslösen des Ereignisses:] Wille des Studenten
  	\item[Beschreibung:]~
  	\begin{enumerate}[1.]
  	  \item Student gibt Name und Passwort in das Formular zum erstellen einer Lerngruppe ein und schickt dieses ab
  	  \item Die Lerngruppe wird in der Datenbank angelegt und der Ersteller wird erstes Mitglied
  	\end{enumerate}
  	\item[Erweiterungen:]
  	\item[Alternativen:] ~
  	\begin{enumerate}
  	  \item[a)] Eine Lerngruppe mit diesem Namen existiert schon. Dann wird keine weitere Lerngruppe angelegt und dem Studenten wird mitgeteilt, dass er den Namen der Lerngruppe verändern soll
  	 \end{enumerate}  
  \end{description}
  \pagebreak
  
  \item[\textbf{\textbackslash Z40\textbackslash}] \begin{description}
  \item[Anwendung:] Student tritt Lerngruppe bei
  \item[Ziel:] Student ist Mitglied einer Lerngruppe
  	\item[Vorbedingung:] Student ist angemeldet und befindet sich auf der Website im Studentenportal
  	\item[Nachbedingung(Erfolg):] Der Student ist Mitglied einer Lerngruppe
  	\item[Nachbedingung(Fehlschlag):] Der Student ist nicht Mitglied einer Lerngruppe
  	\item[Akteure:] Student
  	\item[Auslösen des Ereignisses:] Der Student will einer Lerngruppe beitreten
  	\item[Beschreibung:]~
  	\begin{enumerate}[1.]
  	  \item Der Student weiß den Namen und das Passwort einer Lerngruppe und gibt diese in das passende Formular ein und schickt dieses ab
  	  \item In der Datenbank wird eine Lerngruppe mit der Name/Passwort Kombination gesucht, falls eine gefunden wurde, so wird die Mitgliederliste um den Studenten erweitert
  	\end{enumerate}
  	\item[Erweiterungen:]~
  	\begin{enumerate}
  	  \item Der Lerngruppenersteller wird per E-Mail informiert
  	 \end{enumerate}
  	\item[Alternativen:] ~
  	\begin{enumerate}
  	  \item[a)] Befindet sich keine passende Lerngruppe in der Datenbank, so schlägt der Beitritt fehl und der Student ist weiterhin kein Mitglied einer Lerngruppe
  	 \end{enumerate}  
  \end{description}
  \pagebreak
  
  \item[\textbf{\textbackslash Z50\textbackslash}] \begin{description}
  \item[Anwendung:] Student bewertet Projekte
  \item[Ziel:] Die Lerngruppe assoziiert mit dem Studenten hat eine Bewertung abgegeben
  	\item[Vorbedingung:] Der Student ist angemeldet und im Studentenportal der Website
  	\item[Nachbedingung(Erfolg):] Für die Lerngruppe ist eine Bewertung eingetragen
  	\item[Nachbedingung(Fehlschlag):] Die Lerngruppe hat weiterhin keine (neue) Bewertung eingetragen
  	\item[Akteure:] Student
  	\item[Auslösen des Ereignisses:] Des Studentens Wille
  	\item[Beschreibung:]~
  	 \begin{enumerate}[1.]
  	   \item Der Student sieht in der Bewertungsliste die aktuelle eingetragene Bewertung (falls keine vorhanden, dann eine Standard-Bewertung) und ändert diese nach seinem Willen
  	   \item Der Student schließt die Bewertung ab
  	 \end{enumerate}
  	\item[Erweiterungen:]~
  	\item[Alternativen:] ~
  \end{description}
\end{enumerate}

\end{document}
