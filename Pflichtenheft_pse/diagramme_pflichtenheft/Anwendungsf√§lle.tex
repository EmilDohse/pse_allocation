%%This is a very basic article template.
%%There is just one section and two subsections.
\documentclass[a4paper]{article}
\usepackage{geometry}
\usepackage[ngerman]{babel}
\usepackage[T1]{fontenc}
\usepackage[utf8]{inputenc}
\usepackage{paralist}

\begin{document}
\author{Philipp Loth}
\title{\large Anwendungsfälle - Admin}
\maketitle

\begin{enumerate}%[\textbf{{\textbackslash Z}10{\textbackslash}}]
  
  \item[\textbf{\textbackslash Z10\textbackslash}] \begin{description}
  	\item[Anwendung:] Hinzufügen/Ändern eines Projekts
  	\item[Ziel:] Einfügen/Änderung von Projektdaten in der Datenbank
  	\item[Vorbedingung:] -keine-
  	\item[Nachbedingung(Erfolg):] Neue Projektdaten sind eingetragen
  	\item[Nachbedingung(Fehlschlag):] Neue Projektdaten sind nicht eingetragen
  	\item[Akteure:] Administrator, Institut
  	\item[Auslösen des Ereignisses:] Administrator bekommt Projektdaten von einem
  	Institut\\ vorgelegt
  	\item[Beschreibung:]~
  	\begin{enumerate}[1.]
  	  \item Projektübersicht öffnen
  	  \item Wenn das Projekt schon existiert, so werden die Projektdaten
  	  verglichen
  	  \item Stimmen die Projektdaten nicht überein, so werden die betroffenen
  	  Daten\\ angepasst
  	  \item Durch Abspeichern der Eingaben werden die Daten in der Datenbank
  	  angepasst
  	\end{enumerate}
  	\item[Erweiterungen:]
  	\item[Alternativen:]~
  	\begin{enumerate}
  	  \item[2a)] Wenn nicht, dann wird das Projekt neu angelegt
  	  \item[3a)] Falls doch, so ist dieser Anwendungsfall abgeschlossen
  	\end{enumerate} 
  \end{description}
  \pagebreak
  
  \item[\textbf{\textbackslash Z20\textbackslash}] \begin{description}
  \item[Anwendung:] Löschen eines Projekts
  \item[Ziel:] Löschen eines Projekts in die Datenbank
  	\item[Vorbedingung:] Projekt existiert bereits in der Datenbank
  	\item[Nachbedingung(Erfolg):] Projekt ist aus der Datenbank entfernt
  	\item[Nachbedingung(Fehlschlag):] Projekt ist immernoch in der Datenbank
  	\item[Akteure:] Administrator, Institut
  	\item[Auslösen des Ereignisses:] Institut zieht ein Projekt zurück oder es
  	fehlen Daten bei Beginn der Einteilung
  	\item[Beschreibung:]~
  	\begin{enumerate}[1.]
  	  \item Projektübersicht öffnen
  	  \item Projekt aus Liste entfernen
  	  \item Projektdaten werden aus der Datenbank entfernt
  	\end{enumerate}
  	\item[Erweiterungen:]
  	\item[Alternativen:]
  \end{description}
  \pagebreak
  
  \item[\textbf{\textbackslash Z30\textbackslash}] \begin{description}
  \item[Anwendung:] Teameinteilung
  \item[Ziel:] Finden einer Teameinteilung nach ausgewählten Parametern
  	\item[Vorbedingung:] Alle Projekte in der Datenbank sind vollständig
  	\item[Nachbedingung(Erfolg):] Es wurde eine Teameinteilung gefunden
  	\item[Nachbedingung(Fehlschlag):] Es wurde keine Teameinteilung gefunden
  	\item[Akteure:] Administrator
  	\item[Auslösen des Ereignisses:] Entscheidung des Administrators bzw.
  	zeitliche Deadline
  	\item[Beschreibung:]~
  	\begin{enumerate}[1.]
  	  \item Einstellen der Parameter nach den eingeteilt werden soll
  	  \item Starten der Berechnung über Schaltfläche
  	  \item Sind alle Projekte vollständig angegeben, so startet die Berechnung
  	  \item Nach Abschluss der Berechnung wird das Ergebnis inklusive
  	  eingestellter Parameter abgespeichert.
  	\end{enumerate}
  	\item[Erweiterungen:]
  	\item[Alternativen:] ~
  	\begin{enumerate}
  	  \item[3a)] Wenn nicht, so startet die Berechnung nicht und der Fall ist
  	  beendet
  	  \item[4a)] Hier ist auch ein frühzeitiger Abbruch durch den Administrator
  	  möglich. Dies wird im Ergebnis vermerkt.
  	 \end{enumerate}  
  \end{description}
  \pagebreak
  
  \item[\textbf{\textbackslash Z40\textbackslash}] \begin{description}
  \item[Anwendung:] Finale Auswahl der Einteilung
  \item[Ziel:] Finden der bestmöglichen Teameinteilung
  	\item[Vorbedingung:] Es wurde mindestens eine Berechnung durchgeführt
  	\item[Nachbedingung(Erfolg):] Es wurde eine Teameinteilung ausgewählt
  	\item[Nachbedingung(Fehlschlag):] Es wurde keine Teameinteilung ausgewählt
  	\item[Akteure:] Administrator
  	\item[Auslösen des Ereignisses:] Entscheidung des Administrators bzw.
  	zeitliche Deadline
  	\item[Beschreibung:]~
  	\begin{enumerate}[1.]
  	  \item Der Administrator wählt eine der berechneten Einteilung final aus
  	  \item Wurde ein Student zugewiesen, so wird seinem Datenbankeintrag sein
  	  Team hinzugefügt
  	\end{enumerate}
  	\item[Erweiterungen:]~
  	\begin{enumerate}
  	  \item[nach 2] Betreuer werden über ihre Teams informiert
  	  \item[nach 2] Die Studenten werden über ihre Einteilung informiert
  	 \end{enumerate}
  	\item[Alternativen:] ~
  	\begin{enumerate}
  	  \item[2a)] Wird ein Student nicht zugewiesen, so wird dies ebenfalls in der
  	  Datenbank vermerkt (!!!Wie? Nochmal Besprechen am Montag?!!!)
  	 \end{enumerate}  
  \end{description}
  \pagebreak
  
  \item[\textbf{\textbackslash Z50\textbackslash}] \begin{description}
  \item[Anwendung:] Einteilung eines Studenten
  \item[Ziel:] Manuelle Einteilung eines Studenten
  	\item[Vorbedingung:] Automatische Teameinteilung ist abgeschlossen
  	\item[Nachbedingung(Erfolg):] Der Student wurde hinzugefügt
  	\item[Nachbedingung(Fehlschlag):] Der Student wurde nicht hinzugefügt
  	\item[Akteure:] Administrator, Student
  	\item[Auslösen des Ereignisses:] Student meldet sich beim Administrator
  	\item[Beschreibung:]~
  	 \begin{enumerate}[1.]
  	   \item Wenn sich der Student noch nicht im System befindet überprüft der
  	   Administrator, ob die Anforderungen für das PSE erfüllt sind
  	   \item Hat der Student die Anforderungen erfüllt, so versucht der
  	   Administrator ihn einem Team zuzuteilen
  	   \item Hat der Administrator ein Team gefunden und sich dabei an die
  	   generellen Regeln gehalten, so wird der Student über eine Eingabemaske der
  	   Datenbank hinzugefügt
  	 \end{enumerate}
  	\item[Erweiterungen:]~
  	 \begin{enumerate}
  	   \item [nach 3] Betreuer, sowie andere Mitglieder des Projekts, werden
  	   über das neue Mitglied per E-Mail informiert 
  	 \end{enumerate}  
  	\item[Alternativen:] ~
  	 \begin{enumerate}
  	  \item[1a)] Falls doch, so wurde er schon vorher abgelehnt/nicht zugewiesen
  	  und wird nicht eingeteilt. Dieser Fall ist damit abgeschlossen
  	  \item [2a)] Falls er die Anforderungen nicht erfüllt, so wird er auch nicht
  	  zugeteilt und dieser Fall ist abgeschlossen
  	  \item[3a)] Wenn nicht\ldots (!!!Siehe Kundengespäch Montag!!!)
  	 \end{enumerate}  
  \end{description}



Ab hier von Emil aus Studentensicht

\item[\textbf{\textbackslash Z60\textbackslash}] \begin{description}
    \item[Anwendung:] Registrierung
    \item[Ziel:] Registrierung bei der Seite zur PSE Einteling
    \item[Vorbedingung:] Student muss am KIT eingeschrieben sein
    \item[Nachbedingung(Erfolg):] Studierender kann seine E-mail verifizieren
    \item[Nachbedingung(Fehlschlag):] Studierender muss sich erneut
    registrieren.
    \item[Akteure:] Student
    \item[Auslösen des Ereignisses:] Studierende werden bei der PSE
    Einführungsveransttaltung darauf hingewiesen sich zu Registrieren.
    \item[Beschreibung:]~
    \begin{enumerate}[1.]
      \item Öffnen der Internetseite
      \item Klick auf Registrieren
      \item Anmeldung mit dem u-Account.
      \item Eingabe des Namen's und der E-Mail Adresse 
      \item Durch Abspeichern der Eingaben wird vom Produkt eine
      Verifikations-E-mail an die vom Studierenden angegebene Adresse geschickt.
    \end{enumerate}
    \item[Erweiterungen:]
    \item[Alternativen:]~ 
      \end{description}
  \pagebreak
  
  \item[\textbf{\textbackslash Z70\textbackslash}] \begin{description}
    \item[Anwendung:] Verifikation der E-Mail
    \item[Ziel:] Verifikation der E-mail
    \item[Vorbedingung:] Student muss sich registriert haben und seine E-mail
    korrekt angegeben haben.
    \item[Nachbedingung(Erfolg):] Studierender kann sich nun Anmelden und hat
    Zugriff auf die Studierenden-Sicht
    \item[Nachbedingung(Fehlschlag):] 
    \begin{itemize} 
      \item Falls die E-mail falsch angegeben wurde muss sich der Student erneut
      registrieren.
      \item Falls die E-mail richtg war, jedoch keien E-mail angekommen ist muss
      eine neue E-mail auf der Registrieungsseite angefordert werden.
    \end{itemize}
    \item[Akteure:] Student
    \item[Auslösen des Ereignisses:] Studenten registrieren sich und werden
    darauf hingewiesn inhre E-mail zu verifizieren.
    \item[Beschreibung:]~
    \begin{enumerate}[1.]
      \item E-mails über Drittsoftware abrufen.
      \item Klick auf den Verifikations-Link in der E-mail.
      
    \end{enumerate}
    \item[Erweiterungen:]
    \item[Alternativen:]~
      \end{description}
  \pagebreak
  
  \item[\textbf{\textbackslash Z80\textbackslash}] \begin{description}
    \item[Anwendung:] Anmeldung
    \item[Ziel:] Zugriff auf die Studierenden-Sicht
    \item[Vorbedingung:] Studierender muss seine E-Mail verifiziert haben.
    \item[Nachbedingung(Erfolg):] Studierender kann nun die bestandenen Fächer
    eintragen und falls dies bereits geschehen: einer Lerngruppe beitreten, eine
    Lerngruppe erstellen oder eigene Prioritäten vergeben.
    \item[Nachbedingung(Fehlschlag):] Der Studierende hat keinen Zugriff auf die
    Studierenden-Sicht
    
    \item[Akteure:] Student
    \item[Auslösen des Ereignisses:]Nach der verifikation der E-mail werden
    Studierende darauf hingewiesen das eine Anmeldung nun möglich ist.
    \item[Beschreibung:]~
    \begin{enumerate}[1.]
      \item Öffnen der Internetseite zur PSE-Einteilung
      \item Anmeldung mit dem u-Account.
      
      
    \end{enumerate}
    \item[Erweiterungen:]
    \item[Alternativen:]~
      \end{description}
  \pagebreak
  
    \item[\textbf{\textbackslash Z90\textbackslash}] \begin{description}
    \item[Anwendung:] Eintragen bestandener Fächer
    \item[Ziel:] Eintragen der bestandenen Fächer.
    \item[Vorbedingung:] Studierender muss angemeldet sein
    \item[Nachbedingung(Erfolg):] Studierender kann nun einer Lerngruppe
    beitreten, eine Lernggruppe erstellen oder eigene Prioritäten vergeben.
    \item[Nachbedingung(Fehlschlag):] Das Eintragen muss wiederholt werden.
    
    \item[Akteure:] Student
    \item[Auslösen des Ereignisses:]Bei der ersten Anmeldung werden Studierende
    darauf hingewiesen Bestandene Fächer einzutragen.
    \item[Beschreibung:]~
    \begin{enumerate}[1.]
      \item Klick auf Menüpunkt bestandene Fächer
      \item Auswahl bestandener Fächer über das Setzen von Häckchen.
      \item Speichern über einen klick auf den "`Speichern"'-Knopf
      
    \end{enumerate}
    \item[Erweiterungen:]
    \item[Alternativen:]~
      \end{description}
  \pagebreak
  
  
      \item[\textbf{\textbackslash Z100\textbackslash}] \begin{description}
    \item[Anwendung:] Erstellen einer Lerngrupppe
    \item[Ziel:] Erstellen einer Lerngurppe welche später zusammen eingeteilt
    wird.
    \item[Vorbedingung:] Studierender muss angemeldet sein und bestandene Fächer
    eingetragen haben.
    \item[Nachbedingung(Erfolg):] Studierender kann nun Prioritäten der
    Lerngruppe eingeben.
    \item[Nachbedingung(Fehlschlag):] Studierender wird auf den Fehlschlag
    hingewiesen und befindet sich wieder im Zustand der Vorbedingung.
    
    \item[Akteure:] Student
    \item[Auslösen des Ereignisses:]Der Student möchte mit Komilitonen eine
    Lerngruppe gründen.
    \item[Beschreibung:]~
    \begin{enumerate}[1.]
      \item Klick auf Menüpunkt Lerngruppen.
      \item Klick auf "`Erstelle neue Lerngruppe"'
      \item Eingabe eises Gruppennamens, eines Passworts und der
      voraussichtlichen Gruppengröße.
      \item Klick auf Speichern.
      
    \end{enumerate}
    \item[Erweiterungen:]
    \item[Alternativen:]~
      \end{description}
  \pagebreak
  
    \item[\textbf{\textbackslash Z100\textbackslash}] \begin{description}
    \item[Anwendung:] Beitreten einer Lerngruppe
    \item[Ziel:] Beitreten einer Lerngruppe, welche später zusammen eingeteilt
    wird.
    \item[Vorbedingung:] Ein Komilitone muss eine Lerngruppe erstellt haben und
    dem Studierenden die Anmeldedaten bestehend aus Gruppenname und Passwort
    mitgeteilt haben.
    \item[Nachbedingung(Erfolg):] Studierender kann nun Prioritäten der
    Lerngruppe eingeben.
    \item[Nachbedingung(Fehlschlag):] Studierender wird auf den Fehlschlag
    hingewiesen und befindet sich wieder im Zustand der Vorbedingung.
    
    \item[Akteure:] Student
    \item[Auslösen des Ereignisses:]Der Studierende möchte einer bestehenden
    Lerngrruppe beitreten.
    \item[Beschreibung:]~
    \begin{enumerate}[1.]
      \item Klick auf Menüpunkt Lerngruppen.
      \item Klick auf "`Lerngruppe beitreten"'
      \item Eingabe eises Gruppennamens, eines Passworts
      \item Klick auf Beitreten.
      
    \end{enumerate}
    \item[Erweiterungen:]
    \item[Alternativen:]~
      \end{description}
  \pagebreak
  
     \item[\textbf{\textbackslash Z100\textbackslash}] \begin{description}
    \item[Anwendung:] Eingabe der Projekt-Prioritäten
    \item[Ziel:] Studierende wollen mitteilen, welchen Projekten sie wie gerne
    zugeordnet würden.
    \item[Vorbedingung:] Studeirender muss angemeldet sein. Falls die
    Prioritäten einer Lerngruppe festgelegt werden sollen muss der Studerende
    mitglied einer Lerngruppe sein.
    \item[Nachbedingung(Erfolg):] Prioritäten werden bei der Einteilung
    berücksichtigt.
    \item[Nachbedingung(Fehlschlag):] Studierender muss die Prioritäten erneut
    eingeben.
    
    \item[Akteure:] Student
    \item[Auslösen des Ereignisses:]Der Studierende möchte seine Prioritäten
    oder die Prioritäten seiner Lerngruppe festlegen.
    \item[Beschreibung:]~
    \begin{enumerate}[1.]
      \item Klick auf Menüpunkt Projektbewertung.
      \item Bewertung der Projekte über das setzen von Häckchen bei "`++"',
      "`+"', "`0"', "`-"' und "`--"'.
      

\item Klick auf Speichern
      
    \end{enumerate}
    \item[Erweiterungen:]
    \item[Alternativen:]~
      \end{description}
  \pagebreak
\end{enumerate}
\end{document}
