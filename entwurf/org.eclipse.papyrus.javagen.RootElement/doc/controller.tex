\begin{texdocpackage}{controller}
\label{texdoclet:controller}

\begin{texdocclass}{class}{AdminEditAllocationController}
\label{texdoclet:controller.AdminEditAllocationController}
\begin{texdocclassintro}
Dieser Controller ist für das Bearbeiten der Http-Requests zuständig, welche
 beim Editieren einer Einteilung abgeschickt werden.\end{texdocclassintro}
\begin{texdocclassfields}
\texdocfield{public}{EditAllocationCommand}{undoStack}{}
\end{texdocclassfields}
\begin{texdocclassconstructors}
\texdocconstructor{public}{AdminEditAllocationController}{()}{}{}
\end{texdocclassconstructors}
\begin{texdocclassmethods}
\texdocmethod{public}{Result}{duplicateAllocation}{()}{Diese Methode erstellt eine Kopie einer kompletten Einteilung. Diese
 Funktion ist dafür gedacht, dass der Administrator sehen kann, ob durch
 seine manuelle Änderungen ein besseres Ergebnis entstand. Der
 Administrator wird anschließend auf die Seite zur Einteilungs-Bearbeitung
 zurückgeleitet.}{\texdocreturn{Die Seite, die als Antwort verschickt wird.}
}
\texdocmethod{public}{Result}{moveStudents}{()}{Diese Methode verschiebt einen oder mehrere ausgewählte Studenten in ein
 anderes Team. Das Verschieben in das gleiche Team wird nicht unterbunden,
 hat jedoch keine Auswirkung. Anschließend wird der Administrator auf die
 Seite zur Einteilungs-Bearbeitung zurückgeleitet.}{\texdocreturn{Die Seite, die als Antwort verschickt wird.}
}
\texdocmethod{public}{Result}{publishAllocation}{()}{Diese Methode veröffentlicht eine Einteilung. Dazu gehört, die Einteilung
 als final zu deklarieren und Betreuer und Studenten per E-Mail über deren
 Einteilung zu informieren. Der Administrator wird anschließend auf die
 Einteilungs-Bearbeitungs-Seite zurückgeleitet.}{\texdocreturn{Die Seite, die als Antwort verschickt wird.}
}
\texdocmethod{public}{Result}{removeAllocation}{()}{Diese Methode löscht eine bereits vorhandene Einteilung. Der
 Administrator wird anschließend auf die Seite zur Einteilungs-Bearbeitung
 zurückgeleitet.}{\texdocreturn{Die Seite, die als Antwort verschickt wird.}
}
\texdocmethod{public}{Result}{swapStudents}{()}{Diese Methode tauscht zwei Studenten, welche der Administrator vorher in
 einem Formular ausgewählt hat. Ein Tausch innerhalb eines Teams wird
 nicht unterbunden, hat jedoch keine Auswirkung. Anschließend wird der
 Administrator auf die Seite zur Einteilungs-Bearbeitung zurückgeleitet.}{\texdocreturn{Die Seite, die als Antwort verschickt wird.}
}
\texdocmethod{public}{Result}{undoAllocationEdit}{()}{Diese Methode macht die letzte Editierung rückgängig. Dies ist jedoch
 nicht session-übergreifend möglich. Der Administrator wird anschließend
 auf die Seite zur Einteilungs-Bearbeitung zurückgeleitet.}{\texdocreturn{Die Seite, die als Antwort verschickt wird.}
}
\end{texdocclassmethods}
\end{texdocclass}


\begin{texdocclass}{class}{AdminImportExportController}
\label{texdoclet:controller.AdminImportExportController}
\begin{texdocclassintro}
Dieser Controller ist für das Bearbeiten der Http-Requests zuständig, welche
 Importieren und Exportieren auf der Import$/$Export-Seite regeln.\end{texdocclassintro}
\begin{texdocclassconstructors}
\texdocconstructor{public}{AdminImportExportController}{()}{}{}
\end{texdocclassconstructors}
\begin{texdocclassmethods}
\texdocmethod{public}{Result}{exportAllocation}{()}{Diese Methode lässt den Administrator eine csv-Datei downloaden, welche
 eine Einteilung speichert. Der Administrator wird daraufhin auf die
 Import$/$Export-Seite zurückgeleitet.}{\texdocreturn{Die Seite, die als Antwort verschickt wird.}
}
\texdocmethod{public}{Result}{exportCMSData}{()}{Diese Methode lässt den Administrator eine csv-Datei downloaden, welche
 die Studenten des aktuellen Semester mit den eingetragenen TSE und PSE
 Noten enthält. Der Administrator wird daraufhin auf die
 Import$/$Export-Seite zurückgeleitet.}{\texdocreturn{Die Seite, die als Antwort verschickt wird.}
}
\texdocmethod{public}{Result}{exportProjects}{()}{Diese Methode lässt den Administrator eine csv-Datei downloaden, welche
 alle Projekte des aktuellen Semesters abspeichert. Der Administrator wird
 daraufhin auf die Import$/$Export-Seite zurückgeleitet.}{\texdocreturn{Die Seite, die als Antwort verschickt wird.}
}
\texdocmethod{public}{Result}{exportSPO}{()}{Diese Methode lässt den Administrator eine csv-Datei downloaden, welche
 eine SPO speichert. Der Administrator wird daraufhin auf die
 Import$/$Export-Seite zurückgeleitet.}{\texdocreturn{Die Seite, die als Antwort verschickt wird.}
}
\texdocmethod{public}{Result}{exportStudents}{()}{Diese Methode lässt den Administrator eine csv-Datei downloaden, welche
 alle Studenten des aktuellen Semesters abspeichert. Der Administrator
 wird daraufhin auf die Import$/$Export-Seite zurückgeleitet.}{\texdocreturn{Die Seite, die als Antwort verschickt wird.}
}
\texdocmethod{public}{Result}{importAllocation}{()}{Diese Methode importiert eine Einteilung, sodass sie in der
 Einteilungsübersicht erscheint. Der Administrator wird daraufhin auf die
 Import$/$Export-Seite zurückgeleitet.}{\texdocreturn{Die Seite, die als Antwort verschickt wird.}
}
\texdocmethod{public}{Result}{importCMSData}{()}{Diese Methode importiert eine csv-Datei mit Daten aus dem CMS
 (CampusManagementSystem) und fügt die Daten zu den bereits vorhandenen
 hinzu. Der Administrator wird daraufhin auf die Import$/$Export-Seite
 zurückgeleitet.}{\texdocreturn{Die Seite, die als Antwort verschickt wird.}
}
\texdocmethod{public}{Result}{importProjects}{()}{Diese Methode importiert eine Liste an Projekten, welche daraufhin zum
 aktuellen Semester hinzugefügt werden. Der Administrator wird daraufhin
 auf die Import$/$Export-Seite zurückgeleitet.}{\texdocreturn{Die Seite, die als Antwort verschickt wird.}
}
\texdocmethod{public}{Result}{importSPO}{()}{Diese Methode importiert eine SPO, sodass sie in der SPO-Auswahl eines
 Semesters erscheint. Der Administrator wird daraufhin auf die
 Import$/$Export-Seite zurückgeleitet.}{\texdocreturn{Die Seite, die als Antwort verschickt wird.}
}
\texdocmethod{public}{Result}{importStudents}{()}{Diese Methode importiert eine Liste an Studenten, welche daraufhin zum
 aktuelle Semester hinzugefügt werden. Der Administrator wird daraufhin
 auf die Import$/$Export-Seite zurückgeleitet.}{\texdocreturn{Die Seite, die als Antwort verschickt wird.}
}
\end{texdocclassmethods}
\end{texdocclass}


\begin{texdocclass}{class}{AdminPageController}
\label{texdoclet:controller.AdminPageController}
\begin{texdocclassintro}
Dieser Controller ist zuständig für das Bearbeiten der Http-Requests, welche
 durch das Klicken eines Links und nicht eines Buttons versendet werden.\end{texdocclassintro}
\begin{texdocclassconstructors}
\texdocconstructor{public}{AdminPageController}{()}{}{}
\end{texdocclassconstructors}
\begin{texdocclassmethods}
\texdocmethod{public}{Result}{adviserPage}{()}{Diese Methode gibt die Seite zurück, auf der der Administrator alle
 Projektbetreuer sehen, neue hinzufügen oder bereits existierende
 entfernen kann.}{\texdocreturn{Die Seite, die als Antwort verschickt wird.}
}
\texdocmethod{public}{Result}{allocationPage}{()}{Diese Methode gibt die Seite zurück, auf der der Administrator
 Einteilungen berechnen und vorher Parameter einstellen kann. Außerdem
 sieht er noch zu berechnende Konfigurationen und kann diese aus der
 Berechnungsliste entfernen.}{\texdocreturn{Die Seite, die als Antwort verschickt wird.}
}
\texdocmethod{public}{Result}{exportImportPage}{()}{Diese Methode gibt die Seite zurück, auf der der Administrator
 Einteilungen, Studentendaten, SPOs, Projekte und CMS-Daten ex- und
 importieren kann.}{\texdocreturn{Die Seite, die als Antwort verschickt wird.}
}
\texdocmethod{public}{Result}{projectEditPage}{(String name)}{Diese Methode gibt die Seite zurück, auf der der Administrator ein Projekt editieren kann.}{\begin{texdocparameters}
\texdocparameter{name}{Der Name des Projektes (da mitgegeben über die URL, ist der String encoded)}
\end{texdocparameters}
\texdocreturn{Die Seite, die als Antwort verschickt wird.}
}
\texdocmethod{public}{Result}{projectPage}{()}{Diese Methode gibt die Seite zurück, auf der der Administrator Projekte
 sieht, neue hinzufügen, sowie existierende löschen kann.}{\texdocreturn{Die Seite, die als Antwort verschickt wird.}
}
\texdocmethod{public}{Result}{propertiesPage}{()}{Diese Methode gibt die Seite zurück, auf der der Administrator die
 Semester-Einstellungen vornehmen kann.}{\texdocreturn{Die Seite, die als Antwort verschickt wird.}
}
\texdocmethod{public}{Result}{resultsPage}{()}{Diese Methode gibt die Seite zurück, auf der der Administrator die
 Ergebnisse der Berechnungen sehen, vergleichen und editieren kann.}{\texdocreturn{Die Seite, die als Antwort verschickt wird.}
}
\texdocmethod{public}{Result}{studentEditPage}{()}{Diese Methode gibt die Seite zurück, auf der der Administrator Studenten
 manuell hinzufügen oder löschen kann.}{\texdocreturn{Die Seite, die als Antwort verschickt wird.}
}
\end{texdocclassmethods}
\end{texdocclass}


\begin{texdocclass}{class}{AdminProjectController}
\label{texdoclet:controller.AdminProjectController}
\begin{texdocclassintro}
Dieser Controller ist für das Bearbeiten der Http-Requests zuständig, welche
 beim Hinzufügen, Ändern und Löschen eines Projektes im Administratorbereich
 abgeschickt werden.\end{texdocclassintro}
\begin{texdocclassconstructors}
\texdocconstructor{public}{AdminProjectController}{()}{}{}
\end{texdocclassconstructors}
\begin{texdocclassmethods}
\texdocmethod{public}{Result}{addProject}{()}{Diese Methode fügt ein neues Projekt in das System ein und leitet den
 Administrator zurück auf die Seite zum Editieren des Projektes.}{\texdocreturn{Die Seite, die als Antwort verschickt wird.}
}
\texdocmethod{public}{Result}{editProject}{()}{Diese Methode editiert ein bereits vorhandenes Projekt. Die zu
 editierenden Daten übermittelt der Administrator über ein Formular,
 welches er zum Editieren abschickt. Anschließend wird der Administrator
 auf die Seite zum Hinzufügen und Editieren von Projekten weitergeleitet.}{\texdocreturn{Die Seite, die als Antwort verschickt wird.}
}
\texdocmethod{public}{Result}{removeProject}{()}{Diese Methode löscht ein Projekt und alle dazugehörenden Daten aus dem
 System und leitet den Administrator weiter auf die Seite zum Editieren
 und Hinzufügen von Projekten.}{\texdocreturn{Die Seite, die als Antwort verschickt wird.}
}
\end{texdocclassmethods}
\end{texdocclass}


\begin{texdocclass}{class}{AdminPropertiesController}
\label{texdoclet:controller.AdminPropertiesController}
\begin{texdocclassintro}
Dieser Controller ist für das Bearbeiten der Http-Requests zuständig, welche
 beim Ändern der Einstellungen abgeschickt werden.\end{texdocclassintro}
\begin{texdocclassconstructors}
\texdocconstructor{public}{AdminPropertiesController}{()}{}{}
\end{texdocclassconstructors}
\begin{texdocclassmethods}
\texdocmethod{public}{Result}{addAchievement}{()}{Diese Methode fügt eine neue Teilleistung zu einer bereits vorhandenen
 SPO hinzu. Der Administrator kann die Teilleistung als notwendig oder als
 nicht notwendig deklarieren und deren Namen ändern. Der Administrator
 wird daraufhin zur Einstellungsseite zurückgeleitet.}{\texdocreturn{Die Seite, die als Antwort verschickt wird.}
}
\texdocmethod{public}{Result}{addSemester}{()}{Diese Methode lässt den Administrator ein neues Semester erstellen und
 anschließend konfigurieren. Nach dem Erstellen wird der Administrator
 deshalb auf die Einstellungsseite für das Semester weitergeleitet.}{\texdocreturn{Die Seite, die als Antwort verschickt wird.}
}
\texdocmethod{public}{Result}{addSPO}{()}{Diese Methode fügt eine neue leere SPO, mit einem vom Administrator
 bestimmten Namen, hinzu. Der Administrator wird daraufhin auf die
 Einstellungsseite zurückgeleitet.}{\texdocreturn{Die Seite, die als Antwort verschickt wird.}
}
\texdocmethod{public}{Result}{editSemester}{()}{Diese Methode übernimmt die Änderungen, welche der Administrator im
 Semester-ändern-Formular festgelegt hat. Dazu gehören die Deadlines und
 die allgemeinen Informationen.}{\texdocreturn{Die Seite, die als Antwort verschickt wird.}
}
\texdocmethod{public}{Result}{removeAchievement}{()}{Diese Methode löscht eine bereits existierende Teilleistung aus einer
 SPO. Der Administrator wird daraufhin zur Einstellungsseite
 zurückgeleitet.}{\texdocreturn{Die Seite, die als Antwort verschickt wird.}
}
\texdocmethod{public}{Result}{removeSemester}{()}{Diese Methode lässt den Administrator ein Semester löschen, wenn mit
 diesem keine Studentendaten verbunden sind. Der Administrator wird
 daraufhin zur Einstellungsseite zurückgeleitet.}{\texdocreturn{Die Seite, die als Antwort verschickt wird.}
}
\texdocmethod{public}{Result}{removeSPO}{()}{Diese Methode löscht eine bereits vorhandene SPO. Die SPO kann nur
 gelöscht werden, wenn kein Student diese SPO verwendet. Der Administrator
 wird daraufhin auf die Einstellungsseite zurückgeleitet.}{\texdocreturn{Die Seite, die als Antwort verschickt wird.}
}
\end{texdocclassmethods}
\end{texdocclass}


\begin{texdocclass}{class}{AdviserPageController}
\label{texdoclet:controller.AdviserPageController}
\begin{texdocclassintro}
Dieser Controller ist für das Bearbeiten der Http-Requests zuständig, welche
 im Betreuerbereich aufkommen.\end{texdocclassintro}
\begin{texdocclassconstructors}
\texdocconstructor{public}{AdviserPageController}{()}{}{}
\end{texdocclassconstructors}
\begin{texdocclassmethods}
\texdocmethod{public}{Result}{accountPage}{()}{Diese Methode gibt die Seite zurück, auf der der Betreuer seine
 Benutzerdaten wie E-Mail-Adresse und Passwort ändern kann.}{\texdocreturn{die Seite, die als Antwort verschickt wird.}
}
\texdocmethod{public}{Result}{addProject}{()}{Diese Methode fügt ein neues Projekt in das System ein und leitet den
 Betreuer zurück auf die Seite zum Editieren des Projektes.}{\texdocreturn{die Seite, die als Antwort verschickt wird.}
}
\texdocmethod{public}{Result}{editAccount}{()}{Diese Methode editiert die Daten des Betreuers, welche er auf der
 Account-Seite geändert hat.}{\texdocreturn{die Seite, die als Antwort verschickt wird.}
}
\texdocmethod{public}{Result}{editProject}{()}{Diese Methode editiert ein bereits vorhandenes Projekt. Die zu
 editierenden Daten übermittelt der Betreuer über ein Formular, welches er
 zum Editieren abschickt. Nur Betreuer welche dem Projekt beigetreten sind
 können dieses editieren. Anschließend wird der Betreuer auf die Seite zum
 Hinzufügen und Editieren von Projekten weitergeleitet.}{\texdocreturn{die Seite, die als Antwort verschickt wird.}
}
\texdocmethod{public}{Result}{joinProject}{()}{Diese Methode fügt einen Betreuer zu einem bereits existierenden Projekt
 hinzu, sodass dieser auch die Möglichkeit besitzt das Projekt zu
 editieren oder zu löschen und nach der Veröffentlichung einer Einteilung
 auch Teams und deren Mitglieder sieht. Nach dem Beitreten wird der
 Betreuer auf die Seite zum Hinzufügen und Editieren von Projekten
 weitergeleitet.}{\texdocreturn{die Seite, die als Antwort verschickt wird.}
}
\texdocmethod{public}{Result}{leaveProject}{()}{Diese Methode entfernt einen Betreuer aus einem existierenden Projekt,
 sodass dieser nicht mehr das Projekt editieren oder löschen kann.
 Anschließend wird der Betreuer auf die Seite zum Hinzufügen und Editieren
 von Projekten weitergeleitet.}{\texdocreturn{die Seite, die als Antwort verschickt wird.}
}
\texdocmethod{public}{Result}{projectsPage}{(String name)}{Diese Methode gibt die Seite zurück, auf der der Betreuer Projekte sieht,
 Projekte hinzufügen, editieren oder entfernen kann. Editieren und
 Entfernen eines Projektes ist beschränkt auf Betreuer, welche dem Projekt
 beigetreten sind.}{\begin{texdocparameters}
\texdocparameter{name}{Der Name des Projektes (da mitgegeben über die URL, ist der String encoded)}
\end{texdocparameters}
\texdocreturn{die Seite, die als Antwort verschickt wird.}
}
\texdocmethod{public}{Result}{removeProject}{()}{Diese Methode löscht ein Projekt und alle dazugehörenden Daten aus dem
 System und leitet den Betreuer auf die Seite zum Editieren und Hinzufügen
 von Projekten. Nur Betreuer welche dem Projekt beigetreten sind können
 dieses editieren.}{\texdocreturn{die Seite, die als Antwort verschickt wird.}
}
\texdocmethod{public}{Result}{saveStudentsGrades}{()}{Diese Methode speichert alle von einem Betreuer in ein Formular
 eingegebenen Noten, sodass diese vom Administrator in das CMS importiert
 werden können. Anschließend wird der Betreuer auf die Projektseite des
 jeweiligen Projektes weitergeleitet.}{\texdocreturn{die Seite, die als Antwort verschickt wird.}
}
\end{texdocclassmethods}
\end{texdocclass}


\begin{texdocclass}{class}{EditAllocationCommand}
\label{texdoclet:controller.EditAllocationCommand}
\begin{texdocclassintro}
Die abstrakte Oberklasse aller Kommandos, die in eine Veränderung einer Einteilung durch den Administrator dartstellen.\end{texdocclassintro}
\begin{texdocclassconstructors}
\texdocconstructor{public}{EditAllocationCommand}{()}{}{}
\end{texdocclassconstructors}
\begin{texdocclassmethods}
\texdocmethod{public abstract}{void}{execute}{()}{Führt das Kommando aus.}{}
\texdocmethod{public}{void}{undo}{()}{Macht die Ausführung des Kommandos rückgängig. Kann nur ausgeführt werde, wenn das Kommando zuvor ausgeführt wurde.}{}
\end{texdocclassmethods}
\end{texdocclass}


\begin{texdocclass}{class}{GeneralAdminController}
\label{texdoclet:controller.GeneralAdminController}
\begin{texdocclassintro}
Dieser Controller ist für das Bearbeiten der Http-Requests zuständig, welche
 abgeschickt werden, wenn Betreuer, Studenten oder Einteilungen hinzugefügt
 oder gelöscht werden sollen.\end{texdocclassintro}
\begin{texdocclassconstructors}
\texdocconstructor{public}{GeneralAdminController}{()}{}{}
\end{texdocclassconstructors}
\begin{texdocclassmethods}
\texdocmethod{public}{Result}{addAdviser}{()}{Diese Methode fügt einen Betreuer mit den Daten aus dem vom Administrator
 auszufüllenden Formular zum System hinzu. Der Administrator wird
 anschließend auf die Betreuerübersicht weitergeleitet.}{\texdocreturn{Die Seite, die als Antwort verschickt wird.}
}
\texdocmethod{public}{Result}{addAllocation}{()}{Diese Methode fügt eine Einteilungskonfiguration in die
 Berechnungswarteschlange hinzu. Der Administrator wird anschließend auf
 die Berechnungsübersichtsseite weitergeleitet.}{\texdocreturn{Die Seite, die als Antwort verschickt wird.}
}
\texdocmethod{public}{Result}{addStudent}{()}{Diese Methode fügt einen Studenten in das System hinzu. Der Administrator
 wird anschließend auf die Seite zum weiteren Hinzufügen und Löschen von
 Studenten weitergeleitet.}{\texdocreturn{Die Seite, die als Antwort verschickt wird.}
}
\texdocmethod{public}{Result}{removeAdviser}{()}{Diese Methode entfernt einen Betreuer und dessen Daten aus dem System.
 Der Administrator wird anschließend auf die Betreuerübersicht
 weitergeleitet.}{\texdocreturn{Die Seite, die als Antwort verschickt wird.}
}
\texdocmethod{public}{Result}{removeStudent}{()}{Diese Methode löscht einen Studenten aus dem System. Der Administrator
 wird anschließend auf die Seite zum weiteren Hinzufügen und Löschen von
 Studenten weitergeleitet.}{\texdocreturn{Die Seite, die als Antwort verschickt wird.}
}
\end{texdocclassmethods}
\end{texdocclass}


\begin{texdocclass}{class}{IndexPageController}
\label{texdoclet:controller.IndexPageController}
\begin{texdocclassintro}
Dieser Controller ist zuständig für alle Http-Requests, die in dem Bereich
 aufkommen, welche ohne Anmeldung zugänglich sind. Dazu zählt neben der
 Index-Seite auch die Passwort vergessen-Seite und die
 E-Mail-Verifikations-Seite.\end{texdocclassintro}
\begin{texdocclassconstructors}
\texdocconstructor{public}{IndexPageController}{()}{}{}
\end{texdocclassconstructors}
\begin{texdocclassmethods}
\texdocmethod{public}{Result}{indexPage}{()}{Diese Methode gibt die Startseite zurück. Auf dieser Seite können sich
 Administrator, Betreuer und Studenten anmelden oder aktuelle Informationen
 einsehen.}{\texdocreturn{Die Seite, die als Antwort verschickt wird.}
}
\texdocmethod{public}{Result}{login}{()}{Diese Methode initiiert die Login-Prozedur und leitet den Anzumeldenden
 je nach Autorisierung auf die passende Seite weiter.}{\texdocreturn{Die Seite, die als Antwort verschickt wird.}
}
\texdocmethod{public}{Result}{passwordReset}{()}{Diese Methode schickt eine E-Mail anhand der Daten aus dem
 Passwort-Rücksetz-Formular an den Studenten oder den Betreuer, welche
 ein neues Passwort enthält.}{\texdocreturn{Die Seite, die als Antwort verschickt wird.}
}
\texdocmethod{public}{Result}{passwordResetPage}{()}{Diese Methode gibt die Seite zurück, die ein Passwort-Rücksetz-Formular
 für Studenten und Betreuer anzeigt.}{\texdocreturn{Die Seite, die als Antwort verschickt wird.}
}
\texdocmethod{public}{Result}{register}{()}{Diese Methode registriert einen Studenten und fügt diesen in die
 Datenbank ein, sofern alle notwendigen Teillestungen als bestanden
 angegeben wurden.}{\texdocreturn{Die Seite, die als Antwort verschickt wird.}
}
\texdocmethod{public}{Result}{registerPage}{()}{Diese Methode gibt die Seite zurück, auf der sich ein Student
 registrieren kann.}{\texdocreturn{Die Seite, die als Antwort verschickt wird.}
}
\texdocmethod{public}{Result}{verificationPage}{()}{Diese Methode gibt die Seite zurück, welche einen Studenten verifiziert.
 Dies funktioniert, indem der Student eine Mail mit einen Link auf diese
 Seite erhält, welche noch einen Code als Parameter übergibt. Anhand
 dieses Parameters wird der Student verifiziert.}{\texdocreturn{Die Seite, die als Antwort verschickt wird.}
}
\end{texdocclassmethods}
\end{texdocclass}


\begin{texdocclass}{class}{MoveStudentCommand}
\label{texdoclet:controller.MoveStudentCommand}
\begin{texdocclassintro}
Konkretes Kommando zum verschieben eines Studierenden von seinem aktuellen Team in ein neues.\end{texdocclassintro}
\begin{texdocclassconstructors}
\texdocconstructor{public}{MoveStudentCommand}{(Allocation allocation, Student student, Team newTeam)}{Erzeugt ein neues Kommando zum verschieben eines Studierenden.}{\begin{texdocparameters}
\texdocparameter{allocation}{Einteilung, auf die sich die Änderung bezieht.}
\texdocparameter{student}{Studierender, der verschoben werden soll.}
\texdocparameter{newTeam}{Neues Team, in das der Studierende eingeteilt wird.}
\end{texdocparameters}
}
\end{texdocclassconstructors}
\begin{texdocclassmethods}
\texdocmethod{public}{void}{execute}{()}{\texdocinheritdoc{controller.EditAllocationCommand}{Führt das Kommando aus.}}{}
\texdocmethod{public}{void}{undo}{()}{\texdocinheritdoc{controller.EditAllocationCommand}{Macht die Ausführung des Kommandos rückgängig. Kann nur ausgeführt werde, wenn das Kommando zuvor ausgeführt wurde.}}{}
\end{texdocclassmethods}
\end{texdocclass}


\begin{texdocclass}{class}{StudentPageController}
\label{texdoclet:controller.StudentPageController}
\begin{texdocclassintro}
Dieser Controller ist zuständig für alle Http-Requests, welche im
 Studentenbereich aufkommen. Dazu zählen das Senden einer neuen HTML-Seite bei
 einem Klick auf einen Link, als auch das Reagieren auf Benutzereingaben, wie
 das Abschicken eines Formulars.\end{texdocclassintro}
\begin{texdocclassconstructors}
\texdocconstructor{public}{StudentPageController}{()}{}{}
\end{texdocclassconstructors}
\begin{texdocclassmethods}
\texdocmethod{public}{Result}{accountPage}{()}{Diese Methode gibt die Seite zurück, auf der der Student seine
 Studentendaten wie E-Mail-Adresse und Passwort ändern kann.}{\texdocreturn{Die Seite, die als Antwort verschickt wird.}
}
\texdocmethod{public}{Result}{createLearningGroup}{()}{Diese Methode erstellt eine neue Lerngruppe im System und fügt den
 Ersteller der Lerngruppe als erstes Mitglied in diese ein. Der Student
 wird anschließend auf die Lerngruppen-Seite zurückgeleitet.}{\texdocreturn{Die Seite, die als Antwort verschickt wird.}
}
\texdocmethod{public}{Result}{editAccount}{()}{Diese Methode editiert die Daten des Studenten, welche er auf der
 Account-Seite geändert hat.}{\texdocreturn{Die Seite, die als Antwort verschickt wird.}
}
\texdocmethod{public}{Result}{joinLearningGroup}{()}{Diese Methode fügt den Studenten zu einer Lerngruppe hinzu, falls eine
 Lerngruppe mit dem Namen und dem zugehörigen Passwort existiert.
 Anschließend wird der Student auf die Lerngruppen-Seite zurückgeleitet.}{\texdocreturn{Die Seite, die als Antwort verschickt wird.}
}
\texdocmethod{public}{Result}{learningGroupPage}{()}{Diese Methode gibt die Seite zurück, auf der der Student sieht in welcher
 Lerngruppe er ist, oder wenn er in keiner aktuell ist, eine erstellen
 oder einer beitreten kann.}{\texdocreturn{Die Seite, die als Antwort verschickt wird.}
}
\texdocmethod{public}{Result}{leaveLearningGroup}{()}{Diese Methode entfernt den Student aus der aktuellen Lerngruppe.
 Anschließend wird der Student auf die Lerngruppen-Seite zurück geleitet.}{\texdocreturn{Die Seite, die als Antwort verschickt wird.}
}
\texdocmethod{public}{Result}{rate}{()}{Diese Methode fügt die Daten der Bewertungen eines Studenten in das
 System ein und leitet den Studenten wieder zurück auf die
 Bewertungsseite, wo er nun seine eingegebene Bewertungen sehen kann.}{\texdocreturn{Die Seite, die als Antwort verschickt wird.}
}
\texdocmethod{public}{Result}{ratingPage}{()}{Diese Methode gibt die Seite zurück, auf der der Student seine
 Bewertungen abgeben kann.}{\texdocreturn{Die Seite, die als Antwort verschickt wird.}
}
\texdocmethod{public}{Result}{resultsPage}{()}{Diese Methode gibt die Seite zurück, auf der der Student das Ergebnis der
 Einteilungsberechnung einsehen kann. Er sieht also sein Projekt und seine
 Teammitglieder.}{\texdocreturn{Die Seite, die als Antwort verschickt wird.}
}
\end{texdocclassmethods}
\end{texdocclass}


\begin{texdocclass}{class}{SwapStudentCommand}
\label{texdoclet:controller.SwapStudentCommand}
\begin{texdocclassintro}
Konkretes Kommando zum vertauschen der Teamzugehörigkeit von zwei Studierenden.\end{texdocclassintro}
\begin{texdocclassconstructors}
\texdocconstructor{public}{SwapStudentCommand}{(Allocation allocation, Student firstStudent, Student secondStudent)}{Erzeugt ein neues Kommando um die Teams von zwei Studenten zu vertauschen.}{\begin{texdocparameters}
\texdocparameter{allocation}{Einteilung, auf die sich die Änderung bezieht.}
\texdocparameter{firstStudent}{Erster Studierender, der verschoben werden soll.}
\texdocparameter{secondStudent}{Zweiter Studierender, der verschoben werden soll.}
\end{texdocparameters}
}
\end{texdocclassconstructors}
\begin{texdocclassmethods}
\texdocmethod{public}{void}{execute}{()}{\texdocinheritdoc{controller.EditAllocationCommand}{Führt das Kommando aus.}}{}
\texdocmethod{public}{void}{undo}{()}{\texdocinheritdoc{controller.EditAllocationCommand}{Macht die Ausführung des Kommandos rückgängig. Kann nur ausgeführt werde, wenn das Kommando zuvor ausgeführt wurde.}}{}
\end{texdocclassmethods}
\end{texdocclass}


\end{texdocpackage}



