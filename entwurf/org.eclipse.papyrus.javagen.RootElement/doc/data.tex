\begin{texdocpackage}{data}
\label{texdoclet:data}

\begin{texdocclass}{class}{Achievement}
\label{texdoclet:data.Achievement}
\begin{texdocclassintro}
Diese Klasse stellt eine Teilleistung im Studium dar.\end{texdocclassintro}
\begin{texdocclassconstructors}
\texdocconstructor{public}{Achievement}{()}{}{}
\end{texdocclassconstructors}
\begin{texdocclassmethods}
\texdocmethod{public static}{Achievement}{getAchievement}{(String name)}{Diese Methode gibt eine bestimmte Teilleistung zurück, die durch ihren
 Namen identifiziert wird.}{\begin{texdocparameters}
\texdocparameter{name}{Der Name der Teilleistung.}
\end{texdocparameters}
\texdocreturn{Die bestimmte Teilleistung.}
}
\texdocmethod{public static}{Achievement}{getAchievements}{()}{Diese Methode gibt alle Teilleistungen zurück.}{\texdocreturn{Alle existierenden Teilleistungen.}
}
\texdocmethod{public}{String}{getName}{()}{Getter für den Namen der Teilleistung.}{\texdocreturn{Der Name der Teilleistung.}
}
\texdocmethod{public}{void}{setName}{(String name)}{Setter für den Namen der Teilleistung.}{\begin{texdocparameters}
\texdocparameter{name}{Der Name der Teilleistung.}
\end{texdocparameters}
}
\end{texdocclassmethods}
\end{texdocclass}


\begin{texdocclass}{class}{Adviser}
\label{texdoclet:data.Adviser}
\begin{texdocclassintro}
Diese Klasse stellt einen Betreuer dar.\end{texdocclassintro}
\begin{texdocclassconstructors}
\texdocconstructor{public}{Adviser}{()}{}{}
\end{texdocclassconstructors}
\begin{texdocclassmethods}
\texdocmethod{public static}{List}{getAdvisers}{()}{Diese Methode gibt alle Betreuer zurück, die es gibt.}{\texdocreturn{Liste aller Betreuer.}
}
\texdocmethod{public}{List\textless{}Project\textgreater{}}{getProjects}{()}{Getter für die Projekte, die der Betreuer beaufsichtigt.}{\texdocreturn{Die Projekte, die der Betreuer beaufsichtigt.}
}
\end{texdocclassmethods}
\end{texdocclass}


\begin{texdocclass}{class}{Allocation}
\label{texdoclet:data.Allocation}
\begin{texdocclassintro}
Diese Klasse stellt eine Einteilung von Studierenden in einem Semester dar.\end{texdocclassintro}
\begin{texdocclassconstructors}
\texdocconstructor{public}{Allocation}{()}{}{}
\end{texdocclassconstructors}
\begin{texdocclassmethods}
\texdocmethod{public static}{Allocation}{getAllocation}{(String name)}{Diese Methode gibt eine spezifische Einteilung zurück, die über ihren
 Namen identifiziert wird.}{\begin{texdocparameters}
\texdocparameter{name}{Der Name der Einteilung.}
\end{texdocparameters}
\texdocreturn{Die Einteilung mit dem gegebenen Namen.}
}
\texdocmethod{public static}{List}{getAllocations}{()}{Diese Methode gibt alle Einteilungen zurück.}{\texdocreturn{Alle Einteilungen.}
}
\texdocmethod{public}{String}{getName}{()}{Getter für den Namen der Einteilung.}{\texdocreturn{Name der Einteilung.}
}
\texdocmethod{public}{List}{getParameters}{()}{Getter für die Parameter der Einteilung.}{\texdocreturn{Parameter der Einteilung.}
}
\texdocmethod{public}{Team}{getTeam}{(Student student)}{Gibt das Team zurück, in das ein bestimmter Student eingeteilt wurde.
 Gibt null zurück, wenn der Student nicht zugeteilt wurde.}{\begin{texdocparameters}
\texdocparameter{student}{Student, zu welchem das zugeteilte Team zurückgegeben wird.}
\end{texdocparameters}
\texdocreturn{Team, das dem Studenten zugeteilt wurde.}
}
\texdocmethod{public}{List\textless{}Team\textgreater{}}{getTeams}{()}{Getter für die Liste der Teams.}{\texdocreturn{Liste der Teams.}
}
\texdocmethod{public}{void}{setName}{(String name)}{Setter für den Namen der Einteilung.}{\begin{texdocparameters}
\texdocparameter{name}{Name der Einteilung.}
\end{texdocparameters}
}
\texdocmethod{public}{void}{setSemester}{(List parameters)}{Setter für die Parameter der Einteilung.}{\begin{texdocparameters}
\texdocparameter{parameters}{Parameter der Einteilung.}
\end{texdocparameters}
}
\texdocmethod{public}{void}{setStudentsTeam}{(Student student, Team team)}{Ändert für einen Studenten das eingeteilte Team.
 Wenn das Team null ist, wird der Student keinem Team zugeteilt.}{\begin{texdocparameters}
\texdocparameter{student}{Student, dessen Team geändert wird.}
\texdocparameter{team}{neues Team, in das der Student eingeteilt ist.}
\end{texdocparameters}
}
\texdocmethod{public}{void}{setTeams}{(List\textless{}Team\textgreater{} teams)}{Setter für die Liste der Teams.}{\begin{texdocparameters}
\texdocparameter{teams}{Liste der Teams.}
\end{texdocparameters}
}
\end{texdocclassmethods}
\end{texdocclass}


\begin{texdocclass}{class}{AllocationParameter}
\label{texdoclet:data.AllocationParameter}
\begin{texdocclassintro}
Diese Klasse stellt einen Parameter für die Einteilungsberechnung dar.\end{texdocclassintro}
\begin{texdocclassconstructors}
\texdocconstructor{public}{AllocationParameter}{()}{}{}
\end{texdocclassconstructors}
\begin{texdocclassmethods}
\texdocmethod{public}{String}{getName}{()}{Getter für den Namen des Parameters.}{\texdocreturn{Der Name des Parameters.}
}
\texdocmethod{public}{double}{getValue}{()}{Getter für den Wert des Parameters.}{\texdocreturn{Der Wert des Parameters.}
}
\texdocmethod{public}{void}{setName}{(String name)}{Setter für den Namen des Parameters.}{\begin{texdocparameters}
\texdocparameter{name}{Der Name des Parameters.}
\end{texdocparameters}
}
\texdocmethod{public}{void}{setValue}{(double value)}{Setter für den Wert des Parameters.}{\begin{texdocparameters}
\texdocparameter{value}{Der Wert des Parameters.}
\end{texdocparameters}
}
\end{texdocclassmethods}
\end{texdocclass}


\begin{texdocclass}{class}{GeneralData}
\label{texdoclet:data.GeneralData}
\begin{texdocclassintro}
Diese Klasse beinhaltet generelle Daten, über den Zustand der Software.\end{texdocclassintro}
\begin{texdocclassconstructors}
\texdocconstructor{public}{GeneralData}{()}{}{}
\end{texdocclassconstructors}
\begin{texdocclassmethods}
\texdocmethod{public static}{String}{getAdminName}{()}{Getter für den Anmeldenamen des Administrators.}{\texdocreturn{Der Anmeldename des Administrators.}
}
\texdocmethod{public static}{String}{getAdminPassword}{()}{Getter für das Anmeldepasswort des Administrators.}{\texdocreturn{Das Anmeldepasswort des Administrators.}
}
\texdocmethod{public}{Semester}{getCurrentSemester}{()}{Getter für das aktuelle Semester.}{\texdocreturn{Das aktuelle Semester.}
}
\texdocmethod{public}{void}{setCurrentSemester}{(Semester currentSemester)}{Setter für das aktuelle Semester.}{\begin{texdocparameters}
\texdocparameter{currentSemester}{Das aktuelle Semester.}
\end{texdocparameters}
}
\end{texdocclassmethods}
\end{texdocclass}


\begin{texdocclass}{class}{LearningGroup}
\label{texdoclet:data.LearningGroup}
\begin{texdocclassintro}
Diese Klasse repräsentiert eine Lerngruppe, das heißt eine Gruppe von
 Studierenden, die sich gemeinsam zum PSE anmelden wollen.\end{texdocclassintro}
\begin{texdocclassconstructors}
\texdocconstructor{public}{LearningGroup}{()}{}{}
\end{texdocclassconstructors}
\begin{texdocclassmethods}
\texdocmethod{public}{void}{addMember}{(Student student)}{Fügt einen Studenten zu der Lerngruppe hinzu.}{\begin{texdocparameters}
\texdocparameter{student}{Student, der hinzugefügt wird.}
\end{texdocparameters}
}
\texdocmethod{public static}{LearningGroup}{getLearningGroup}{(String name, Semester semester)}{Diese Methode gibt eine spezifische Lerngruppe zurück.}{\begin{texdocparameters}
\texdocparameter{name}{Der Name der Lerngruppe.}
\texdocparameter{semester}{Das Semster, in dem die Lerngruppe erstellt wurde.}
\end{texdocparameters}
\texdocreturn{Die spezifische Lerngruppe.}
}
\texdocmethod{public static}{List\textless{}LearningGroup\textgreater{}}{getLearningGroups}{()}{Diese Methode gibt alle Lerngruppen zurück.}{\texdocreturn{Alle Lerngruppen.}
}
\texdocmethod{public}{List\textless{}Student\textgreater{}}{getMembers}{()}{Getter für die Mitglieder der Lerngruppe.}{\texdocreturn{Die Mitglieder der Lerngruppe.}
}
\texdocmethod{public}{String}{getName}{()}{Getter für den Namen.}{\texdocreturn{Name der Lerngruppe.}
}
\texdocmethod{public}{String}{getPassword}{()}{Getter für das Passwort.}{\texdocreturn{Das Passwort, um der Lerngruppe beizutreten.}
}
\texdocmethod{public}{int}{getRating}{(Project project)}{Gibt die Bewertung für ein Projekt zurück.}{\begin{texdocparameters}
\texdocparameter{project}{Projekt, für welches die Bewertung zurückgegeben wird.}
\end{texdocparameters}
\texdocreturn{Bewertung des Projekts.}
}
\texdocmethod{public}{List\textless{}Rating\textgreater{}}{getRatings}{()}{Getter für die Projektbewertungen.}{\texdocreturn{Projektbewertungen der Lerngruppe.}
}
\texdocmethod{public}{boolean}{isPrivate}{()}{Getter, ob Lerngruppe privat ist.}{\texdocreturn{Wahr, wenn privat, sonst falsch.}
}
\texdocmethod{public}{void}{rate}{(Project project, int rating)}{Ändert die Bewertung für ein Projekt.}{\begin{texdocparameters}
\texdocparameter{project}{Projekt, für das die Bewertung geändert wird.}
\texdocparameter{rating}{Bewertung des Projekts.}
\end{texdocparameters}
}
\texdocmethod{public}{void}{removeMember}{(Student student)}{Entfernt einen Studenten von der Lengruppe.}{\begin{texdocparameters}
\texdocparameter{student}{Student, der entfernt wird.}
\end{texdocparameters}
}
\texdocmethod{public}{void}{setMembers}{(List\textless{}Student\textgreater{} members)}{Setter für die Mitglieder der Lerngruppe.}{\begin{texdocparameters}
\texdocparameter{members}{Die Mitglieder der Lerngruppe.}
\end{texdocparameters}
}
\texdocmethod{public}{void}{setName}{(String name)}{Setter für den Namen.}{\begin{texdocparameters}
\texdocparameter{name}{Name der Lerngruppe.}
\end{texdocparameters}
}
\texdocmethod{public}{void}{setPassword}{(String password)}{Setter für das Passwort.}{\begin{texdocparameters}
\texdocparameter{password}{Das Passwort, um der Lerngruppe beizutreten.}
\end{texdocparameters}
}
\texdocmethod{public}{boolean}{setPrivate}{(boolean isPrivate)}{Setter, ob Lerngruppe privat ist.}{\begin{texdocparameters}
\texdocparameter{isPrivate}{Wahr, wenn privat, sonst falsch.}
\end{texdocparameters}
}
\texdocmethod{public}{void}{setRatings}{(List\textless{}Rating\textgreater{} ratings)}{Setter für die Projektbewertungen.}{\begin{texdocparameters}
\texdocparameter{ratings}{Projektbewertungen der Lerngruppe.}
\end{texdocparameters}
}
\end{texdocclassmethods}
\end{texdocclass}


\begin{texdocclass}{class}{Project}
\label{texdoclet:data.Project}
\begin{texdocclassintro}
Klasse, die ein Project repräsentiert\end{texdocclassintro}
\begin{texdocclassconstructors}
\texdocconstructor{public}{Project}{()}{}{}
\end{texdocclassconstructors}
\begin{texdocclassmethods}
\texdocmethod{public}{void}{addAdviser}{(Adviser adviser)}{Fügt dem Projekt einen Betreuer hinzu.}{\begin{texdocparameters}
\texdocparameter{adviser}{Betreuer der hinzugefügt wird.}
\end{texdocparameters}
}
\texdocmethod{public}{List}{getAdvisers}{()}{Getter für die Betreuer des Projekts.}{\texdocreturn{Betreuer des Projekts.}
}
\texdocmethod{public}{String}{getInstitute}{()}{Gibt den Institutsnamen des Institutes zurück, welches das Projekt
 anbietet.}{\texdocreturn{den Namen}
}
\texdocmethod{public}{int}{getMaxTeamSize}{()}{Getter der maximalen Größe für Teams dieses Projektes.}{\texdocreturn{Die maximale Teamgröße.}
}
\texdocmethod{public}{int}{getMinTeamSize}{()}{Getter der minimalen Größe für Teams dieses Projektes.}{\texdocreturn{Die minimale Teamgröße.}
}
\texdocmethod{public}{String}{getName}{()}{Getter für den Namen des Projektes.}{\texdocreturn{Der Name des Projektes.}
}
\texdocmethod{public}{int}{getNumberOfTeams}{()}{Getter für die Anzahl der Teams.}{\texdocreturn{Anzahl der Teams.}
}
\texdocmethod{public static}{Project}{getProject}{(String name, Semester semester)}{Diese Methode gibt ein spezifisches Projekt zurück, welches über seinen
 Namen und das Semester, in dem es erstellt wurde, identifiziert wird.}{\begin{texdocparameters}
\texdocparameter{name}{Der Name des Projektes.}
\texdocparameter{semester}{Das Semester, in dem das Projekt erstellt wurde.}
\end{texdocparameters}
\texdocreturn{Das spezifische Projekt.}
}
\texdocmethod{public}{String}{getProjectInfo}{()}{Getter für die Information über dieses Projektes.}{\texdocreturn{Die Information des Projektes.}
}
\texdocmethod{public static}{List\textless{}Project\textgreater{}}{getProjects}{()}{Diese Methode gibt alle Projekte zurück.}{\texdocreturn{Alle Projekte.}
}
\texdocmethod{public}{String}{getProjectURL}{()}{Getter für die URL des Projektes.}{\texdocreturn{Die URL des Projektes.}
}
\texdocmethod{public}{int}{getRating}{(Student student)}{Diese Methode gibt die Bewertung eines spezifischen Studenten für dieses
 Projekt zurück.}{\begin{texdocparameters}
\texdocparameter{student}{Der Student, dessen Bewertung zurückgegeben werdedn soll.}
\end{texdocparameters}
\texdocreturn{Die Bewertung des Studenten.}
}
\texdocmethod{public}{void}{removeAdviser}{(Adviser adviser)}{Entfernt einen Betreuer vom Projekt.}{\begin{texdocparameters}
\texdocparameter{adviser}{Betreuer der entfernt wird.}
\end{texdocparameters}
}
\texdocmethod{public}{void}{setAdvisers}{(List advisers)}{Setter für die Betreuer des Projekts.}{\begin{texdocparameters}
\texdocparameter{advisers}{Betreuer des Projekts.}
\end{texdocparameters}
}
\texdocmethod{public}{void}{setInstitute}{(String institute)}{Setzt den Institutsnamen.}{\begin{texdocparameters}
\texdocparameter{institute}{der Name des Instituts.}
\end{texdocparameters}
}
\texdocmethod{public}{void}{setMaxTeamSize}{(int maxTeamSize)}{Setter der maximalen Größe für Teams dieses Projektes.}{\begin{texdocparameters}
\texdocparameter{maxTeamSize}{Die maximale Größe für Teams dieses Projektes.}
\end{texdocparameters}
}
\texdocmethod{public}{void}{setMinTeamSize}{(int minTeamSize)}{Setter der minimalen Größe für Teams dieses Projektes.}{\begin{texdocparameters}
\texdocparameter{minTeamSize}{Die minimale Größe für Teams dieses Projektes.}
\end{texdocparameters}
}
\texdocmethod{public}{void}{setName}{(String name)}{Setter für den Namen des Projektes.}{\begin{texdocparameters}
\texdocparameter{name}{Der Name des Projektes.}
\end{texdocparameters}
}
\texdocmethod{public}{void}{setNumberOfTeams}{(int numberOfTeams)}{Setter für die Anzahl der Teams.}{\begin{texdocparameters}
\texdocparameter{numberOfTeams}{Anzahl der Teams.}
\end{texdocparameters}
}
\texdocmethod{public}{void}{setProjectInfo}{(String projectInfo)}{Setter für die Information über dieses Projektes.}{\begin{texdocparameters}
\texdocparameter{projektInfo}{Die Information des Projektes.}
\end{texdocparameters}
}
\texdocmethod{public}{void}{setProjectURL}{(String projectURL)}{Setter für die URL des Projektes.}{\begin{texdocparameters}
\texdocparameter{projectURL}{Die URL des Projektes.}
\end{texdocparameters}
}
\end{texdocclassmethods}
\end{texdocclass}


\begin{texdocclass}{class}{PSEServerConfigStartup}
\label{texdoclet:data.PSEServerConfigStartup}
\begin{texdocclassintro}
Klasse, die die Datenbank initialisiert.\end{texdocclassintro}
\begin{texdocclassconstructors}
\texdocconstructor{public}{PSEServerConfigStartup}{()}{}{}
\end{texdocclassconstructors}
\begin{texdocclassmethods}
\texdocmethod{public}{void}{onStartup}{(ServerConfig serverConfig)}{Initialisiert die Datenbank.}{\begin{texdocparameters}
\texdocparameter{serverConfig}{Ein Objekt, welches zur Konfiguration des Servers verwendet wird.}
\end{texdocparameters}
}
\end{texdocclassmethods}
\end{texdocclass}


\begin{texdocclass}{class}{Rating}
\label{texdoclet:data.Rating}
\begin{texdocclassintro}
Diese Klasse stellt eine Bewertung eines Studierenden oder einerr Lerngruppe für ein Projekt dar.\end{texdocclassintro}
\begin{texdocclassconstructors}
\texdocconstructor{public}{Rating}{()}{}{}
\end{texdocclassconstructors}
\begin{texdocclassmethods}
\texdocmethod{public}{Project}{getProject}{()}{Getter für das Projekt der Bewertung.}{\texdocreturn{Das Projekt, das bewertet wird.}
}
\texdocmethod{public}{int}{getRating}{()}{Getter für den Wert der Bewertung.}{\texdocreturn{Der Wert der Bewertung.}
}
\texdocmethod{public}{void}{setProject}{(Project project)}{Setter für das Projekt der Bewertung.}{\begin{texdocparameters}
\texdocparameter{project}{Das Projekt, das bewertet wird.}
\end{texdocparameters}
}
\texdocmethod{public}{void}{setRating}{(int rating)}{Setter für den Wert der Bewertung.}{\begin{texdocparameters}
\texdocparameter{rating}{Der Wert der Bewertung.}
\end{texdocparameters}
}
\end{texdocclassmethods}
\end{texdocclass}


\begin{texdocclass}{class}{Semester}
\label{texdoclet:data.Semester}
\begin{texdocclassintro}
Diese Klasse repräsentiert ein Semseter.\end{texdocclassintro}
\begin{texdocclassconstructors}
\texdocconstructor{public}{Semester}{()}{}{}
\end{texdocclassconstructors}
\begin{texdocclassmethods}
\texdocmethod{public}{void}{addAllocation}{(Allocation allocation)}{Fügt dem Semester eine Einteilung hinzu.}{\begin{texdocparameters}
\texdocparameter{allocation}{Einteilung, die hinzugefügt wird.}
\end{texdocparameters}
}
\texdocmethod{public}{void}{addLearningGroup}{(LearningGroup learningGroup)}{Fügt eine Lerngruppe hinzu.}{\begin{texdocparameters}
\texdocparameter{learningGroup}{Lerngruppe, die hinzugefügt wird.}
\end{texdocparameters}
}
\texdocmethod{public}{void}{addProject}{(Project project)}{Fügt ein Projekt hinzu.}{\begin{texdocparameters}
\texdocparameter{project}{Projekt, das hinzugefügt wird.}
\end{texdocparameters}
}
\texdocmethod{public}{void}{addSPO}{(SPO spo)}{Fügt eine SPO hinzu.}{\begin{texdocparameters}
\texdocparameter{spo}{SPO, die hinzugefügt wird.}
\end{texdocparameters}
}
\texdocmethod{public}{void}{addStudent}{(Student student)}{Fügt einen Studenten hinzu.}{\begin{texdocparameters}
\texdocparameter{student}{Student, der hinzugefügt wird.}
\end{texdocparameters}
}
\texdocmethod{public}{List}{getAdvisers}{()}{Diese Methode gibt alle Betreuer dieses Semesters zurück.}{\texdocreturn{Alle Betreuer des Semesters.}
}
\texdocmethod{public}{List}{getAllocations}{()}{Getter für die Einteilungen.}{\texdocreturn{Alle in diesem Semester berechneten Einteilungen.}
}
\texdocmethod{public}{Allocation}{getFinalAllocation}{()}{Getter für die finale Einteilung.}{\texdocreturn{Die finale Einteilung.}
}
\texdocmethod{public}{String}{getInfoText}{()}{Getter für den Infotext.}{\texdocreturn{Der Infotext des Semesters.}
}
\texdocmethod{public}{List\textless{}LearningGroup\textgreater{}}{getLearningGroups}{()}{Getter für die Lerngruppen.}{\texdocreturn{Alle Lerngruppen dieses Semesters.}
}
\texdocmethod{public}{String}{getName}{()}{Getter für den Namen des Semesters.}{\texdocreturn{Der Name des Semesters.}
}
\texdocmethod{public}{List\textless{}Project\textgreater{}}{getProjects}{()}{Getter für die Projekte.}{\texdocreturn{Die Projekte, die in diesem Semester existieren.}
}
\texdocmethod{public}{Date}{getRegistrationEnd}{()}{Getter für den Endzeitpunkt der Registrierung.}{}
\texdocmethod{public}{Date}{getRegistrtaionStart}{()}{Getter für den Startpunkt der Registrierung.}{}
\texdocmethod{public static}{Semester}{getSemester}{(String semesterName)}{Diese Methode gibt ein spezifisches Semester zurück.}{\begin{texdocparameters}
\texdocparameter{semesterName}{Der Name des Semseters.}
\end{texdocparameters}
\texdocreturn{Das gesuchte Semester.}
}
\texdocmethod{public static}{List\textless{}Semester\textgreater{}}{getSemesters}{()}{Diese Methode gibt alles Semseter zurück, die erstellt wurden.}{\texdocreturn{Alle Semseter.}
}
\texdocmethod{public}{List\textless{}SPO\textgreater{}}{getSpos}{()}{Getter für die SPOs des Semesters.}{\texdocreturn{Die verfügbaren SPOs des Semesters.}
}
\texdocmethod{public}{List\textless{}Student\textgreater{}}{getStudents}{()}{Getter für die Studenten.}{\texdocreturn{Alle Studenten, die in diesem Semester angemeldet sind.}
}
\texdocmethod{public}{List\textless{}Team\textgreater{}}{getTeams}{()}{Diese Methode gibt alle Teams zurück.}{\texdocreturn{Alle existierenden Teams.}
}
\texdocmethod{public}{void}{removeAllocation}{(Allocation allocation)}{Entfernt eine Einteilung aus dem Semester.}{\begin{texdocparameters}
\texdocparameter{allocation}{Einteilung, die entfernt wird.}
\end{texdocparameters}
}
\texdocmethod{public}{void}{removeLearningGroup}{(LearningGroup learningGroup)}{Entfernt eine Lerngruppe.}{\begin{texdocparameters}
\texdocparameter{learningGroup}{Lerngruppe, die entfernt wird.}
\end{texdocparameters}
}
\texdocmethod{public}{void}{removeProject}{(Project project)}{Entfernt ein Projekt aus dem Semester.}{\begin{texdocparameters}
\texdocparameter{project}{Projekt, das entfernt wird.}
\end{texdocparameters}
}
\texdocmethod{public}{void}{removeSPO}{(SPO spo)}{Entfernt eine SPO.}{\begin{texdocparameters}
\texdocparameter{spo}{SPO, die entfernt wird.}
\end{texdocparameters}
}
\texdocmethod{public}{void}{removeStudent}{(Student student)}{Entfernt einen Studenten.}{\begin{texdocparameters}
\texdocparameter{student}{Student, der entfernt wird.}
\end{texdocparameters}
}
\texdocmethod{public}{void}{setAllocations}{(List allocations)}{Setter für die Einteilungen.}{\begin{texdocparameters}
\texdocparameter{allocations}{Die Einteilungen.}
\end{texdocparameters}
}
\texdocmethod{public}{void}{setFinalAllocation}{(Allocation finalAllocation)}{Setter für die finale Einteilung.}{\begin{texdocparameters}
\texdocparameter{finalAllocation}{Die finale Einteilung.}
\end{texdocparameters}
}
\texdocmethod{public}{void}{setInfoText}{(String infoText)}{Setter für den Infotext.}{\begin{texdocparameters}
\texdocparameter{infoText}{Der Infotext des Semesters.}
\end{texdocparameters}
}
\texdocmethod{public}{void}{setLearningGroups}{(List\textless{}LearningGroup\textgreater{} learningGroups)}{Setter für die Lerngruppen.}{\begin{texdocparameters}
\texdocparameter{learningGroups}{Die Lerngruppen, die dem Semester übergeben werden.}
\end{texdocparameters}
}
\texdocmethod{public}{void}{setName}{(String name)}{Setter für den Namen des Semesters.}{\begin{texdocparameters}
\texdocparameter{name}{Der Name des Semesters.}
\end{texdocparameters}
}
\texdocmethod{public}{void}{setProjects}{(List\textless{}Project\textgreater{} projects)}{Setter für die Projekte.}{\begin{texdocparameters}
\texdocparameter{projects}{Die Projekte, die dem Semester übergeben werden.}
\end{texdocparameters}
}
\texdocmethod{public}{void}{setRegistrationEnd}{(Date end)}{Setter für den Endzeitpunkt der Registrierung.}{\begin{texdocparameters}
\texdocparameter{start}{der Endzeitpunkt.}
\end{texdocparameters}
}
\texdocmethod{public}{void}{setRegistrationStart}{(Date start)}{Setter für den Startzeitpunkt der Registrierung.}{\begin{texdocparameters}
\texdocparameter{start}{der Startzeitpunkt.}
\end{texdocparameters}
}
\texdocmethod{public}{void}{setSpos}{(List\textless{}SPO\textgreater{} spos)}{Setter für die SPOs des Semesters.}{\begin{texdocparameters}
\texdocparameter{spos}{Die verfügbaren SPOs des Semesters.}
\end{texdocparameters}
}
\texdocmethod{public}{void}{setStudents}{(List\textless{}Student\textgreater{} students)}{Setter für die Studenten.}{\begin{texdocparameters}
\texdocparameter{students}{Studenten, die dem Semester übergeben werden.}
\end{texdocparameters}
}
\end{texdocclassmethods}
\end{texdocclass}


\begin{texdocclass}{class}{SPO}
\label{texdoclet:data.SPO}
\begin{texdocclassintro}
Diese Klasse stellt eine Studienprüfungsordnung dar.\end{texdocclassintro}
\begin{texdocclassconstructors}
\texdocconstructor{public}{SPO}{()}{}{}
\end{texdocclassconstructors}
\begin{texdocclassmethods}
\texdocmethod{public}{void}{addAdditionalAchievement}{(Achievement achievement)}{Fügt eine zusätzliche Teilleistung hinzu.}{\begin{texdocparameters}
\texdocparameter{achievement}{Teilleistung, die hinzugefügt wird.}
\end{texdocparameters}
}
\texdocmethod{public}{void}{addNecessaryAchievement}{(Achievement achievement)}{Fügt eine benötigte Teilleistung hinzu.}{\begin{texdocparameters}
\texdocparameter{achievement}{Teilleistung, die hinzugefügt wird.}
\end{texdocparameters}
}
\texdocmethod{public}{List}{getAdditionalAchievements}{()}{Getter-Methode für die zusätzlichen Teilleistungen.}{\texdocreturn{Die zusätzlichen Teilleistungen.}
}
\texdocmethod{public}{String}{getName}{()}{Getter-Methode für den Namen.}{\texdocreturn{Der Name der SPO.}
}
\texdocmethod{public}{List}{getNecessaryAchievements}{()}{Getter-Methode für die benötigten Teilleistungen.}{\texdocreturn{Die benötigten Teilleistungen.}
}
\texdocmethod{public static}{SPO}{getSPO}{(String name)}{Dies Methode gibt eine bestimmte SPO zurück, die über ihren Namen
 identifiziert wird.}{\begin{texdocparameters}
\texdocparameter{name}{Der Name der SPO.}
\end{texdocparameters}
\texdocreturn{Die SPO}
}
\texdocmethod{public static}{List\textless{}SPO\textgreater{}}{getSPOs}{()}{Diese Methode gibt alle SPOs zurück.}{\texdocreturn{Alle SPOs.}
}
\texdocmethod{public}{void}{removeAdditionalAchievement}{(Achievement achievement)}{Entfernt eine zusätzliche Teilleistung.}{\begin{texdocparameters}
\texdocparameter{achievement}{Teilleistung, die entfernt wird.}
\end{texdocparameters}
}
\texdocmethod{public}{void}{removeNecessaryAchievement}{(Achievement achievement)}{Entfernt eine benötigte Teilleistung.}{\begin{texdocparameters}
\texdocparameter{achievement}{Teilleistung, die entfernt wird.}
\end{texdocparameters}
}
\texdocmethod{public}{void}{setadditionalAchievements}{(List additionalAchievements)}{Setter-Methode für die zusätzlichen Teilleistungen.}{\begin{texdocparameters}
\texdocparameter{additionalAchievements}{Die zusätzlichen Teilleistungen.}
\end{texdocparameters}
}
\texdocmethod{public}{void}{setName}{(String name)}{Setter-Methode für den Name.}{\begin{texdocparameters}
\texdocparameter{name}{Der Name der SPO.}
\end{texdocparameters}
}
\texdocmethod{public}{void}{setNecessaryAchievements}{(List necessaryAchievements)}{Setter-Methode für die benötigten Teilleistungen.}{\begin{texdocparameters}
\texdocparameter{neccessaryAchievemens}{Die benötigten Teilleistungen.}
\end{texdocparameters}
}
\end{texdocclassmethods}
\end{texdocclass}


\begin{texdocclass}{class}{Student}
\label{texdoclet:data.Student}
\begin{texdocclassintro}
Diese Klasse stellt einen Studierenden dar, der am PSE teilnimmt.\end{texdocclassintro}
\begin{texdocclassconstructors}
\texdocconstructor{public}{Student}{()}{}{}
\end{texdocclassconstructors}
\begin{texdocclassmethods}
\texdocmethod{public}{List}{getCompletedAchievements}{()}{Getter für die abgeschlossenen Teilleistungen des Studierenden.}{\texdocreturn{Die abgeschlossenen Teilleistungen des Studierenden.}
}
\texdocmethod{public}{Project}{getCurrentProject}{()}{Diese Methode gibt das Projekt zurück, dem der Studierende zugeteilt ist.}{\texdocreturn{Das Projekt des Studierenden.}
}
\texdocmethod{public}{Team}{getCurrentTeam}{()}{Diese Methode gibt das Team, in dem der Studierende sich befindet, zurück.}{\texdocreturn{Das Team des Studierenden.}
}
\texdocmethod{public}{int}{getGradePSE}{()}{Getter für die PSE-Note des Studierenden.}{\texdocreturn{Die Note des Studierenden für das PSE.}
}
\texdocmethod{public}{int}{getGradeTSE}{()}{Getter für die TSE-Note des Studierenden.}{\texdocreturn{Die Note des Studierenden für das TSE.}
}
\texdocmethod{public}{int}{getMatriculationNumber}{()}{Getter für die Matrikelnummer.}{\texdocreturn{Die Matrikelnummer.}
}
\texdocmethod{public}{List}{getOralTestAchievement}{()}{Getter für die noch ausstehenden Teilleistungen des Studierenden.}{\texdocreturn{Die noch ausstehenden Teilleistungen des Studierenden.}
}
\texdocmethod{public}{int}{getRating}{(Project project)}{Diese Methode gibt die Bewertung des Studiereden zu einem bestimmten Projekt zurück.}{\begin{texdocparameters}
\texdocparameter{project}{Das Projekt.}
\end{texdocparameters}
\texdocreturn{Die Bewertung des Studierenden für das bestimmte Projekt.}
}
\texdocmethod{public}{int}{getSemester}{()}{Getter für das Fachsemester des Studierenden.}{\texdocreturn{Das aktuelle Fachsemester des Studierenden.}
}
\texdocmethod{public}{SPO}{getSPO}{()}{Getter für die SPO des Studierenden.}{\texdocreturn{Die SPO des Studierenden.}
}
\texdocmethod{public static}{Student}{getStudent}{(int matriculationNumber)}{Diese Methode gibt einen spezifischen Studierenden zurück, der durch seine Matrikelnummer
 identifiziert wird.}{\begin{texdocparameters}
\texdocparameter{matriculationNumber}{Die Matrikelnummer des Studierenden.}
\end{texdocparameters}
\texdocreturn{Der Studierende.}
}
\texdocmethod{public static}{List\textless{}Student\textgreater{}}{getStudents}{()}{Diese Methode gibt alle Studierenden zurück.}{\texdocreturn{Alle Studierende.}
}
\texdocmethod{public}{boolean}{isEmailVerified}{()}{Diese Methode gibt zurück, ob die E-Mail-Adresse verifiziert wurde.}{\texdocreturn{wahr, wenn die E-Mail-Adresse verifiziert wurde, falsch sonst.}
}
\texdocmethod{public}{boolean}{isRegisteredPSE}{()}{Getter für die Variable, ob der Studierende im Campus Management System für das PSE angemeldet ist.}{\texdocreturn{Wahr, wenn er angemeldet ist, falsch sonst.}
}
\texdocmethod{public}{boolean}{isRegisteredTSE}{()}{Getter für die Variable, ob der Studierende im Campus Management System für das TSE angemeldet ist.}{\texdocreturn{Wahr, wenn er angemeldet ist, falsch sonst.}
}
\texdocmethod{public}{void}{setCompletedAchievements}{(List completedAchievements)}{Setter für die abgeschlossenen Teilleistungen des Studierenden.}{\begin{texdocparameters}
\texdocparameter{completedAchievements}{Die vom Studierenden abgeschlossenen Teilleistungen.}
\end{texdocparameters}
}
\texdocmethod{public}{void}{setGradePSE}{(int gradePSE)}{Setter für die PSE-Note des Studierenden.}{\begin{texdocparameters}
\texdocparameter{Die}{Note des Studierenden fürs PSE.}
\end{texdocparameters}
}
\texdocmethod{public}{void}{setGradeTSE}{(int gradeTSE)}{Setter für die TSE-Note des Studierenden.}{\begin{texdocparameters}
\texdocparameter{gradeTSE}{Die Note des Studierenden fürs TSE.}
\end{texdocparameters}
}
\texdocmethod{public}{void}{setIsEmailVerified}{(boolean isEmailVerified)}{Diese Methode setzt per Boolean, ob die E-Mail-Adresse verifiziert wurde oder nicht.}{\begin{texdocparameters}
\texdocparameter{isEmailVerified}{wahr, wenn die E-Mail-Adresse verifiziert wurde, falsch sonst.}
\end{texdocparameters}
}
\texdocmethod{public}{void}{setMatriculationNumber}{(int matriculationNumber)}{Setter für die Matrikelnummer.}{\begin{texdocparameters}
\texdocparameter{matriculationNumber}{Die Matrikelnummer des Studierenden.}
\end{texdocparameters}
}
\texdocmethod{public}{void}{setOralTestAchievement}{(List oralTestAchievement)}{Setter für die noch ausstehenden Teilleistungen des Studierenden.}{\begin{texdocparameters}
\texdocparameter{oralTestAchievement}{Die noch aussteheneden Teilleistungen des Studierenden.}
\end{texdocparameters}
}
\texdocmethod{public}{void}{setRating}{(Project project, int rating)}{Diese Methode setzt eine Bewertung des Studierenden für ein bestimmtes Projekt.}{\begin{texdocparameters}
\texdocparameter{project}{Das zu bewertende Projekt.}
\texdocparameter{rating}{Die Bewertung des Studierenden.}
\end{texdocparameters}
}
\texdocmethod{public}{void}{setRegisteredPSE}{(boolean registeredPSE)}{Setter für die Variable, ob der Studierende im campus management system für das PSE angemeldet ist.}{\begin{texdocparameters}
\texdocparameter{registeredPSE}{Wahr, wenn der Studierende im CMS angemeldet ist, sonst false}
\end{texdocparameters}
}
\texdocmethod{public}{void}{setRegisteredTSE}{(boolean registeredTSE)}{Setter für die Variable, ob der Studierende im campus management system für das TSE angemeldet ist.}{\begin{texdocparameters}
\texdocparameter{registeredTSE}{Wahr, wenn der Studierende im CMS angemeldet ist, sonst false}
\end{texdocparameters}
}
\texdocmethod{public}{void}{setSemester}{(int semester)}{Setter für das Fachsemester des Studierenden.}{\begin{texdocparameters}
\texdocparameter{semester}{Das Fachsemester des Studierenden.}
\end{texdocparameters}
}
\texdocmethod{public}{void}{setSPO}{(SPO spo)}{Setter für die SPO des Studierenden.}{\begin{texdocparameters}
\texdocparameter{spo}{Die SPO des Studierenden.}
\end{texdocparameters}
}
\end{texdocclassmethods}
\end{texdocclass}


\begin{texdocclass}{class}{Team}
\label{texdoclet:data.Team}
\begin{texdocclassintro}
Diese KLasse stellt ein Team eines Projektes dar.\end{texdocclassintro}
\begin{texdocclassconstructors}
\texdocconstructor{public}{Team}{()}{}{}
\end{texdocclassconstructors}
\begin{texdocclassmethods}
\texdocmethod{public}{void}{addMember}{(Student member)}{Fügt einen Studierenden zum Team hinzu.}{\begin{texdocparameters}
\texdocparameter{member}{Der Studierende, der dem Team hinzugefügt wird.}
\end{texdocparameters}
}
\texdocmethod{public}{Adviser}{getAdvisers}{()}{Diese Methode gibt die Betreuer des Teams zurück.}{\texdocreturn{Die Betreuer des Teams.}
}
\texdocmethod{public}{List\textless{}Student\textgreater{}}{getMembers}{()}{Diese Methode gibt die Studierenden (Mitglieder) des Teams zurück.}{\texdocreturn{Die Mitglieder des Teams.}
}
\texdocmethod{public}{Project}{getProject}{()}{Getter für das Projekt.}{\texdocreturn{Das Projekt.}
}
\texdocmethod{public}{int}{getRating}{(Student student)}{Diese Methode gibt die Bewertung eines Studierenden zum Projekt dieses
 Teams zurück.}{\begin{texdocparameters}
\texdocparameter{student}{Der Studierende, dessen Bewertung zurückgegeben werden soll.}
\end{texdocparameters}
\texdocreturn{Die Bewertung des Studierenden.}
}
\texdocmethod{public}{void}{removeMember}{(Student member)}{Entfernt einen Studierenden aus dem Team.}{\begin{texdocparameters}
\texdocparameter{member}{Der Studierende, der aus dem Team entfernt wird.}
\end{texdocparameters}
}
\texdocmethod{public}{void}{setMembers}{(List\textless{}Student\textgreater{} members)}{Setter für die Mitglieder des Teams.}{\begin{texdocparameters}
\texdocparameter{members}{Die Mitglieder des Teams.}
\end{texdocparameters}
}
\texdocmethod{public}{void}{setProject}{(Project project)}{Setter für das Projekt.}{\begin{texdocparameters}
\texdocparameter{project}{Das Projekt.}
\end{texdocparameters}
}
\end{texdocclassmethods}
\end{texdocclass}


\begin{texdocclass}{class}{User}
\label{texdoclet:data.User}
\begin{texdocclassintro}
Diese Klasse stellt einen Benutzer der Anwendung dar.\end{texdocclassintro}
\begin{texdocclassconstructors}
\texdocconstructor{public}{User}{()}{}{}
\end{texdocclassconstructors}
\begin{texdocclassmethods}
\texdocmethod{public}{String}{getEmailAddress}{()}{Getter für die E-Mail-Addresse.}{\texdocreturn{Die E-Mail-Addresse.}
}
\texdocmethod{public}{String}{getFirstName}{()}{Getter für den Vornamen.}{\texdocreturn{Der Vorname.}
}
\texdocmethod{public}{String}{getLastName}{()}{Getter für den Nachnamen.}{\texdocreturn{Der Nachname.}
}
\texdocmethod{public}{String}{getPassword}{()}{Getter für das Benutzerpasswort.}{\texdocreturn{Das Benutzerpasswort.}
}
\texdocmethod{public}{String}{getUserName}{()}{Getter für den Benutzernamen.}{\texdocreturn{Der Benutzername.}
}
\texdocmethod{public}{void}{setEmailAddress}{(String email)}{Setter für die E-Mail-Addresse.}{\begin{texdocparameters}
\texdocparameter{email}{Die E-Mail-Adresse}
\end{texdocparameters}
}
\texdocmethod{public}{void}{setFirstName}{(String firstName)}{Setter für den Vornamen.}{\begin{texdocparameters}
\texdocparameter{firstName}{Der Vorname.}
\end{texdocparameters}
}
\texdocmethod{public}{void}{setLastName}{(String lastName)}{Setter für den Nachnamen.}{\begin{texdocparameters}
\texdocparameter{lastName}{Der Nachname.}
\end{texdocparameters}
}
\texdocmethod{public}{void}{setPassword}{(String password)}{Setter für das Benutzerpasswort.}{\begin{texdocparameters}
\texdocparameter{password}{Das Passwort}
\end{texdocparameters}
}
\texdocmethod{public}{void}{setUserName}{(String username)}{Setter für den Benutzernamen.}{\begin{texdocparameters}
\texdocparameter{username}{Der Benutzername.}
\end{texdocparameters}
}
\end{texdocclassmethods}
\end{texdocclass}


\end{texdocpackage}



