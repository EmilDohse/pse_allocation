\begin{texdocpackage}{security}
\label{texdoclet:security}

\begin{texdocclass}{class}{SecurityModule}
\label{texdoclet:security.SecurityModule}
\begin{texdocclassintro}
Das SecurityModule ist eine von der Bibliothek pac4j vorgeschriebene Klasse, welche die Bibliothek konfiguriert. 
 Darin wird festgelegt, welche Authentifizierungsmethoden verwendet werden sollen.\end{texdocclassintro}
\begin{texdocclassconstructors}
\texdocconstructor{public}{SecurityModule}{()}{}{}
\end{texdocclassconstructors}
\begin{texdocclassmethods}
\texdocmethod{public}{void}{configure}{()}{Diese Methode wird von der Bibliothek aufgerufen und 
 kreiert und konfiguriert die Authentifizierungsmethoden.}{}
\end{texdocclassmethods}
\end{texdocclass}


\begin{texdocclass}{class}{Verifier}
\label{texdoclet:security.Verifier}
\begin{texdocclassintro}
Der Verifier ist für die Verifikation von E-Mail-Adressen der Studenten
 zuständig und managt das Erstellen von Verifikations-Codes und den Prozess
 des Verifizierens an sich.
 
 Die Theorie des Verifikations-Prozesses: Der Student bekommt an die
 E-Mail-Adresse, welche verifiziert werden soll, einen Link zugeschickt, der
 den Code beinhaltet, der vorher mit dem Verifier unter Angabe des Studenten,
 wessen E-Mail-Adresse verifiziert werden soll, generiert wurde. Duch Klicken
 diese Linkes, wird die E-Mail-Adresse verifiziert, wenn der Verifier
 feststellt, dass der Code für die E-Mail-Adresse generiert wurde.\end{texdocclassintro}
\begin{texdocclassconstructors}
\texdocconstructor{public}{Verifier}{()}{}{}
\end{texdocclassconstructors}
\begin{texdocclassmethods}
\texdocmethod{public static}{Verifier}{getInstance}{()}{Diese Methode ist Teil des Singeltons. Sie gibt die einzige Instanz des
 Verifiers zurück.}{\texdocreturn{die einzige Verifier-Instanz.}
}
\texdocmethod{public}{String}{getVerificationCode}{(Student student)}{Diese Methode generiert einen Code, der dazu verwendet werden kann, um
 die E-Mail-Adresse eines Studenten zu verifizieren.}{\begin{texdocparameters}
\texdocparameter{student}{Der Student, dessen E-Mail-Adresse mit dem zurückgegebenen
            Code verifiziert werden soll.}
\end{texdocparameters}
\texdocreturn{der Code, der zur Verifikation der E-Mail-Adresse benötigt wird.}
}
\texdocmethod{public}{boolean}{verify}{(Student student, String code)}{Diese Methode verifiziert eine E-Mail-Adresse. Dazu wird geschaut, ob der
 Code auch wirklich dem Studenten zur Verifikation geschickt wurde.}{\begin{texdocparameters}
\texdocparameter{student}{Der Student, dessen E-Mail-Adresse verifiziert werden soll.}
\texdocparameter{code}{Der Code, mit dem die E-Mail-Adresse verifiziert werden soll.}
\end{texdocparameters}
\texdocreturn{true wenn die Verifikation positiv war, false sonst.}
}
\end{texdocclassmethods}
\end{texdocclass}


\end{texdocpackage}



