\begin{texdocpackage}{allocation}
\label{texdoclet:allocation}

\begin{texdocclass}{class}{AllocationQueue}
\label{texdoclet:allocation.AllocationQueue}
\begin{texdocclassintro}
Die Queue ist dazu da die Stapelverarbeitung von Einteilungs berechnungen zu realisieren\end{texdocclassintro}
\begin{texdocclassconstructors}
\texdocconstructor{public}{AllocationQueue}{()}{}{}
\end{texdocclassconstructors}
\begin{texdocclassmethods}
\texdocmethod{public}{void}{addToQueue}{(Configuration configuration)}{fügt der Berechnungsqueue ein element hinzu das dann berechnet wird}{\begin{texdocparameters}
\texdocparameter{configuration}{Die Konfiguration die zur Berechnungswarteliste hinzugefügt wird}
\end{texdocparameters}
}
\texdocmethod{public}{void}{cancelAllocation}{(Configuration configuration)}{Nimmt eine Konfiguration aus der Berechnungsqueue heraus. Falls diese Konfiguration bereits berechnet wird, wird die Berechnung abgebrochen}{\begin{texdocparameters}
\texdocparameter{configuration}{Die Konfiguration die entfernt werden soll}
\end{texdocparameters}
}
\texdocmethod{public static}{AllocationQueue}{getInstance}{()}{gibt die eine existierende Instanz der AllocationQueue (Singeltion) zurück}{\texdocreturn{die Instanz der AllocationQueue}
}
\texdocmethod{public}{List\textless{}Configuration\textgreater{}}{getQueue}{()}{Gibt die Queue der Berechnungen zurück, inklusive der Konfiguration die aktuell berechnet wird.}{\texdocreturn{queue Liste der Konfigurationen als FIFO-Queue angeordnet}
}
\end{texdocclassmethods}
\end{texdocclass}


\begin{texdocclass}{class}{Configuration}
\label{texdoclet:allocation.Configuration}
\begin{texdocclassintro}
Eine Konfiguration dient als Sammlung von Daten die zur Einteilungsberechnung benötigt werden\end{texdocclassintro}
\begin{texdocclassconstructors}
\texdocconstructor{public}{Configuration}{()}{}{}
\end{texdocclassconstructors}
\begin{texdocclassmethods}
\texdocmethod{public}{String}{getName}{()}{getter für den Einteilungsname}{}
\end{texdocclassmethods}
\end{texdocclass}


\begin{texdocclass}{interface}{Criterion}
\label{texdoclet:allocation.Criterion}
\begin{texdocclassintro}
Ein Kriterium ist dazu da den Optimierungsterm des ILP-Modells zu erweitern\end{texdocclassintro}
\begin{texdocclassmethods}
\texdocmethod{public}{String}{getName}{()}{getter für den Namen des Kriteriums}{\texdocreturn{gibt den Namen zurück}
}
\texdocmethod{public}{void}{useCriteria}{(int weight, GurobiAllocator allocator)}{bildet den Optimierungsterm und fügt ihn dem GurobiAllocator hinzu}{\begin{texdocparameters}
\texdocparameter{weight}{Der vom Admin eingestellte Parameter dieses Kriteriums}
\texdocparameter{allocator}{Die Allocator instanz welche dieses Kriterium verwenden soll}
\end{texdocparameters}
}
\end{texdocclassmethods}
\end{texdocclass}


\begin{texdocclass}{class}{CriterionAdditionalPerfomances}
\label{texdoclet:allocation.CriterionAdditionalPerfomances}
\begin{texdocclassintro}
Das Kriterium sorgt dafür Studierende die mehr als die die zur Teilname am PSE
 benötigten Teilleistungen bestanden haben bevorzugt werden.\end{texdocclassintro}
\begin{texdocclassconstructors}
\texdocconstructor{public}{CriterionAdditionalPerfomances}{()}{}{}
\end{texdocclassconstructors}
\begin{texdocclassmethods}
\texdocmethod{public}{String}{getName}{()}{\texdocinheritdoc{allocation.Criterion}{getter für den Namen des Kriteriums}}{}
\texdocmethod{public}{void}{useCriteria}{(int weight, GurobiAllocator allocator)}{\texdocinheritdoc{allocation.Criterion}{bildet den Optimierungsterm und fügt ihn dem GurobiAllocator hinzu}}{}
\end{texdocclassmethods}
\end{texdocclass}


\begin{texdocclass}{class}{CriterionAllocated}
\label{texdoclet:allocation.CriterionAllocated}
\begin{texdocclassintro}
Das Kriterium sorgt dafür das möglichst viele Studierende Teams zugeteilt
 werden.\end{texdocclassintro}
\begin{texdocclassconstructors}
\texdocconstructor{public}{CriterionAllocated}{()}{}{}
\end{texdocclassconstructors}
\begin{texdocclassmethods}
\texdocmethod{public}{String}{getName}{()}{\texdocinheritdoc{allocation.Criterion}{getter für den Namen des Kriteriums}}{}
\texdocmethod{public}{void}{useCriteria}{(int weight, GurobiAllocator allocator)}{\texdocinheritdoc{allocation.Criterion}{bildet den Optimierungsterm und fügt ihn dem GurobiAllocator hinzu}}{}
\end{texdocclassmethods}
\end{texdocclass}


\begin{texdocclass}{class}{CriterionLearningGroup}
\label{texdoclet:allocation.CriterionLearningGroup}
\begin{texdocclassintro}
Das Kriterium sorgt dafür das Lerngruppen eher zusammenbleiben.\end{texdocclassintro}
\begin{texdocclassconstructors}
\texdocconstructor{public}{CriterionLearningGroup}{()}{}{}
\end{texdocclassconstructors}
\begin{texdocclassmethods}
\texdocmethod{public}{String}{getName}{()}{\texdocinheritdoc{allocation.Criterion}{getter für den Namen des Kriteriums}}{}
\texdocmethod{public}{void}{useCriteria}{(int weight, GurobiAllocator allocator)}{\texdocinheritdoc{allocation.Criterion}{bildet den Optimierungsterm und fügt ihn dem GurobiAllocator hinzu}}{}
\end{texdocclassmethods}
\end{texdocclass}


\begin{texdocclass}{class}{CriterionNoSingularStudent}
\label{texdoclet:allocation.CriterionNoSingularStudent}
\begin{texdocclassintro}
Das Kriterium sorgt dafür das möglichst kein Team aus einer Lerngrupe sowie
 einem einzelnen Studierenden besteht.\end{texdocclassintro}
\begin{texdocclassconstructors}
\texdocconstructor{public}{CriterionNoSingularStudent}{()}{}{}
\end{texdocclassconstructors}
\begin{texdocclassmethods}
\texdocmethod{public}{String}{getName}{()}{\texdocinheritdoc{allocation.Criterion}{getter für den Namen des Kriteriums}}{}
\texdocmethod{public}{void}{useCriteria}{(int weight, GurobiAllocator allocator)}{\texdocinheritdoc{allocation.Criterion}{bildet den Optimierungsterm und fügt ihn dem GurobiAllocator hinzu}}{}
\end{texdocclassmethods}
\end{texdocclass}


\begin{texdocclass}{class}{CriterionPreferHigherSemester}
\label{texdoclet:allocation.CriterionPreferHigherSemester}
\begin{texdocclassintro}
Das Kriterium sorgt dafür das Studierende höheren Semesters bevorzugt werden.\end{texdocclassintro}
\begin{texdocclassconstructors}
\texdocconstructor{public}{CriterionPreferHigherSemester}{()}{}{}
\end{texdocclassconstructors}
\begin{texdocclassmethods}
\texdocmethod{public}{String}{getName}{()}{\texdocinheritdoc{allocation.Criterion}{getter für den Namen des Kriteriums}}{}
\texdocmethod{public}{void}{useCriteria}{(int weight, GurobiAllocator allocator)}{\texdocinheritdoc{allocation.Criterion}{bildet den Optimierungsterm und fügt ihn dem GurobiAllocator hinzu}}{}
\end{texdocclassmethods}
\end{texdocclass}


\begin{texdocclass}{class}{CriterionPreferredTeamSize}
\label{texdoclet:allocation.CriterionPreferredTeamSize}
\begin{texdocclassintro}
Das Kriterium sorgt dafür das Teams möglichst die gewünschte Teamgröße haben.\end{texdocclassintro}
\begin{texdocclassconstructors}
\texdocconstructor{public}{CriterionPreferredTeamSize}{()}{}{}
\end{texdocclassconstructors}
\begin{texdocclassmethods}
\texdocmethod{public}{String}{getName}{()}{\texdocinheritdoc{allocation.Criterion}{getter für den Namen des Kriteriums}}{}
\texdocmethod{public}{void}{useCriteria}{(int weight, GurobiAllocator allocator)}{\texdocinheritdoc{allocation.Criterion}{bildet den Optimierungsterm und fügt ihn dem GurobiAllocator hinzu}}{}
\end{texdocclassmethods}
\end{texdocclass}


\begin{texdocclass}{class}{CriterionRating}
\label{texdoclet:allocation.CriterionRating}
\begin{texdocclassintro}
Das Kriterium sorgt dafür das die Bewertungen der Studierenden berücksichtigt
 werden.\end{texdocclassintro}
\begin{texdocclassconstructors}
\texdocconstructor{public}{CriterionRating}{()}{}{}
\end{texdocclassconstructors}
\begin{texdocclassmethods}
\texdocmethod{public}{String}{getName}{()}{\texdocinheritdoc{allocation.Criterion}{getter für den Namen des Kriteriums}}{}
\texdocmethod{public}{void}{useCriteria}{(int weight, GurobiAllocator allocator)}{\texdocinheritdoc{allocation.Criterion}{bildet den Optimierungsterm und fügt ihn dem GurobiAllocator hinzu}}{}
\end{texdocclassmethods}
\end{texdocclass}


\begin{texdocclass}{class}{CriterionRegisteredAgain}
\label{texdoclet:allocation.CriterionRegisteredAgain}
\begin{texdocclassintro}
Das Kriterium sorgt dafür das Studierenden die sich schon einmal für einen PSE
 Platz beworben haben bevorzugt werden.\end{texdocclassintro}
\begin{texdocclassconstructors}
\texdocconstructor{public}{CriterionRegisteredAgain}{()}{}{}
\end{texdocclassconstructors}
\begin{texdocclassmethods}
\texdocmethod{public}{String}{getName}{()}{\texdocinheritdoc{allocation.Criterion}{getter für den Namen des Kriteriums}}{}
\texdocmethod{public}{void}{useCriteria}{(int weight, GurobiAllocator allocator)}{\texdocinheritdoc{allocation.Criterion}{bildet den Optimierungsterm und fügt ihn dem GurobiAllocator hinzu}}{}
\end{texdocclassmethods}
\end{texdocclass}


\begin{texdocclass}{class}{CriterionSameSemester}
\label{texdoclet:allocation.CriterionSameSemester}
\begin{texdocclassintro}
Das Kriterium sorgt dafür das Studierende des, für das PSE vorgesehenen
 Semesters und Studierende höherer Semester eher nicht in das selbe Team kommen.\end{texdocclassintro}
\begin{texdocclassconstructors}
\texdocconstructor{public}{CriterionSameSemester}{()}{}{}
\end{texdocclassconstructors}
\begin{texdocclassmethods}
\texdocmethod{public}{String}{getName}{()}{\texdocinheritdoc{allocation.Criterion}{getter für den Namen des Kriteriums}}{}
\texdocmethod{public}{void}{useCriteria}{(int weight, GurobiAllocator allocator)}{\texdocinheritdoc{allocation.Criterion}{bildet den Optimierungsterm und fügt ihn dem GurobiAllocator hinzu}}{}
\end{texdocclassmethods}
\end{texdocclass}


\begin{texdocclass}{class}{GurobiAllocator}
\label{texdoclet:allocation.GurobiAllocator}
\begin{texdocclassintro}
Der Gurobi Allocator dient zur Berechnung einer Einteilung mit dem ILP Solvver Gurobi. Weiterhin stellt er ein Basismodell und einen Optimierungsterm zur verfügung, welche von den Kriterien verwendet werden.\end{texdocclassintro}
\begin{texdocclassconstructors}
\texdocconstructor{public}{GurobiAllocator}{()}{}{}
\end{texdocclassconstructors}
\begin{texdocclassmethods}
\texdocmethod{public}{void}{calculate}{(Configuration configuration)}{startet die Berechnung der Einteilung}{\begin{texdocparameters}
\texdocparameter{configuration}{unter berücksichtigung dieser Konfiguration}
\end{texdocparameters}
}
\texdocmethod{public}{GRBVar}{getBasicMatrix}{()}{getter für die Basismatrix}{\texdocreturn{die Basismatrix}
}
\texdocmethod{public}{GRBModel}{getModel}{()}{getter für das Model}{\texdocreturn{das Model}
}
\texdocmethod{public}{GRBLinExpr}{getOptimizationTerm}{()}{getter für den Optimierungsterm}{\texdocreturn{der Optimierungsterm}
}
\end{texdocclassmethods}
\end{texdocclass}


\end{texdocpackage}



