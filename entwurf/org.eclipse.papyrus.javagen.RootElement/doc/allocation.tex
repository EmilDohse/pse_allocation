\begin{texdocpackage}{allocation}
\label{texdoclet:allocation}

\begin{texdocclass}{class}{AllocationQueue}
\label{texdoclet:allocation.AllocationQueue}
\begin{texdocclassintro}
Die AllocationQueue dient dazu, die Berechnung von Einteilungen als FIFO-Warteschlange zu realisieren. 
 Wenn eine berechnung fertig ist wird die nächste Berechnung angestoßen. Die Queue ist als Singelton implementiert.\end{texdocclassintro}
\begin{texdocclassmethods}
\texdocmethod{public}{void}{addToQueue}{(Configuration configuration)}{Fügt der Berechnungsqueue eine Konfiguration hinzu, die zur Berechnung verwendet werden soll.}{\begin{texdocparameters}
\texdocparameter{configuration}{Die Konfiguration, die zur Berechnungsqueue hinzugefügt wird.}
\end{texdocparameters}
}
\texdocmethod{public}{void}{cancelAllocation}{(Configuration configuration)}{Nimmt eine Konfiguration aus der Berechnungsqueue heraus. Falls diese Konfiguration bereits berechnet wird, wird die Berechnung abgebrochen.}{\begin{texdocparameters}
\texdocparameter{configuration}{Die Konfiguration, die entfernt werden soll.}
\end{texdocparameters}
}
\texdocmethod{public static}{AllocationQueue}{getInstance}{()}{Gibt die eine existierende Instanz der AllocationQueue (Singleton) zurück.	 *}{\texdocreturn{Die Instanz der AllocationQueue.}
}
\texdocmethod{public}{List\textless{}Configuration\textgreater{}}{getQueue}{()}{Gibt die Queue der Berechnungen zurück, inklusive der Konfiguration die aktuell berechnet wird.}{\texdocreturn{Liste der Konfigurationen als FIFO-Queue angeordnet.}
}
\end{texdocclassmethods}
\end{texdocclass}


\begin{texdocclass}{class}{Configuration}
\label{texdoclet:allocation.Configuration}
\begin{texdocclassintro}
Eine Konfiguration dient als Sammlung von Daten, die zur
 Einteilungsberechnung benötigt werden.\end{texdocclassintro}
\begin{texdocclassconstructors}
\texdocconstructor{public}{Configuration}{(String allocationName, Student students, LearningGroup learningGroups, Project projects, List parameters)}{Konstruktor, der alle Arrays als Parameter entgegen nimmt.}{\begin{texdocparameters}
\texdocparameter{allocationName}{Der Name der Einteilung, die berechnet werden soll.}
\texdocparameter{students}{Array von Studenten, die eingeteilt werden sollen.}
\texdocparameter{learningGroups}{Liste von Lerngruppen, die zugeteilt werden sollen.}
\texdocparameter{projects}{Liste von Projekten, denen Studenten zugeteilt werden sollen.}
\texdocparameter{parameters}{Liste von Parametern, die der Admin eingestellt hat.}
\end{texdocparameters}
}
\end{texdocclassconstructors}
\begin{texdocclassmethods}
\texdocmethod{public}{LearningGroup}{getLearningGroups}{()}{Getter für Lerngruppen.}{\texdocreturn{Array von Lerngruppen, die zugeteilt werden sollen.}
}
\texdocmethod{public}{String}{getName}{()}{Getter für den Einteilungsname.}{\texdocreturn{Der Name der Einteilung, die berechnet werden soll.}
}
\texdocmethod{public}{List}{getParameters}{()}{Getter für Kriterien-Parameter.}{\texdocreturn{Liste von Parametern, die der Admin eingegeben hat.}
}
\texdocmethod{public}{Project}{getProjects}{()}{Getter für Projekte.}{\texdocreturn{Liste von Projekten, denen Studenten zugeteilt werden sollen.}
}
\texdocmethod{public}{Student}{getStudents}{()}{Getter für Studenten.}{\texdocreturn{Array von Studenten, die eingeteilt werden sollen.}
}
\end{texdocclassmethods}
\end{texdocclass}


\begin{texdocclass}{interface}{Criterion}
\label{texdoclet:allocation.Criterion}
\begin{texdocclassintro}
Ein Kriterium ist dazu da den Optimierungsterm des ILP-Modells zu erweitern und gegebenenfalls zusätzliche Constraints hinzuzufügen.\end{texdocclassintro}
\begin{texdocclassmethods}
\texdocmethod{public}{String}{getName}{()}{Getter für den Namen des Kriteriums.}{\texdocreturn{Der Name des Kriteriums.}
}
\texdocmethod{public}{void}{useCriteria}{(int weight, GurobiAllocator allocator)}{Bildet den Optimierungsterm und fügt ihn dem GurobiAllocator hinzu.}{\begin{texdocparameters}
\texdocparameter{weight}{Der vom Admin eingestellte Parameter dieses Kriteriums.}
\texdocparameter{allocator}{Die Allocator-Instanz welche dieses Kriterium verwenden soll.}
\end{texdocparameters}
}
\end{texdocclassmethods}
\end{texdocclass}


\begin{texdocclass}{class}{CriterionAdditionalPerfomances}
\label{texdoclet:allocation.CriterionAdditionalPerfomances}
\begin{texdocclassintro}
Das Kriterium sorgt dafür, dass Studierende die mehr, als die zur Teilname am
 PSE benötigten, Teilleistungen bestanden haben bevorzugt werden.\end{texdocclassintro}
\begin{texdocclassconstructors}
\texdocconstructor{public}{CriterionAdditionalPerfomances}{()}{Standard-Konstruktor, der den Namen eindeutig setzt}{}
\end{texdocclassconstructors}
\begin{texdocclassmethods}
\texdocmethod{public}{String}{getName}{()}{\texdocinheritdoc{allocation.Criterion}{Getter für den Namen des Kriteriums.}}{}
\texdocmethod{public}{void}{useCriteria}{(int weight, GurobiAllocator allocator)}{\texdocinheritdoc{allocation.Criterion}{Bildet den Optimierungsterm und fügt ihn dem GurobiAllocator hinzu.}}{}
\end{texdocclassmethods}
\end{texdocclass}


\begin{texdocclass}{class}{CriterionAllocated}
\label{texdoclet:allocation.CriterionAllocated}
\begin{texdocclassintro}
Das Kriterium sorgt dafür, dass möglichst viele Studierende in Teams
 eingeteilt werden.\end{texdocclassintro}
\begin{texdocclassconstructors}
\texdocconstructor{public}{CriterionAllocated}{()}{Standard-Konstruktor, der den Namen eindeutig setzt}{}
\end{texdocclassconstructors}
\begin{texdocclassmethods}
\texdocmethod{public}{String}{getName}{()}{\texdocinheritdoc{allocation.Criterion}{Getter für den Namen des Kriteriums.}}{}
\texdocmethod{public}{void}{useCriteria}{(int weight, GurobiAllocator allocator)}{\texdocinheritdoc{allocation.Criterion}{Bildet den Optimierungsterm und fügt ihn dem GurobiAllocator hinzu.}}{}
\end{texdocclassmethods}
\end{texdocclass}


\begin{texdocclass}{class}{CriterionLearningGroup}
\label{texdoclet:allocation.CriterionLearningGroup}
\begin{texdocclassintro}
Das Kriterium sorgt dafür, dass Lerngruppen eher zusammenbleiben.\end{texdocclassintro}
\begin{texdocclassconstructors}
\texdocconstructor{public}{CriterionLearningGroup}{()}{Standard-Konstruktor, der den Namen eindeutig setzt}{}
\end{texdocclassconstructors}
\begin{texdocclassmethods}
\texdocmethod{public}{String}{getName}{()}{\texdocinheritdoc{allocation.Criterion}{Getter für den Namen des Kriteriums.}}{}
\texdocmethod{public}{void}{useCriteria}{(int weight, GurobiAllocator allocator)}{\texdocinheritdoc{allocation.Criterion}{Bildet den Optimierungsterm und fügt ihn dem GurobiAllocator hinzu.}}{}
\end{texdocclassmethods}
\end{texdocclass}


\begin{texdocclass}{class}{CriterionNoSingularStudent}
\label{texdoclet:allocation.CriterionNoSingularStudent}
\begin{texdocclassintro}
Das Kriterium sorgt dafür, dass möglichst kein Team aus einer Lerngrupe sowie
 einem einzelnen Studierenden besteht.\end{texdocclassintro}
\begin{texdocclassconstructors}
\texdocconstructor{public}{CriterionNoSingularStudent}{()}{Standard-Konstruktor, der den Namen eindeutig setzt}{}
\end{texdocclassconstructors}
\begin{texdocclassmethods}
\texdocmethod{public}{String}{getName}{()}{\texdocinheritdoc{allocation.Criterion}{Getter für den Namen des Kriteriums.}}{}
\texdocmethod{public}{void}{useCriteria}{(int weight, GurobiAllocator allocator)}{\texdocinheritdoc{allocation.Criterion}{Bildet den Optimierungsterm und fügt ihn dem GurobiAllocator hinzu.}}{}
\end{texdocclassmethods}
\end{texdocclass}


\begin{texdocclass}{class}{CriterionPreferHigherSemester}
\label{texdoclet:allocation.CriterionPreferHigherSemester}
\begin{texdocclassintro}
Das Kriterium sorgt dafür, dass Studierende höheren Semesters bevorzugt
 werden.\end{texdocclassintro}
\begin{texdocclassconstructors}
\texdocconstructor{public}{CriterionPreferHigherSemester}{()}{Standard-Konstruktor, der den Namen eindeutig setzt}{}
\end{texdocclassconstructors}
\begin{texdocclassmethods}
\texdocmethod{public}{String}{getName}{()}{\texdocinheritdoc{allocation.Criterion}{Getter für den Namen des Kriteriums.}}{}
\texdocmethod{public}{void}{useCriteria}{(int weight, GurobiAllocator allocator)}{\texdocinheritdoc{allocation.Criterion}{Bildet den Optimierungsterm und fügt ihn dem GurobiAllocator hinzu.}}{}
\end{texdocclassmethods}
\end{texdocclass}


\begin{texdocclass}{class}{CriterionPreferredTeamSize}
\label{texdoclet:allocation.CriterionPreferredTeamSize}
\begin{texdocclassintro}
Das Kriterium sorgt dafür, dass Teams möglichst die vom Admin gewünschte
 Teamgröße haben.\end{texdocclassintro}
\begin{texdocclassconstructors}
\texdocconstructor{public}{CriterionPreferredTeamSize}{()}{Standard-Konstruktor, der den Namen eindeutig setzt}{}
\end{texdocclassconstructors}
\begin{texdocclassmethods}
\texdocmethod{public}{String}{getName}{()}{\texdocinheritdoc{allocation.Criterion}{Getter für den Namen des Kriteriums.}}{}
\texdocmethod{public}{void}{useCriteria}{(int weight, GurobiAllocator allocator)}{\texdocinheritdoc{allocation.Criterion}{Bildet den Optimierungsterm und fügt ihn dem GurobiAllocator hinzu.}}{}
\end{texdocclassmethods}
\end{texdocclass}


\begin{texdocclass}{class}{CriterionRating}
\label{texdoclet:allocation.CriterionRating}
\begin{texdocclassintro}
Das Kriterium sorgt dafür, dass die Bewertungen der Studierenden
 berücksichtigt werden.\end{texdocclassintro}
\begin{texdocclassconstructors}
\texdocconstructor{public}{CriterionRating}{()}{Standard-Konstruktor, der den Namen eindeutig setzt}{}
\end{texdocclassconstructors}
\begin{texdocclassmethods}
\texdocmethod{public}{String}{getName}{()}{\texdocinheritdoc{allocation.Criterion}{Getter für den Namen des Kriteriums.}}{}
\texdocmethod{public}{void}{useCriteria}{(int weight, GurobiAllocator allocator)}{\texdocinheritdoc{allocation.Criterion}{Bildet den Optimierungsterm und fügt ihn dem GurobiAllocator hinzu.}}{}
\end{texdocclassmethods}
\end{texdocclass}


\begin{texdocclass}{class}{CriterionRegisteredAgain}
\label{texdoclet:allocation.CriterionRegisteredAgain}
\begin{texdocclassintro}
Das Kriterium sorgt dafür, dass Studierende, die sich schon einmal für einen
 PSE Platz beworben haben, bevorzugt werden.\end{texdocclassintro}
\begin{texdocclassconstructors}
\texdocconstructor{public}{CriterionRegisteredAgain}{()}{Standard-Konstruktor, der den Namen eindeutig setzt}{}
\end{texdocclassconstructors}
\begin{texdocclassmethods}
\texdocmethod{public}{String}{getName}{()}{\texdocinheritdoc{allocation.Criterion}{Getter für den Namen des Kriteriums.}}{}
\texdocmethod{public}{void}{useCriteria}{(int weight, GurobiAllocator allocator)}{\texdocinheritdoc{allocation.Criterion}{Bildet den Optimierungsterm und fügt ihn dem GurobiAllocator hinzu.}}{}
\end{texdocclassmethods}
\end{texdocclass}


\begin{texdocclass}{class}{CriterionSameSemester}
\label{texdoclet:allocation.CriterionSameSemester}
\begin{texdocclassintro}
Das Kriterium sorgt dafür, dass Studierende, des für das PSE vorgesehenen
 Semesters, und Studierende höherer Semester nicht in das selbe Team kommen.\end{texdocclassintro}
\begin{texdocclassconstructors}
\texdocconstructor{public}{CriterionSameSemester}{()}{Standard-Konstruktor, der den Namen eindeutig setzt}{}
\end{texdocclassconstructors}
\begin{texdocclassmethods}
\texdocmethod{public}{String}{getName}{()}{\texdocinheritdoc{allocation.Criterion}{Getter für den Namen des Kriteriums.}}{}
\texdocmethod{public}{void}{useCriteria}{(int weight, GurobiAllocator allocator)}{\texdocinheritdoc{allocation.Criterion}{Bildet den Optimierungsterm und fügt ihn dem GurobiAllocator hinzu.}}{}
\end{texdocclassmethods}
\end{texdocclass}


\begin{texdocclass}{class}{GurobiAllocator}
\label{texdoclet:allocation.GurobiAllocator}
\begin{texdocclassintro}
Der Gurobi Allocator dient zur Berechnung einer Einteilung mit dem ILP
 Solver Gurobi. Weiterhin stellt er ein Basismodell und einen
 Optimierungsterm zur Verfügung, welche von den Kriterien verwendet werden.\end{texdocclassintro}
\begin{texdocclassconstructors}
\texdocconstructor{public}{GurobiAllocator}{()}{Konstruktor, der das Basismodell initialisiert.}{}
\end{texdocclassconstructors}
\begin{texdocclassmethods}
\texdocmethod{public}{void}{calculate}{(Configuration configuration)}{Startet die Berechnung einer Einteilung.}{\begin{texdocparameters}
\texdocparameter{configuration}{Die Konfiguration, nach der die Einteilung berechnet werden soll.}
\end{texdocparameters}
}
\texdocmethod{public}{GRBVar}{getBasicMatrix}{()}{Getter für die Basismatrix.}{\texdocreturn{Die Basismatrix}
}
\texdocmethod{public}{GRBModel}{getModel}{()}{Getter für das Modell.}{\texdocreturn{Das Modell}
}
\texdocmethod{public}{GRBLinExpr}{getOptimizationTerm}{()}{Getter für den Optimierungsterm.}{\texdocreturn{Der Optimierungsterm}
}
\texdocmethod{public}{GRBVar}{getTeamSizes}{()}{Getter für die, über Constraints bestimmten, Teamgrößen}{\texdocreturn{Array von Teamgrößen}
}
\end{texdocclassmethods}
\end{texdocclass}


\end{texdocpackage}



