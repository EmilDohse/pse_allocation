%%This is a very basic article template.
%%There is just one section and two subsections.
\documentclass[parskip=full]{scrartcl}

\usepackage{studarbeit}
\usepackage[binary-units=true]{siunitx}
  \usepackage[autostyle=true,german=quotes]{csquotes}

\begin{document}

\title{Elipse -- Einteilungs Interface für das PSE}
\author{D. Biester, E. Dohse, P. Faller, P. Loth, L. Seufert, S. Kopmann}
\thesistype{Entwurfsdokument}
\zweitgutachter{}
\betreuer{Dipl.-Inform.~Andreas~Zwinkau, M.Sc.~Andreas~Fried}
\coverimage{ElipseLogo.png}

\tableofcontents
\pagebreak


\section{Änderungen zum Pflichtenheft}

\section{Sequenzdiagramme}

\section{Klassendiagramme}

\section{ILP}
Zur berechnung einer Einteilung wird ein ILP verwendet. Hierbei ist das ILP
aufgeteilt in ein Basismodell und einteilungskriterien.
\subsection{Basismodell}
Das Basismodell besteht aus einer N x M Matrix B binärer Variablen. Hierbei ist
N die anzahl der Studierenden die die Pflichtvoraussetzungen für das PSE erfüllt
haben. M ist die Anzahl der Teams plus einem Team
der nicht zugeteilten. Der Matrixeintrag
%TODO fallunterscheidung Student s in team t
Außerdem werden hier die folgenden Basisconstraints definiert:


%TODO formeln zu constraints
 \begin{itemize}
   \item Jeder Student kann maximal in einem Team sein
   \item Die Teamgröße ist größergleich der vorgegebenen minimalen Teamgröße
   oder 0
   \item Die Teamgröße ist kleinergleich der vorgegebenen maximalen Teamgröße
 \end{itemize}
 
 Es gibt weiterhin einen Optimierungsterm der durch, in der für die Berechnung
 relevanten Konfiguration gegebenen Kriterien erweitert wird. Dieser wird leer
 initialisiert.


\subsection{Kriterien}
Die Kriterien dienen dazu, den im Basismodell enthaltenen Optimierungsterm zu
erweitern. Dabei werden die eingestellten Parameter als Multiplikatoren für
die Boni der Kriterien. Diese Boni sind an der ++ Wertung normiert.

\subsubsection{CriterionAllocated}
Das Kriterium sorgt dafür das möglichst viele Studierende Teams zugeteilt
werden. 
\subsubsection{CriterionRating}
Das Kriterium sorgt dafür das die Bewertungen der Studierenden berücksichtigt
werden.
\subsubsection{CriterionLearningGroup}
Das Kriterium sorgt dafür das Lerngruppen eher zusammenbleiben.
\subsubsection{CriterionRegisteredAgain}
Das Kriterium sorgt dafür das Studierenden die sich schon einmal für einen PSE
Platz beworben haben bevorzugt werden.
\subsubsection{CriterionPreferredTeamSize}
Das Kriterium sorgt dafür das Teams möglichst die gewünschte Teamgröße haben.
\subsubsection{CriterionSameSemester}
Das Kriterium sorgt dafür das Studierende des gleichen Semesters im selben Team
landen.
\subsubsection{CriterionPreferHigherSemester}
Das Kriterium sorgt dafür das Studierende höheren Semesters bevorzugt werden
\subsubsection{CriterionAdditionalPerformances}
Das Kriterium sorgt dafür Studierende die mehr als die die zur Teilname am PSE
benötigten Teilleistungen bestanden haben bevorzugt werden.
\subsubsection{CriterionNoSingularStudent}
Das Kriterium sorgt dafür das möglichst kein Team aus einer Lerngrupe sowie
einem einzelnen Studierenden besteht.

\end{document}
