%%This is a very basic article template.
%%There is just one section and two subsections.
\documentclass[parskip=full]{scrartcl}
\usepackage[table]{xcolor}

\usepackage{amsmath}
\usepackage{amsfonts}
\usepackage{mathtools}
\usepackage{studarbeit}
\usepackage{graphicx}
\usepackage{wrapfig}
\usepackage{lscape}
\usepackage{rotating}
\usepackage{epstopdf}
\usepackage{pdfpages}
\usepackage{caption, booktabs}
\usepackage{tabularx}
\usepackage{multirow}
\usepackage{here}
\usepackage[most]{tcolorbox}
\usepackage[binary-units=true]{siunitx}
 \usepackage[autostyle=true,german=quotes]{csquotes}
 \usepackage{longtable, booktabs}
 \newcommand{\swtLabel}[1]{\textbf{/#1\arabic*0/}}
 \renewcommand{\labelitemii}{~}
 %\renewcommand{\labelitemi}{~}
 \newcommand{\fehler}[4]{\textbf{#1}
 							\begin{itemize}
 							  \item \textbf{Synthom}  #2
 							  \item \textbf{Grund} #3
 							  \item \textbf{Behandlung} #4
 							\end{itemize}}
 	\newcommand{\verbesserung}[3]{\textbf{#1}
 							\begin{itemize}
 							  
 							  \item \textbf{Grund} #2
 							  \item \textbf{Behandlung} #3
 							\end{itemize}}
 							
  \newcommand{\code}[1]{{\ttfamily #1}}
 \begin{document}

\title{Elipse -- Einteilungs Interface für das PSE}
\author{D. Biester, E. Dohse, P. Faller, P. Loth, L. Seufert, S. Kopmann}
\thesistype{Testbericht}
\zweitgutachter{}
\betreuer{Dipl.-Inform.~Andreas~Zwinkau, M.Sc.~Andreas~Fried}
\coverimage{ElipseLogo.png}
\mytitlepage
{\setlength{\textheight}{297mm}
\tableofcontents

\setlength{\textheight}{297mm}}
\pagebreak

\section{Einleitung}
Nachdem in Planungs-, Entwurfs- und Implementierungsphase das Produkt
beschrieben, entworfen und implementiert wurde ist die Qualitätssicherungsphase
dazu da, das Produkt zu Testen und die Benutzbarkeit, Zuverlässigkeit, 
Sicherheit und Performanz zu verbessern. Im folgenden Dokument werden die
behobenen Fehler sowie sonstige Verbesserungen festgehalten.

\section{Behobene Fehler}
\subsection{Concurrency}
\begin{itemize}
  \item \fehler{Projektbetreuer löschen und: Projektbetreuer zu Projekt
  hinzufügen}{Wenn der Administrator einen Projektbetreuer in einem Tab seines Browsers löscht, in einem weiteren
  Tab jedoch noch ein Projekt bearbeitet wird der Betreuer in diesem Tab
  noch angezeigt. Wenn der Betreuer zu dem Projekt hinzugefügt werden soll und
  das Projekt gespeichert wird kommt es zu einer
  \code{null}-Pointer-Exception.}{Da das andere Tab nicht neu geladen wird wird
  der Betreuer hier noch angezeigt, existiert jedoch nicht mehr. }{In der
  Projektbearbeitung wird nun überprüft ob der Projektbetreuer gelöscht wurde
  und das Löschen und Editieren von Projekten ist synchronisiert. }
    \item \fehler{SPO löschen und: SPO exportieren/bearbeiten oder Semester
    bearbeiten }{Wenn der Administrator eine SPO in einem Tab
    seines Browsers löscht, in einem weiteren Tab jedoch noch die SPO
    exportiert/bearbeitet oder einem Semester diese SPO hinzufügt wird der
    Betreuer in diesem Tab noch angezeigt.
    Wenn der Betreuer zu dem Projekt hinzugefügt werden soll und das Projekt gespeichert wird kommt es zu einer
  \code{null}-Pointer-Exception.}{Da das andere Tab nicht neu geladen wird wird
  der Betreuer hier noch angezeigt, existiert jedoch nicht mehr. }{In der
  Projektbearbeitung wird nun überprüft ob der Projektbetreuer gelöscht wurde
  und das Löschen und Editieren von Projekten ist synchronisiert. }
\end{itemize}


\section{Sonstige Verbesserungen}

\begin{itemize}
\item
\end{itemize}


\end{document}