%% LaTeX-Beamer template for KIT design
%% by Erik Burger, Christian Hammer
%% title picture by Klaus Krogmann
%%
%% version 2.1
%%
%% mostly compatible to KIT corporate design v2.0
%% http://intranet.kit.edu/gestaltungsrichtlinien.php
%%
%% Problems, bugs and comments to
%% burger@kit.edu

\documentclass[18pt]{beamer}

%% SLIDE FORMAT

% use 'beamerthemekit' for standard 4:3 ratio
% for widescreen slides (16:9), use 'beamerthemekitwide'
\usepackage[utf8]{inputenc}
\usepackage[T1]{fontenc}
\usepackage{templates/beamerthemekit}
% \usepackage{templates/beamerthemekitwide}

%% TITLE PICTURE

% if a custom picture is to be used on the title page, copy it into the 'logos'
% directory, in the line below, replace 'mypicture' with the 
% filename (without extension) and uncomment the following line
% (picture proportions: 63 : 20 for standard, 169 : 40 for wide
% *.eps format if you use latex+dvips+ps2pdf, 
% *.jpg/*.png/*.pdf if you use pdflatex)

%\titleimage{mypicture}

%% TITLE LOGO

% for a custom logo on the front page, copy your file into the 'logos'
% directory, insert the filename in the line below and uncomment it

\titlelogo{ElipseLogo}

% (*.eps format if you use latex+dvips+ps2pdf,
% *.jpg/*.png/*.pdf if you use pdflatex)

%% TikZ INTEGRATION

% use these packages for PCM symbols and UML classes
% \usepackage{templates/tikzkit}
% \usepackage{templates/tikzuml}

% the presentation starts here

\title[Elipse]{Elipse -- Einteilungs Interface für das PSE}
\subtitle{Abschlusspräsentation}
\author{D. Biester, E. Dohse, P. Faller, P. Loth, L. Seufert, S. Kopmann}
\institute{IPD Snelting}

% Bibliography

\usepackage[citestyle=authoryear,bibstyle=numeric,hyperref,backend=biber]{biblatex}
\addbibresource{templates/example.bib}
\bibhang1em

\begin{document}

% change the following line to "ngerman" for German style date and logos
\selectlanguage{ngerman}

%title page
\begin{frame}
\titlepage
\end{frame}

%table of contents
% \begin{frame}{Gliederung}
% \tableofcontents
% \end{frame}

\section{Motivation}
\begin{frame}{Motivation des Projektes}
\begin{itemize}
\item Programm zur Einteilung über Webinterface
\item Speziell auf PSE zugeschnitten
\item Fehler von Webinscribe verbessern
\item Adminsicht vereinfachen
\end{itemize}
\end{frame}

% \section{Ziele}
% \begin{frame}{Ziele}
% \begin{itemize}
% \item Geringere Änderungen an Studentensicht
% \item Komplette Verwaltung über ein Webinterinface
% \item Zusätzlicher Zugriff für Betreuer
% \item Bei Webinscribe fehlende Funktionen hinzufügen
% \item Erweiterbarkeit gewährleisten
% \item Berechnungszeit reduzieren
% \end{itemize}
% \end{frame}

\section{Schwierigkeiten}
% Bullet Points durch Ebean-Codebeispiel ersetzen
\begin{frame}{Schwierigkeiten während des Projektes}
\begin{itemize}
\item Übergang zwischen Model und Controller im Entwurf
\item Fehlende oder falsche Dokumentation der verwendeten Bibliotheken
und Frameworks
\begin{itemize}
\item Play
\item Pac4J
\item Ebean
\end{itemize}
\item Einbindung von Shibboleth
\item Konsistenz über mehrere Semester
\end{itemize}
\end{frame}

\section{Ergebnis}
\subsection{Statistiken}
% Zahlen durch Diagramm ersetzen
\begin{frame}{Statistiken}
\begin{itemize}
\item ??? Lines of Code \\ Davon:
\begin{itemize}
\item ??? Programmcode
\item ??? Testcode
\end{itemize}
\item ??? \% Testabdeckung
\end{itemize} 
\end{frame}

\subsection{Umgesetzte Funktionen}
\begin{frame}{Ergebnis} 
\begin{itemize}
% Keine Bullet Points, dafür Bild zu Betreuer-Ansicht und Bild mit geöffneter E-Mail
 \item Getrennte Ansichten für Studenten, Betreuer und Administrator
 \item E-Mail Benachrichtungen über finale Einteilung
 \item Verwaltung mehrerer Semester
 \item Einteilung mit genormten Parametern
 \item Vergleich von Einteilungen inklusive Gütekriterien
\end{itemize}
\end{frame}

\subsection{Was würden wir anders machen?}
% Folie mit verwendeten Tools
\begin{frame}{Was würden wir anders machen?}
 \begin{itemize}
  \item Früher über Dokumentationen informieren
 \end{itemize}

\end{frame}


\section{Live-Demo}
\begin{frame}
 \begin{center}
  \Huge Live-Demo
 \end{center}
\end{frame}

\end{document}
